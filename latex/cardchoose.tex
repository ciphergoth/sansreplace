% Copyright 2018 Google LLC
%
% Licensed under the Apache License, Version 2.0 (the "License");
% you may not use this file except in compliance with the License.
% You may obtain a copy of the License at
%
%     https://www.apache.org/licenses/LICENSE-2.0
%
% Unless required by applicable law or agreed to in writing, software
% distributed under the License is distributed on an "AS IS" BASIS,
% WITHOUT WARRANTIES OR CONDITIONS OF ANY KIND, either express or implied.
% See the License for the specific language governing permissions and
% limitations under the License.

%!BIB program = biber
%!TeX program = lualatex
%!TeX spellcheck = en-US

\documentclass[letterpaper,luatex,11pt]{article}

\usepackage{fontspec}
\usepackage[hmargin=4cm,vmargin=3cm,nohead]{geometry}
\usepackage{parskip}
\usepackage[style]{abstract}
\usepackage{titling}
\usepackage{sectsty}
\usepackage{authblk}
\usepackage{floatrow}
\usepackage[svgnames]{xcolor}
\usepackage[breaklinks,colorlinks,urlcolor=DarkBlue,linkcolor=DarkRed,citecolor=DarkGreen]{hyperref}
\usepackage[style=alphabetic,backend=biber]{biblatex}
\usepackage{amsmath}
\usepackage{amssymb}
\usepackage{algorithm}
\usepackage{algpseudocode}
\usepackage[landau,operators,probability,sets]{cryptocode}
\usepackage[outputdir=build/latex.out,cachedir=build/_minted-cardchoose]{minted}
\usepackage{pgf}

\renewcommand{\supportname}{supp}
% work around cryptocode bug
\makeatletter 
\renewcommand{\supp}[1]{\ensuremath{\operatorname{\supportname}\pc@olrk*{#1}}}
\makeatother

% https://tex.stackexchange.com/a/350287
\newcommand{\multichoose}[2]{
\left.\mathchoice
  {\left(\kern-0.48em\binom{#1}{#2}\kern-0.48em\right)}
  {\big(\kern-0.30em\binom{\smash{#1}}{\smash{#2}}\kern-0.30em\big)}
  {\left(\kern-0.30em\binom{\smash{#1}}{\smash{#2}}\kern-0.30em\right)}
  {\left(\kern-0.30em\binom{\smash{#1}}{\smash{#2}}\kern-0.30em\right)}
\right.}

\newcommand*{\defeq}{\stackrel{\text{def}}{=}}

\addbibresource{bib.bib}

\setmainfont{TeX Gyre Pagella}
\setsansfont{TeX Gyre Heros}
\setmathrm{Latin Modern Roman}
\defaultfontfeatures{}
\setmonofont{TeX Gyre Cursor}[Ligatures={NoCommon,NoRequired,NoContextual},Scale=0.9]

\pretitle{\begin{center}\LARGE\bfseries\sffamily} % sf title
\renewcommand{\abstitlestyle}{\sffamily\bfseries\centering} % sf abstract
\allsectionsfont{\sffamily} % sf sections

\raggedright
\raggedbottom

\title{A new algorithm for sampling without replacement}
\author{Paul~Crowley}
\affil{Google LLC}

\begin{document}
\maketitle
\begin{abstract}
    A variety of algorithms exist for fairly choosing $k$ distinct
    natural numbers below \(n\). However, they either require a
    hash-based data structure such as a set or dictionary, or show
    asymptotically poor performance for some values of $k, n$. We
    present here an algorithm that requires no such data structure or
    auxiliary storage, only an integer sort taking $\bigO{k \log k}$
    time; the algorithm is based on a method of fair multiset
    choosing. In our benchmarks the algorithm outperforms all we
    compare it to wherever \(k > 100\) and \(n > 100k\), and has
    acceptable performance for all values of \(k, n\).
\end{abstract}

\section{Introduction}

We consider the problem of choosing a subset of $\{0, 1, \ldots,
n-1\}$ such that every \(k\)-element subset is equally likely to be
returned. Many programming languages include facilities for this in
their standard libraries, for example Python's
\mintinline{python}{random.sample()} or Rust's
\mintinline{rust}{rand::seq::index::sample}. In \autoref{otheralgs} we
discuss several known algorithms to this end; these all have their
strengths, but all suffer from one of two disadvantages:
\begin{itemize}
    \item Some are fast for certain values of \(k, n\), e.g. where
    \(k\) is small or where \(\frac{n}{k}\) is small, but very slow
    for other values.
    \item Others achieve much better asymptotic performance, but at
    the cost of updating a hash-based data structure such as a set or
    dictionary for each of the \(k\) values generated, adding a
    significant constant multiplier to the overall runtime as well as
    memory cost.
\end{itemize}

We here present an algorithm which requires no storage beyond the array in which
the result is written but achieves a \(\bigO{k \log k}\) runtime.

In Python the algorithm works as follows:

\inputminted{Python}{code/cardchoose.py}

\section{Comparison algorithms}\label{otheralgs}
Let \(\NN_{<n}\) denote the \(n\)-element set \(\{0, 1, \ldots,
n-1\}\). Though the literature includes algorithms that return an
iterator, we here focus on algorithms that return an array of \(k\)
elements. Some applications require that the returned elements be in
sorted order, while others need the order of the elements to be a fair
random draw. Both Python's \mintinline{python}{random.sample()} and
Rust's \mintinline{rust}{rand::seq::index::sample} return in a fair
random order.

A routine which returns one of these can straightforwardly be
converted into a routine that returns another, using a sort or a
Fisher-Yates shuffle as appropriate. However each of these algorithms
returns one of the above types ``naturally'', and so returning a
different one will take an extra step, which means that which
algorithm is most efficient can depend on which of the above three the
caller wants to receive. In our benchmarks, we time both a ``random''
and a ``sorted'' variant of each algorithm.

\begin{center}
    \begin{tabular}{l|l|l|l}
    Algorithm & Order & Data structure & Time \\
    \hline
    Quadratic rejection sampling & Random &  & \(k^2\) \\
    Set-based rejection sampling & Random & Set & \(k\) \\
    Algorithm S & Sorted &  & \(n\) \\
    Algorithm R & Random &  & \(n\) \\
    SELECT & Random & \(n\)-element array & \(n\) \\
    HSEL & Random & Dictionary & \(k\) \\
    Floyd's F2 &  & Set & \(k\) \\
    Quadratic F2 & Random & & \(k^2\) \\
    Our work & Sorted &  & \(k \log k\)
    \end{tabular}
\end{center}

\subsection{Quadratic rejection sampling}
\inputminted{Python}{code/quadraticreject.py}
Start with an empty array of integers. Generate an integer \(0 \leq x
< n\), and check if \(x\) is already present in the array by iterating
through the array checking every element. If it is not present, append
it to the array. Repeat this process until the array is of the desired
length.

\subsection{Set-based rejection sampling}
\inputminted{Python}{code/rejectionsample.py}
As above, but use a data structure representing sets with an efficient membership test.
\cite{goodman}

\subsection{Iterative choosing (``Algorithm S'')}
\inputminted{Python}{code/iterativechoose.py}
Iterate in order through each candidate to add to the list,
calculate for each the probability it should be part of the list given
the number of items added so far and the number remaining, and add
it with that probability.
\cite{taocp2}
\cite[Method 1, p.391]{fanetal}

\subsection{Reservoir sampling (``Algorithm R'')}
\inputminted{Python}{code/reservoirsample.py}
Initialize a \(k\)-element array with the elements of \(\NN_{<k}\)
in random order. Iterate over the elements of \(\NN_{<n} \setminus \NN_{<k}\),
and replace a random element of the array with each element with the
appropriate probability.
\cite{taocp2}

\subsection{SELECT}
\inputminted{Python}{code/select.py}
Use a Fisher-Yates shuffle to randomize only the first \(k\) items of
an array of the elements of \(\NN_{<n}\). \cite{goodman}

\subsection{HSEL}
\inputminted{Python}{code/hsel.py}
As above, except that instead of allocating and initializing an
\(n\)-element array, we use a ``virtual'' array, with a hash table
storing only the modified elements.
\cite{hsel}

\subsection{Floyd's F2}
\inputminted{Python}{code/floydf2.py}
A twist on set-based rejection that uses compensatingly biased samples
to avoid multiple tests.
\cite{floydf2}

\subsection{Quadratic F2}
\inputminted{Python}{code/quadraticf2.py}
Quadratic variant on Floyd's F2 which needs no external data
structure. Inspired by the implementation in \cite{rust-random} but
with some optimizations.

\section{Notation}

Let \(\NN_{<n} \defeq \{i \in \NN: i < n\}\), the set of natural
numbers less than \(n\) where \(\NN = \{0, 1, 2, \ldots\}\). Let
\(\binom{S}{k} \defeq \{s \subseteq S: |s| = k\}\), the set of all
\(k\)-element subsets of \(S\); note that \(\left|\binom{S}{k}\right|
= \binom{|S|}{k}\).

A \emph{multiset} is an extension of a set in which elements can
appear more than once; we use brackets to delimit the elements of a
multiset, so $[0, 1, 1]$ is the same multiset as $[1, 0, 1]$ but
distinct from $[0, 0, 1]$. We represent multisets as functions $m: U
\rightarrow \NN$; the set $U$ is the \emph{universe}, and in what
follows we consider only finite universes. For any $y \in U$ we call
$m(y)$ the \emph{multiplicity} of $y$ in $m$. A multiset has a
\emph{cardinality} $|m| \defeq \sum_{x \in U} m(x)$ and a
\emph{support} set $\supp{m} \defeq \{x \in U: m(x) > 0\}$.

Where $U$ is clear from context, for any set $S \subseteq U$ we
consider $\overline{S}$ to be $S$ viewed as a multiset, i.e. the
multiset such that $\supp{\overline{S}} = S$ and $|\overline{S}| =
|S|$:
\begin{displaymath}
    \overline{S}(x) =
    \begin{cases}
        1 & \text{if $x \in S$} \\
        0 & \text{otherwise} \\
    \end{cases}
\end{displaymath}

The sum of multisets $m_1 \uplus m_2$ is the multiset such that $(m_1
\uplus m_2)(x) = m_1(x) + m_2(x)$ for all $x \in U$. $m = m_1 \ominus
m_2$ is the unique multiset such that $m_1 = m \uplus m_2$, and is
defined only if this exists.

We define random sampling from a multiset to be analogous to drawing
from a set, where each element's probability of being drawn is
proportional to its multiplicity: $\prob{x = y |x \sample m} =
\frac{m(y)}{|m|}$.

Define $\multichoose{U}{k} \defeq \{m \in U \rightarrow \NN: |m| =
k\}$ the set of multisets over universe \(U\) of cardinality \(k\),
and \(\multichoose{n}{k} \defeq
\left|\multichoose{\NN_{<n}}{k}\right|\) the number of distinct
\(k\)-cardinality multisets over an \(n\)-element universe.

\section{Intuition behind the algorithm}

Following \cite{feller}, we can represent an element of
\(\multichoose{\NN_{<7}}{8}\) such as \([0,0,0,1,5,5,5,5]\) using six
bars and eight stars:

\begin{displaymath}
    \begin{array}{ c c c c c c c c c c c c c c c }
        \star & \star & \star & \big| & \star & \big| & \big| & \big| & \big| & \star & \star & \star & \star & \big| \\
    \end{array}
\end{displaymath}

The six bars divide the line into seven ``bins'', one for each element
of \(\NN_{<7}\); the number of stars in each bin indicates the
multiplicity of that element in our multiset. With this technique, we
define a bijection between \(\multichoose{\NN_{<n}}{k}\) and
\(\binom{\NN_{<n + k - 1}}{k}\) wherever \(n > 0\), from which we
infer that \(\multichoose{n}{k} = \binom{n + k - 1}{k}\) where \(n >
0\).

Suppose now that we want to choose 6 integers from the range
$\NN_{<11}$ at random. As per the above we see that this is equivalent
to choosing a multiset from \(\multichoose{\NN_{<6}}{6}\) or a way of
arranging 6 stars and 5 bars into a sequence. To choose fairly, we
start with a sequence of five bars \(|||||\) and insert six stars, one
after another, in randomly chosen positions.

For the first star, there are six possible places it can go, and we
choose one at random: \(|||\star||\). We record that it has three bars
to its left in a list: \([3]\).

There are now six items in the sequence, and thus seven possible
places to place the second star. In two of those seven cases---before
the existing star, and after it---it will have three bars to its left.
Let's suppose we choose the first position: \(\star|||\star||\). We
append the number of bars to the left of the new star to our record,
which becomes \([3, 0]\). The positions have changed from \(\{3\}\) to
\(\{4, 0\}\) but because we're not recording positions, only bars to
the left, we don't need to update the first entry.

We place three more stars in random positions, ending up with
\(\star\star\star|\star||\star||\) and a record of \([3, 0, 0, 1,
0]\). Now there's one star left to place; there is one position it can
be placed after the last bar, but four before the first bar, so it is
four times more likely to be placed before the first bar than after
the last. Let's suppose it's placed at the fifth position:
\(\star\star\star|\star\star||\star||\), \([3, 0, 0, 1, 0, 1]\) We now
want to know the position of each star; we find this by sorting the
list \([0, 0, 0, 1, 1, 3]\) and adding to each entry its index so that
the value reflects the stars as well as the bars to its left,
returning the answer $\{0, 1, 2, 4, 5, 8\}$

\begin{displaymath}
\begin{array}{ c c c c c c c c c c c }
    0 & 0 & 0 & & 1 & 1 & & & 3 & & \\
    \star & \star & \star & \big| & \star & \star & \big| & \big| & \star & \big| & \big| \\
    0 & 1 & 2 & & 4 & 5 & & & 8 & & \\
\end{array}
\end{displaymath}

Thus to get a sequence without duplicates, we start with a procedure
that deliberately biases towards duplicates.

\section{Proof}

\subsection{Multiset choosing}

We consider the problem of choosing an element from
$\multichoose{U}{k}$ fairly. For example, $\multichoose{\NN_{<2}}{3} =
\{[0, 0, 0], [0, 0, 1], [0, 1, 1], [1, 1, 1]\}$; for each of these
four, our algorithm should output it with probability $\frac{1}{4}$.
If we choose three independent elements from $U$ and add them together
to make a multiset, our answer will favour multisets with lower
multiplicities; in accordance with the binomial theorem, $[0, 0, 0]$
will be drawn with probability $\frac{1}{8}$, while $[0, 0, 1]$ will
be drawn with probability $\frac{3}{8}$, reflecting the three ways
this multiset can be written as a sequence.

%It is straightforward to show by
%induction that in general, this naive method chooses $m \in \multichoose{U}{k}$
%with probability

%\begin{displaymath}
%    \frac{k!}{{|U|}^k \prod_{x \in U} m(x)!}
%\end{displaymath}

\begin{algorithm}
\caption{Fair multiset choosing}\label{ChooseMultiset}
\begin{algorithmic}[0]
\Procedure{ChooseMultiset}{$U, k$}
    \If{k = 0}
        \State \textbf{return} $\overline{\varnothing}$
    \Else
        \State $m' \leftarrow \textproc{ChooseMultiset}(U, k-1)$
        \State $x \sample \overline{U} \uplus m'$
        \State \textbf{return} $m' \uplus \overline{\{x\}}$
    \EndIf
\EndProcedure
\end{algorithmic}
\end{algorithm}

To address this, in \textproc{ChooseMultiset} we introduce a
counter-bias in the selection of $x$, which favours duplicates.
\textproc{ChooseMultiset} is trivially fair for $k = 0$, so we assume
it is fair for $k - 1$ and proceed by induction. For a multiset $m \in
\multichoose{U}{k}$:

\begin{align*}
    &\prob{m' = m | m' \leftarrow \textproc{ChooseMultiset}(U, k)}
    \\
    =&
    \prob{m' \uplus \overline{\{x\}} = m
        | m' \leftarrow \textproc{ChooseMultiset}(U, k-1), x \sample \overline{U} \uplus m'}
    \\
    =&
    \prob{m' \uplus \overline{\{x\}} = m
        | m' \sample \multichoose{U}{k - 1}, x \sample \overline{U} \uplus m'}
    \\
    =&
    \sum_{y \in \supp{m}}
    \prob{x = y, m' \uplus \overline{\{y\}} = m
        | m' \sample \multichoose{U}{k - 1}, x \sample \overline{U} \uplus m'}
    \\
    =&
    \sum_{y \in \supp{m}}
    \prob{m' = m \ominus \overline{\{y\}} | m' \sample \multichoose{U}{k - 1}}
    \prob{x = y | x \sample \overline{U} \uplus (m \ominus \overline{\{y\}})}
    \\
    =&
    \sum_{y \in \supp{m}}
    \frac{1}{\multichoose{|U|}{k - 1}}
    \frac{(U \uplus (m \ominus \overline{\{y\}}))(y)}{|U \uplus (m \ominus \overline{\{y\}})|}
    \\
    =&
    \frac{1}{\multichoose{|U|}{k - 1}}
    \sum_{y \in \supp{m}}
    \frac{1 + (m \ominus \overline{\{y\}})(y)}{|U| + |m \ominus \overline{\{y\}}|}
    \\
    =&
    \frac{1}{\binom{|U| + k - 2}{k-1}}
    \sum_{y \in \supp{m}}
    \frac{m(y)}{|U| + k -1}
    \\
    =& \frac{k}{(|U| + k -1)\binom{|U| + k - 2}{k-1}}
    \\
    =& \frac{1}{\binom{|U| + k - 1}{k}}
    \\
    =& \frac{1}{\multichoose{|U|}{k}}
\end{align*}

A direct translation of \autoref{ChooseMultiset} into recursive 
Python, taking integers $n, k$ and returning a list of
integers in \(\NN_{<n}\), might be:

\inputminted{Python}{code/choose_multiset_recursive.py}

This is easily converted to an iterative form:

\inputminted{Python}{code/choose_multiset.py}

\subsection{Multisets and choices}

To generate a random \(k\)-element subset of \(\NN_{<n}\), we can
apply this method to generate a random multiset from the universe
\(\NN_{<n-k + 1}\) and use the ``stars and bars'' bijection to convert
to the desired subset.

Our implementation of \textproc{ChooseMultiset} represents its result
in \(\multichoose{\NN_{<n-k + 1}}{k}\) as an unordered list of
integers. In ``stars and bars'' representation, each entry in the list
represents a star, and the integer is the number of bars to its left.
If we sort this list, then converting this to a sorted \(k\)-element
subset of \(\NN_{<n}\) simply means adding to each the number of stars
to its left, which is equal to its position in the sequence; the
Python code below returns a sorted list of $k$ distinct integers in
\(\NN_{<n}\) fairly among all ways of doing so.

\inputminted{Python}{code/choose_binom.py}

\section{Benchmarks}

(\url{https://github.com/ciphergoth/sansreplace/cpp}) includes C++
implementations of these algorithms, which were benchmarked for a
variety of values of \(n\) and \(k\), for both sorted output and
random output, on a Hewlett-Packard Z840 workstation:

\begin{itemize}
    \item CPU: Intel Xeon E5-2690 v3
    \item Frequency: 3.5GHz
    \item Cache: 768kiB L1, 3 MiB L2, 30MiB L3
    \item Compiler: gcc 12.2, \verb|-Ofast|
\end{itemize}

In \autoref{randomsmalln} and \autoref{randomlargen} we graph
time/\(k\) in ns against \(k\) for two different values of \(n\) for
random order output, and in \autoref{sortedlargen} a single value of
\(n\) for sorted output. Animated graphs showing many values of \(n\)
up to 701,408,733 (the 44th Fibonacci number) can be seen at
\url{https://github.com/ciphergoth/sansreplace/blob/master/results.md},
and complete benchmark data at
\url{https://github.com/ciphergoth/sansreplace/blob/master/results/latest}

\begin{figure}
    %% Creator: Matplotlib, PGF backend
%%
%% To include the figure in your LaTeX document, write
%%   \input{<filename>.pgf}
%%
%% Make sure the required packages are loaded in your preamble
%%   \usepackage{pgf}
%%
%% Also ensure that all the required font packages are loaded; for instance,
%% the lmodern package is sometimes necessary when using math font.
%%   \usepackage{lmodern}
%%
%% Figures using additional raster images can only be included by \input if
%% they are in the same directory as the main LaTeX file. For loading figures
%% from other directories you can use the `import` package
%%   \usepackage{import}
%%
%% and then include the figures with
%%   \import{<path to file>}{<filename>.pgf}
%%
%% Matplotlib used the following preamble
%%   
%%   \usepackage{fontspec}
%%   \makeatletter\@ifpackageloaded{underscore}{}{\usepackage[strings]{underscore}}\makeatother
%%
\begingroup%
\makeatletter%
\begin{pgfpicture}%
\pgfpathrectangle{\pgfpointorigin}{\pgfqpoint{6.400000in}{4.800000in}}%
\pgfusepath{use as bounding box, clip}%
\begin{pgfscope}%
\pgfsetbuttcap%
\pgfsetmiterjoin%
\definecolor{currentfill}{rgb}{1.000000,1.000000,1.000000}%
\pgfsetfillcolor{currentfill}%
\pgfsetlinewidth{0.000000pt}%
\definecolor{currentstroke}{rgb}{1.000000,1.000000,1.000000}%
\pgfsetstrokecolor{currentstroke}%
\pgfsetdash{}{0pt}%
\pgfpathmoveto{\pgfqpoint{0.000000in}{0.000000in}}%
\pgfpathlineto{\pgfqpoint{6.400000in}{0.000000in}}%
\pgfpathlineto{\pgfqpoint{6.400000in}{4.800000in}}%
\pgfpathlineto{\pgfqpoint{0.000000in}{4.800000in}}%
\pgfpathlineto{\pgfqpoint{0.000000in}{0.000000in}}%
\pgfpathclose%
\pgfusepath{fill}%
\end{pgfscope}%
\begin{pgfscope}%
\pgfsetbuttcap%
\pgfsetmiterjoin%
\definecolor{currentfill}{rgb}{1.000000,1.000000,1.000000}%
\pgfsetfillcolor{currentfill}%
\pgfsetlinewidth{0.000000pt}%
\definecolor{currentstroke}{rgb}{0.000000,0.000000,0.000000}%
\pgfsetstrokecolor{currentstroke}%
\pgfsetstrokeopacity{0.000000}%
\pgfsetdash{}{0pt}%
\pgfpathmoveto{\pgfqpoint{0.800000in}{0.528000in}}%
\pgfpathlineto{\pgfqpoint{5.760000in}{0.528000in}}%
\pgfpathlineto{\pgfqpoint{5.760000in}{4.224000in}}%
\pgfpathlineto{\pgfqpoint{0.800000in}{4.224000in}}%
\pgfpathlineto{\pgfqpoint{0.800000in}{0.528000in}}%
\pgfpathclose%
\pgfusepath{fill}%
\end{pgfscope}%
\begin{pgfscope}%
\pgfsetbuttcap%
\pgfsetroundjoin%
\definecolor{currentfill}{rgb}{0.000000,0.000000,0.000000}%
\pgfsetfillcolor{currentfill}%
\pgfsetlinewidth{0.803000pt}%
\definecolor{currentstroke}{rgb}{0.000000,0.000000,0.000000}%
\pgfsetstrokecolor{currentstroke}%
\pgfsetdash{}{0pt}%
\pgfsys@defobject{currentmarker}{\pgfqpoint{0.000000in}{-0.048611in}}{\pgfqpoint{0.000000in}{0.000000in}}{%
\pgfpathmoveto{\pgfqpoint{0.000000in}{0.000000in}}%
\pgfpathlineto{\pgfqpoint{0.000000in}{-0.048611in}}%
\pgfusepath{stroke,fill}%
}%
\begin{pgfscope}%
\pgfsys@transformshift{0.800000in}{0.528000in}%
\pgfsys@useobject{currentmarker}{}%
\end{pgfscope}%
\end{pgfscope}%
\begin{pgfscope}%
\definecolor{textcolor}{rgb}{0.000000,0.000000,0.000000}%
\pgfsetstrokecolor{textcolor}%
\pgfsetfillcolor{textcolor}%
\pgftext[x=0.800000in,y=0.430778in,,top]{\color{textcolor}\rmfamily\fontsize{10.000000}{12.000000}\selectfont \(\displaystyle {10^{0}}\)}%
\end{pgfscope}%
\begin{pgfscope}%
\pgfsetbuttcap%
\pgfsetroundjoin%
\definecolor{currentfill}{rgb}{0.000000,0.000000,0.000000}%
\pgfsetfillcolor{currentfill}%
\pgfsetlinewidth{0.803000pt}%
\definecolor{currentstroke}{rgb}{0.000000,0.000000,0.000000}%
\pgfsetstrokecolor{currentstroke}%
\pgfsetdash{}{0pt}%
\pgfsys@defobject{currentmarker}{\pgfqpoint{0.000000in}{-0.048611in}}{\pgfqpoint{0.000000in}{0.000000in}}{%
\pgfpathmoveto{\pgfqpoint{0.000000in}{0.000000in}}%
\pgfpathlineto{\pgfqpoint{0.000000in}{-0.048611in}}%
\pgfusepath{stroke,fill}%
}%
\begin{pgfscope}%
\pgfsys@transformshift{1.902222in}{0.528000in}%
\pgfsys@useobject{currentmarker}{}%
\end{pgfscope}%
\end{pgfscope}%
\begin{pgfscope}%
\definecolor{textcolor}{rgb}{0.000000,0.000000,0.000000}%
\pgfsetstrokecolor{textcolor}%
\pgfsetfillcolor{textcolor}%
\pgftext[x=1.902222in,y=0.430778in,,top]{\color{textcolor}\rmfamily\fontsize{10.000000}{12.000000}\selectfont \(\displaystyle {10^{2}}\)}%
\end{pgfscope}%
\begin{pgfscope}%
\pgfsetbuttcap%
\pgfsetroundjoin%
\definecolor{currentfill}{rgb}{0.000000,0.000000,0.000000}%
\pgfsetfillcolor{currentfill}%
\pgfsetlinewidth{0.803000pt}%
\definecolor{currentstroke}{rgb}{0.000000,0.000000,0.000000}%
\pgfsetstrokecolor{currentstroke}%
\pgfsetdash{}{0pt}%
\pgfsys@defobject{currentmarker}{\pgfqpoint{0.000000in}{-0.048611in}}{\pgfqpoint{0.000000in}{0.000000in}}{%
\pgfpathmoveto{\pgfqpoint{0.000000in}{0.000000in}}%
\pgfpathlineto{\pgfqpoint{0.000000in}{-0.048611in}}%
\pgfusepath{stroke,fill}%
}%
\begin{pgfscope}%
\pgfsys@transformshift{3.004444in}{0.528000in}%
\pgfsys@useobject{currentmarker}{}%
\end{pgfscope}%
\end{pgfscope}%
\begin{pgfscope}%
\definecolor{textcolor}{rgb}{0.000000,0.000000,0.000000}%
\pgfsetstrokecolor{textcolor}%
\pgfsetfillcolor{textcolor}%
\pgftext[x=3.004444in,y=0.430778in,,top]{\color{textcolor}\rmfamily\fontsize{10.000000}{12.000000}\selectfont \(\displaystyle {10^{4}}\)}%
\end{pgfscope}%
\begin{pgfscope}%
\pgfsetbuttcap%
\pgfsetroundjoin%
\definecolor{currentfill}{rgb}{0.000000,0.000000,0.000000}%
\pgfsetfillcolor{currentfill}%
\pgfsetlinewidth{0.803000pt}%
\definecolor{currentstroke}{rgb}{0.000000,0.000000,0.000000}%
\pgfsetstrokecolor{currentstroke}%
\pgfsetdash{}{0pt}%
\pgfsys@defobject{currentmarker}{\pgfqpoint{0.000000in}{-0.048611in}}{\pgfqpoint{0.000000in}{0.000000in}}{%
\pgfpathmoveto{\pgfqpoint{0.000000in}{0.000000in}}%
\pgfpathlineto{\pgfqpoint{0.000000in}{-0.048611in}}%
\pgfusepath{stroke,fill}%
}%
\begin{pgfscope}%
\pgfsys@transformshift{4.106667in}{0.528000in}%
\pgfsys@useobject{currentmarker}{}%
\end{pgfscope}%
\end{pgfscope}%
\begin{pgfscope}%
\definecolor{textcolor}{rgb}{0.000000,0.000000,0.000000}%
\pgfsetstrokecolor{textcolor}%
\pgfsetfillcolor{textcolor}%
\pgftext[x=4.106667in,y=0.430778in,,top]{\color{textcolor}\rmfamily\fontsize{10.000000}{12.000000}\selectfont \(\displaystyle {10^{6}}\)}%
\end{pgfscope}%
\begin{pgfscope}%
\pgfsetbuttcap%
\pgfsetroundjoin%
\definecolor{currentfill}{rgb}{0.000000,0.000000,0.000000}%
\pgfsetfillcolor{currentfill}%
\pgfsetlinewidth{0.803000pt}%
\definecolor{currentstroke}{rgb}{0.000000,0.000000,0.000000}%
\pgfsetstrokecolor{currentstroke}%
\pgfsetdash{}{0pt}%
\pgfsys@defobject{currentmarker}{\pgfqpoint{0.000000in}{-0.048611in}}{\pgfqpoint{0.000000in}{0.000000in}}{%
\pgfpathmoveto{\pgfqpoint{0.000000in}{0.000000in}}%
\pgfpathlineto{\pgfqpoint{0.000000in}{-0.048611in}}%
\pgfusepath{stroke,fill}%
}%
\begin{pgfscope}%
\pgfsys@transformshift{5.208889in}{0.528000in}%
\pgfsys@useobject{currentmarker}{}%
\end{pgfscope}%
\end{pgfscope}%
\begin{pgfscope}%
\definecolor{textcolor}{rgb}{0.000000,0.000000,0.000000}%
\pgfsetstrokecolor{textcolor}%
\pgfsetfillcolor{textcolor}%
\pgftext[x=5.208889in,y=0.430778in,,top]{\color{textcolor}\rmfamily\fontsize{10.000000}{12.000000}\selectfont \(\displaystyle {10^{8}}\)}%
\end{pgfscope}%
\begin{pgfscope}%
\definecolor{textcolor}{rgb}{0.000000,0.000000,0.000000}%
\pgfsetstrokecolor{textcolor}%
\pgfsetfillcolor{textcolor}%
\pgftext[x=3.280000in,y=0.251889in,,top]{\color{textcolor}\rmfamily\fontsize{10.000000}{12.000000}\selectfont k}%
\end{pgfscope}%
\begin{pgfscope}%
\pgfsetbuttcap%
\pgfsetroundjoin%
\definecolor{currentfill}{rgb}{0.000000,0.000000,0.000000}%
\pgfsetfillcolor{currentfill}%
\pgfsetlinewidth{0.803000pt}%
\definecolor{currentstroke}{rgb}{0.000000,0.000000,0.000000}%
\pgfsetstrokecolor{currentstroke}%
\pgfsetdash{}{0pt}%
\pgfsys@defobject{currentmarker}{\pgfqpoint{-0.048611in}{0.000000in}}{\pgfqpoint{-0.000000in}{0.000000in}}{%
\pgfpathmoveto{\pgfqpoint{-0.000000in}{0.000000in}}%
\pgfpathlineto{\pgfqpoint{-0.048611in}{0.000000in}}%
\pgfusepath{stroke,fill}%
}%
\begin{pgfscope}%
\pgfsys@transformshift{0.800000in}{0.528000in}%
\pgfsys@useobject{currentmarker}{}%
\end{pgfscope}%
\end{pgfscope}%
\begin{pgfscope}%
\definecolor{textcolor}{rgb}{0.000000,0.000000,0.000000}%
\pgfsetstrokecolor{textcolor}%
\pgfsetfillcolor{textcolor}%
\pgftext[x=0.633333in, y=0.479806in, left, base]{\color{textcolor}\rmfamily\fontsize{10.000000}{12.000000}\selectfont \(\displaystyle {0}\)}%
\end{pgfscope}%
\begin{pgfscope}%
\pgfsetbuttcap%
\pgfsetroundjoin%
\definecolor{currentfill}{rgb}{0.000000,0.000000,0.000000}%
\pgfsetfillcolor{currentfill}%
\pgfsetlinewidth{0.803000pt}%
\definecolor{currentstroke}{rgb}{0.000000,0.000000,0.000000}%
\pgfsetstrokecolor{currentstroke}%
\pgfsetdash{}{0pt}%
\pgfsys@defobject{currentmarker}{\pgfqpoint{-0.048611in}{0.000000in}}{\pgfqpoint{-0.000000in}{0.000000in}}{%
\pgfpathmoveto{\pgfqpoint{-0.000000in}{0.000000in}}%
\pgfpathlineto{\pgfqpoint{-0.048611in}{0.000000in}}%
\pgfusepath{stroke,fill}%
}%
\begin{pgfscope}%
\pgfsys@transformshift{0.800000in}{0.990000in}%
\pgfsys@useobject{currentmarker}{}%
\end{pgfscope}%
\end{pgfscope}%
\begin{pgfscope}%
\definecolor{textcolor}{rgb}{0.000000,0.000000,0.000000}%
\pgfsetstrokecolor{textcolor}%
\pgfsetfillcolor{textcolor}%
\pgftext[x=0.563888in, y=0.941806in, left, base]{\color{textcolor}\rmfamily\fontsize{10.000000}{12.000000}\selectfont \(\displaystyle {25}\)}%
\end{pgfscope}%
\begin{pgfscope}%
\pgfsetbuttcap%
\pgfsetroundjoin%
\definecolor{currentfill}{rgb}{0.000000,0.000000,0.000000}%
\pgfsetfillcolor{currentfill}%
\pgfsetlinewidth{0.803000pt}%
\definecolor{currentstroke}{rgb}{0.000000,0.000000,0.000000}%
\pgfsetstrokecolor{currentstroke}%
\pgfsetdash{}{0pt}%
\pgfsys@defobject{currentmarker}{\pgfqpoint{-0.048611in}{0.000000in}}{\pgfqpoint{-0.000000in}{0.000000in}}{%
\pgfpathmoveto{\pgfqpoint{-0.000000in}{0.000000in}}%
\pgfpathlineto{\pgfqpoint{-0.048611in}{0.000000in}}%
\pgfusepath{stroke,fill}%
}%
\begin{pgfscope}%
\pgfsys@transformshift{0.800000in}{1.452000in}%
\pgfsys@useobject{currentmarker}{}%
\end{pgfscope}%
\end{pgfscope}%
\begin{pgfscope}%
\definecolor{textcolor}{rgb}{0.000000,0.000000,0.000000}%
\pgfsetstrokecolor{textcolor}%
\pgfsetfillcolor{textcolor}%
\pgftext[x=0.563888in, y=1.403806in, left, base]{\color{textcolor}\rmfamily\fontsize{10.000000}{12.000000}\selectfont \(\displaystyle {50}\)}%
\end{pgfscope}%
\begin{pgfscope}%
\pgfsetbuttcap%
\pgfsetroundjoin%
\definecolor{currentfill}{rgb}{0.000000,0.000000,0.000000}%
\pgfsetfillcolor{currentfill}%
\pgfsetlinewidth{0.803000pt}%
\definecolor{currentstroke}{rgb}{0.000000,0.000000,0.000000}%
\pgfsetstrokecolor{currentstroke}%
\pgfsetdash{}{0pt}%
\pgfsys@defobject{currentmarker}{\pgfqpoint{-0.048611in}{0.000000in}}{\pgfqpoint{-0.000000in}{0.000000in}}{%
\pgfpathmoveto{\pgfqpoint{-0.000000in}{0.000000in}}%
\pgfpathlineto{\pgfqpoint{-0.048611in}{0.000000in}}%
\pgfusepath{stroke,fill}%
}%
\begin{pgfscope}%
\pgfsys@transformshift{0.800000in}{1.914000in}%
\pgfsys@useobject{currentmarker}{}%
\end{pgfscope}%
\end{pgfscope}%
\begin{pgfscope}%
\definecolor{textcolor}{rgb}{0.000000,0.000000,0.000000}%
\pgfsetstrokecolor{textcolor}%
\pgfsetfillcolor{textcolor}%
\pgftext[x=0.563888in, y=1.865806in, left, base]{\color{textcolor}\rmfamily\fontsize{10.000000}{12.000000}\selectfont \(\displaystyle {75}\)}%
\end{pgfscope}%
\begin{pgfscope}%
\pgfsetbuttcap%
\pgfsetroundjoin%
\definecolor{currentfill}{rgb}{0.000000,0.000000,0.000000}%
\pgfsetfillcolor{currentfill}%
\pgfsetlinewidth{0.803000pt}%
\definecolor{currentstroke}{rgb}{0.000000,0.000000,0.000000}%
\pgfsetstrokecolor{currentstroke}%
\pgfsetdash{}{0pt}%
\pgfsys@defobject{currentmarker}{\pgfqpoint{-0.048611in}{0.000000in}}{\pgfqpoint{-0.000000in}{0.000000in}}{%
\pgfpathmoveto{\pgfqpoint{-0.000000in}{0.000000in}}%
\pgfpathlineto{\pgfqpoint{-0.048611in}{0.000000in}}%
\pgfusepath{stroke,fill}%
}%
\begin{pgfscope}%
\pgfsys@transformshift{0.800000in}{2.376000in}%
\pgfsys@useobject{currentmarker}{}%
\end{pgfscope}%
\end{pgfscope}%
\begin{pgfscope}%
\definecolor{textcolor}{rgb}{0.000000,0.000000,0.000000}%
\pgfsetstrokecolor{textcolor}%
\pgfsetfillcolor{textcolor}%
\pgftext[x=0.494444in, y=2.327806in, left, base]{\color{textcolor}\rmfamily\fontsize{10.000000}{12.000000}\selectfont \(\displaystyle {100}\)}%
\end{pgfscope}%
\begin{pgfscope}%
\pgfsetbuttcap%
\pgfsetroundjoin%
\definecolor{currentfill}{rgb}{0.000000,0.000000,0.000000}%
\pgfsetfillcolor{currentfill}%
\pgfsetlinewidth{0.803000pt}%
\definecolor{currentstroke}{rgb}{0.000000,0.000000,0.000000}%
\pgfsetstrokecolor{currentstroke}%
\pgfsetdash{}{0pt}%
\pgfsys@defobject{currentmarker}{\pgfqpoint{-0.048611in}{0.000000in}}{\pgfqpoint{-0.000000in}{0.000000in}}{%
\pgfpathmoveto{\pgfqpoint{-0.000000in}{0.000000in}}%
\pgfpathlineto{\pgfqpoint{-0.048611in}{0.000000in}}%
\pgfusepath{stroke,fill}%
}%
\begin{pgfscope}%
\pgfsys@transformshift{0.800000in}{2.838000in}%
\pgfsys@useobject{currentmarker}{}%
\end{pgfscope}%
\end{pgfscope}%
\begin{pgfscope}%
\definecolor{textcolor}{rgb}{0.000000,0.000000,0.000000}%
\pgfsetstrokecolor{textcolor}%
\pgfsetfillcolor{textcolor}%
\pgftext[x=0.494444in, y=2.789806in, left, base]{\color{textcolor}\rmfamily\fontsize{10.000000}{12.000000}\selectfont \(\displaystyle {125}\)}%
\end{pgfscope}%
\begin{pgfscope}%
\pgfsetbuttcap%
\pgfsetroundjoin%
\definecolor{currentfill}{rgb}{0.000000,0.000000,0.000000}%
\pgfsetfillcolor{currentfill}%
\pgfsetlinewidth{0.803000pt}%
\definecolor{currentstroke}{rgb}{0.000000,0.000000,0.000000}%
\pgfsetstrokecolor{currentstroke}%
\pgfsetdash{}{0pt}%
\pgfsys@defobject{currentmarker}{\pgfqpoint{-0.048611in}{0.000000in}}{\pgfqpoint{-0.000000in}{0.000000in}}{%
\pgfpathmoveto{\pgfqpoint{-0.000000in}{0.000000in}}%
\pgfpathlineto{\pgfqpoint{-0.048611in}{0.000000in}}%
\pgfusepath{stroke,fill}%
}%
\begin{pgfscope}%
\pgfsys@transformshift{0.800000in}{3.300000in}%
\pgfsys@useobject{currentmarker}{}%
\end{pgfscope}%
\end{pgfscope}%
\begin{pgfscope}%
\definecolor{textcolor}{rgb}{0.000000,0.000000,0.000000}%
\pgfsetstrokecolor{textcolor}%
\pgfsetfillcolor{textcolor}%
\pgftext[x=0.494444in, y=3.251806in, left, base]{\color{textcolor}\rmfamily\fontsize{10.000000}{12.000000}\selectfont \(\displaystyle {150}\)}%
\end{pgfscope}%
\begin{pgfscope}%
\pgfsetbuttcap%
\pgfsetroundjoin%
\definecolor{currentfill}{rgb}{0.000000,0.000000,0.000000}%
\pgfsetfillcolor{currentfill}%
\pgfsetlinewidth{0.803000pt}%
\definecolor{currentstroke}{rgb}{0.000000,0.000000,0.000000}%
\pgfsetstrokecolor{currentstroke}%
\pgfsetdash{}{0pt}%
\pgfsys@defobject{currentmarker}{\pgfqpoint{-0.048611in}{0.000000in}}{\pgfqpoint{-0.000000in}{0.000000in}}{%
\pgfpathmoveto{\pgfqpoint{-0.000000in}{0.000000in}}%
\pgfpathlineto{\pgfqpoint{-0.048611in}{0.000000in}}%
\pgfusepath{stroke,fill}%
}%
\begin{pgfscope}%
\pgfsys@transformshift{0.800000in}{3.762000in}%
\pgfsys@useobject{currentmarker}{}%
\end{pgfscope}%
\end{pgfscope}%
\begin{pgfscope}%
\definecolor{textcolor}{rgb}{0.000000,0.000000,0.000000}%
\pgfsetstrokecolor{textcolor}%
\pgfsetfillcolor{textcolor}%
\pgftext[x=0.494444in, y=3.713806in, left, base]{\color{textcolor}\rmfamily\fontsize{10.000000}{12.000000}\selectfont \(\displaystyle {175}\)}%
\end{pgfscope}%
\begin{pgfscope}%
\pgfsetbuttcap%
\pgfsetroundjoin%
\definecolor{currentfill}{rgb}{0.000000,0.000000,0.000000}%
\pgfsetfillcolor{currentfill}%
\pgfsetlinewidth{0.803000pt}%
\definecolor{currentstroke}{rgb}{0.000000,0.000000,0.000000}%
\pgfsetstrokecolor{currentstroke}%
\pgfsetdash{}{0pt}%
\pgfsys@defobject{currentmarker}{\pgfqpoint{-0.048611in}{0.000000in}}{\pgfqpoint{-0.000000in}{0.000000in}}{%
\pgfpathmoveto{\pgfqpoint{-0.000000in}{0.000000in}}%
\pgfpathlineto{\pgfqpoint{-0.048611in}{0.000000in}}%
\pgfusepath{stroke,fill}%
}%
\begin{pgfscope}%
\pgfsys@transformshift{0.800000in}{4.224000in}%
\pgfsys@useobject{currentmarker}{}%
\end{pgfscope}%
\end{pgfscope}%
\begin{pgfscope}%
\definecolor{textcolor}{rgb}{0.000000,0.000000,0.000000}%
\pgfsetstrokecolor{textcolor}%
\pgfsetfillcolor{textcolor}%
\pgftext[x=0.494444in, y=4.175806in, left, base]{\color{textcolor}\rmfamily\fontsize{10.000000}{12.000000}\selectfont \(\displaystyle {200}\)}%
\end{pgfscope}%
\begin{pgfscope}%
\definecolor{textcolor}{rgb}{0.000000,0.000000,0.000000}%
\pgfsetstrokecolor{textcolor}%
\pgfsetfillcolor{textcolor}%
\pgftext[x=0.438888in,y=2.376000in,,bottom,rotate=90.000000]{\color{textcolor}\rmfamily\fontsize{10.000000}{12.000000}\selectfont time/k (ns)}%
\end{pgfscope}%
\begin{pgfscope}%
\pgfpathrectangle{\pgfqpoint{0.800000in}{0.528000in}}{\pgfqpoint{4.960000in}{3.696000in}}%
\pgfusepath{clip}%
\pgfsetrectcap%
\pgfsetroundjoin%
\pgfsetlinewidth{1.505625pt}%
\definecolor{currentstroke}{rgb}{0.121569,0.466667,0.705882}%
\pgfsetstrokecolor{currentstroke}%
\pgfsetdash{}{0pt}%
\pgfpathmoveto{\pgfqpoint{0.800000in}{0.767148in}}%
\pgfpathlineto{\pgfqpoint{0.965901in}{0.796826in}}%
\pgfpathlineto{\pgfqpoint{1.062947in}{0.803483in}}%
\pgfpathlineto{\pgfqpoint{1.185210in}{0.833112in}}%
\pgfpathlineto{\pgfqpoint{1.297703in}{0.837226in}}%
\pgfpathlineto{\pgfqpoint{1.413907in}{0.875868in}}%
\pgfpathlineto{\pgfqpoint{1.528690in}{0.951284in}}%
\pgfpathlineto{\pgfqpoint{1.644015in}{0.987959in}}%
\pgfpathlineto{\pgfqpoint{1.759133in}{1.031898in}}%
\pgfpathlineto{\pgfqpoint{1.874330in}{1.063585in}}%
\pgfpathlineto{\pgfqpoint{1.989498in}{1.107548in}}%
\pgfpathlineto{\pgfqpoint{2.104676in}{1.141005in}}%
\pgfpathlineto{\pgfqpoint{2.219850in}{1.181360in}}%
\pgfpathlineto{\pgfqpoint{2.335026in}{1.217959in}}%
\pgfpathlineto{\pgfqpoint{2.450201in}{1.253159in}}%
\pgfpathlineto{\pgfqpoint{2.565377in}{1.288625in}}%
\pgfpathlineto{\pgfqpoint{2.680552in}{1.328697in}}%
\pgfpathlineto{\pgfqpoint{2.795728in}{1.373740in}}%
\pgfpathlineto{\pgfqpoint{2.910903in}{1.413855in}}%
\pgfpathlineto{\pgfqpoint{3.026079in}{1.444143in}}%
\pgfpathlineto{\pgfqpoint{3.141254in}{1.483238in}}%
\pgfpathlineto{\pgfqpoint{3.256429in}{1.518958in}}%
\pgfpathlineto{\pgfqpoint{3.371605in}{1.553224in}}%
\pgfpathlineto{\pgfqpoint{3.486780in}{1.606597in}}%
\pgfpathlineto{\pgfqpoint{3.601956in}{1.651364in}}%
\pgfpathlineto{\pgfqpoint{3.717131in}{1.706004in}}%
\pgfpathlineto{\pgfqpoint{3.832306in}{1.764237in}}%
\pgfpathlineto{\pgfqpoint{3.947482in}{1.828712in}}%
\pgfpathlineto{\pgfqpoint{4.062657in}{1.883341in}}%
\pgfpathlineto{\pgfqpoint{4.177833in}{0.941631in}}%
\pgfusepath{stroke}%
\end{pgfscope}%
\begin{pgfscope}%
\pgfpathrectangle{\pgfqpoint{0.800000in}{0.528000in}}{\pgfqpoint{4.960000in}{3.696000in}}%
\pgfusepath{clip}%
\pgfsetrectcap%
\pgfsetroundjoin%
\pgfsetlinewidth{1.505625pt}%
\definecolor{currentstroke}{rgb}{1.000000,0.498039,0.054902}%
\pgfsetstrokecolor{currentstroke}%
\pgfsetdash{}{0pt}%
\pgfpathmoveto{\pgfqpoint{0.800000in}{1.606338in}}%
\pgfpathlineto{\pgfqpoint{0.965901in}{1.501957in}}%
\pgfpathlineto{\pgfqpoint{1.062947in}{1.498272in}}%
\pgfpathlineto{\pgfqpoint{1.185210in}{1.473843in}}%
\pgfpathlineto{\pgfqpoint{1.297703in}{1.560035in}}%
\pgfpathlineto{\pgfqpoint{1.413907in}{1.509537in}}%
\pgfpathlineto{\pgfqpoint{1.528690in}{1.525678in}}%
\pgfpathlineto{\pgfqpoint{1.644015in}{1.524192in}}%
\pgfpathlineto{\pgfqpoint{1.759133in}{1.508366in}}%
\pgfpathlineto{\pgfqpoint{1.874330in}{1.962836in}}%
\pgfpathlineto{\pgfqpoint{1.989498in}{1.998135in}}%
\pgfpathlineto{\pgfqpoint{2.104676in}{1.990570in}}%
\pgfpathlineto{\pgfqpoint{2.219850in}{2.002653in}}%
\pgfpathlineto{\pgfqpoint{2.335026in}{1.984682in}}%
\pgfpathlineto{\pgfqpoint{2.450201in}{1.977874in}}%
\pgfpathlineto{\pgfqpoint{2.565377in}{1.979324in}}%
\pgfpathlineto{\pgfqpoint{2.680552in}{1.963098in}}%
\pgfpathlineto{\pgfqpoint{2.795728in}{1.967878in}}%
\pgfpathlineto{\pgfqpoint{2.910903in}{2.013051in}}%
\pgfpathlineto{\pgfqpoint{3.026079in}{2.089273in}}%
\pgfpathlineto{\pgfqpoint{3.141254in}{2.163966in}}%
\pgfpathlineto{\pgfqpoint{3.256429in}{2.233138in}}%
\pgfpathlineto{\pgfqpoint{3.371605in}{2.243522in}}%
\pgfpathlineto{\pgfqpoint{3.486780in}{2.259806in}}%
\pgfpathlineto{\pgfqpoint{3.601956in}{2.257417in}}%
\pgfpathlineto{\pgfqpoint{3.717131in}{2.245971in}}%
\pgfpathlineto{\pgfqpoint{3.832306in}{2.183705in}}%
\pgfpathlineto{\pgfqpoint{3.947482in}{2.029479in}}%
\pgfpathlineto{\pgfqpoint{4.062657in}{2.110699in}}%
\pgfpathlineto{\pgfqpoint{4.177833in}{2.077987in}}%
\pgfusepath{stroke}%
\end{pgfscope}%
\begin{pgfscope}%
\pgfpathrectangle{\pgfqpoint{0.800000in}{0.528000in}}{\pgfqpoint{4.960000in}{3.696000in}}%
\pgfusepath{clip}%
\pgfsetrectcap%
\pgfsetroundjoin%
\pgfsetlinewidth{1.505625pt}%
\definecolor{currentstroke}{rgb}{0.172549,0.627451,0.172549}%
\pgfsetstrokecolor{currentstroke}%
\pgfsetdash{}{0pt}%
\pgfpathmoveto{\pgfqpoint{0.800000in}{1.652250in}}%
\pgfpathlineto{\pgfqpoint{0.965901in}{1.633148in}}%
\pgfpathlineto{\pgfqpoint{1.062947in}{1.670863in}}%
\pgfpathlineto{\pgfqpoint{1.185210in}{1.703667in}}%
\pgfpathlineto{\pgfqpoint{1.297703in}{1.770129in}}%
\pgfpathlineto{\pgfqpoint{1.413907in}{1.787242in}}%
\pgfpathlineto{\pgfqpoint{1.528690in}{1.834178in}}%
\pgfpathlineto{\pgfqpoint{1.644015in}{1.837852in}}%
\pgfpathlineto{\pgfqpoint{1.759133in}{1.840150in}}%
\pgfpathlineto{\pgfqpoint{1.874330in}{2.354130in}}%
\pgfpathlineto{\pgfqpoint{1.989498in}{2.367118in}}%
\pgfpathlineto{\pgfqpoint{2.104676in}{2.372319in}}%
\pgfpathlineto{\pgfqpoint{2.219850in}{2.387069in}}%
\pgfpathlineto{\pgfqpoint{2.335026in}{2.340101in}}%
\pgfpathlineto{\pgfqpoint{2.450201in}{2.348226in}}%
\pgfpathlineto{\pgfqpoint{2.565377in}{2.315969in}}%
\pgfpathlineto{\pgfqpoint{2.680552in}{2.287101in}}%
\pgfpathlineto{\pgfqpoint{2.795728in}{2.273421in}}%
\pgfpathlineto{\pgfqpoint{2.910903in}{2.293137in}}%
\pgfpathlineto{\pgfqpoint{3.026079in}{2.368521in}}%
\pgfpathlineto{\pgfqpoint{3.141254in}{2.428417in}}%
\pgfpathlineto{\pgfqpoint{3.256429in}{2.491890in}}%
\pgfpathlineto{\pgfqpoint{3.371605in}{2.470113in}}%
\pgfpathlineto{\pgfqpoint{3.486780in}{2.479434in}}%
\pgfpathlineto{\pgfqpoint{3.601956in}{2.464435in}}%
\pgfpathlineto{\pgfqpoint{3.717131in}{2.463884in}}%
\pgfpathlineto{\pgfqpoint{3.832306in}{2.412589in}}%
\pgfpathlineto{\pgfqpoint{3.947482in}{2.351351in}}%
\pgfpathlineto{\pgfqpoint{4.062657in}{2.361727in}}%
\pgfpathlineto{\pgfqpoint{4.177833in}{2.785787in}}%
\pgfusepath{stroke}%
\end{pgfscope}%
\begin{pgfscope}%
\pgfpathrectangle{\pgfqpoint{0.800000in}{0.528000in}}{\pgfqpoint{4.960000in}{3.696000in}}%
\pgfusepath{clip}%
\pgfsetrectcap%
\pgfsetroundjoin%
\pgfsetlinewidth{1.505625pt}%
\definecolor{currentstroke}{rgb}{0.839216,0.152941,0.156863}%
\pgfsetstrokecolor{currentstroke}%
\pgfsetdash{}{0pt}%
\pgfpathmoveto{\pgfqpoint{3.242835in}{4.234000in}}%
\pgfpathlineto{\pgfqpoint{3.256429in}{4.000076in}}%
\pgfpathlineto{\pgfqpoint{3.371605in}{2.759213in}}%
\pgfpathlineto{\pgfqpoint{3.486780in}{2.017338in}}%
\pgfpathlineto{\pgfqpoint{3.601956in}{1.582801in}}%
\pgfpathlineto{\pgfqpoint{3.717131in}{1.327566in}}%
\pgfpathlineto{\pgfqpoint{3.832306in}{1.152716in}}%
\pgfpathlineto{\pgfqpoint{3.947482in}{1.022800in}}%
\pgfpathlineto{\pgfqpoint{4.062657in}{0.878431in}}%
\pgfpathlineto{\pgfqpoint{4.177833in}{0.746628in}}%
\pgfusepath{stroke}%
\end{pgfscope}%
\begin{pgfscope}%
\pgfpathrectangle{\pgfqpoint{0.800000in}{0.528000in}}{\pgfqpoint{4.960000in}{3.696000in}}%
\pgfusepath{clip}%
\pgfsetrectcap%
\pgfsetroundjoin%
\pgfsetlinewidth{1.505625pt}%
\definecolor{currentstroke}{rgb}{0.580392,0.403922,0.741176}%
\pgfsetstrokecolor{currentstroke}%
\pgfsetdash{}{0pt}%
\pgfpathmoveto{\pgfqpoint{0.800000in}{0.633651in}}%
\pgfpathlineto{\pgfqpoint{0.965901in}{0.630608in}}%
\pgfpathlineto{\pgfqpoint{1.062947in}{0.642055in}}%
\pgfpathlineto{\pgfqpoint{1.185210in}{0.663787in}}%
\pgfpathlineto{\pgfqpoint{1.297703in}{0.666592in}}%
\pgfpathlineto{\pgfqpoint{1.413907in}{0.702118in}}%
\pgfpathlineto{\pgfqpoint{1.528690in}{0.748771in}}%
\pgfpathlineto{\pgfqpoint{1.644015in}{0.823285in}}%
\pgfpathlineto{\pgfqpoint{1.759133in}{0.959936in}}%
\pgfpathlineto{\pgfqpoint{1.874330in}{1.120152in}}%
\pgfpathlineto{\pgfqpoint{1.989498in}{1.296812in}}%
\pgfpathlineto{\pgfqpoint{2.104676in}{1.583209in}}%
\pgfpathlineto{\pgfqpoint{2.219850in}{2.014734in}}%
\pgfpathlineto{\pgfqpoint{2.335026in}{2.690258in}}%
\pgfpathlineto{\pgfqpoint{2.450201in}{3.765131in}}%
\pgfpathlineto{\pgfqpoint{2.480294in}{4.234000in}}%
\pgfusepath{stroke}%
\end{pgfscope}%
\begin{pgfscope}%
\pgfpathrectangle{\pgfqpoint{0.800000in}{0.528000in}}{\pgfqpoint{4.960000in}{3.696000in}}%
\pgfusepath{clip}%
\pgfsetrectcap%
\pgfsetroundjoin%
\pgfsetlinewidth{1.505625pt}%
\definecolor{currentstroke}{rgb}{0.549020,0.337255,0.294118}%
\pgfsetstrokecolor{currentstroke}%
\pgfsetdash{}{0pt}%
\pgfpathmoveto{\pgfqpoint{0.800000in}{0.629031in}}%
\pgfpathlineto{\pgfqpoint{0.965901in}{0.640335in}}%
\pgfpathlineto{\pgfqpoint{1.062947in}{0.637379in}}%
\pgfpathlineto{\pgfqpoint{1.185210in}{0.648255in}}%
\pgfpathlineto{\pgfqpoint{1.297703in}{0.661782in}}%
\pgfpathlineto{\pgfqpoint{1.413907in}{0.687822in}}%
\pgfpathlineto{\pgfqpoint{1.528690in}{0.720921in}}%
\pgfpathlineto{\pgfqpoint{1.644015in}{0.768136in}}%
\pgfpathlineto{\pgfqpoint{1.759133in}{0.906322in}}%
\pgfpathlineto{\pgfqpoint{1.874330in}{1.036263in}}%
\pgfpathlineto{\pgfqpoint{1.989498in}{1.211835in}}%
\pgfpathlineto{\pgfqpoint{2.104676in}{1.488955in}}%
\pgfpathlineto{\pgfqpoint{2.219850in}{1.916000in}}%
\pgfpathlineto{\pgfqpoint{2.335026in}{2.591962in}}%
\pgfpathlineto{\pgfqpoint{2.450201in}{3.681381in}}%
\pgfpathlineto{\pgfqpoint{2.486154in}{4.234000in}}%
\pgfusepath{stroke}%
\end{pgfscope}%
\begin{pgfscope}%
\pgfpathrectangle{\pgfqpoint{0.800000in}{0.528000in}}{\pgfqpoint{4.960000in}{3.696000in}}%
\pgfusepath{clip}%
\pgfsetrectcap%
\pgfsetroundjoin%
\pgfsetlinewidth{1.505625pt}%
\definecolor{currentstroke}{rgb}{0.890196,0.466667,0.760784}%
\pgfsetstrokecolor{currentstroke}%
\pgfsetdash{}{0pt}%
\pgfpathmoveto{\pgfqpoint{0.800000in}{1.933351in}}%
\pgfpathlineto{\pgfqpoint{0.965901in}{1.535480in}}%
\pgfpathlineto{\pgfqpoint{1.062947in}{1.457401in}}%
\pgfpathlineto{\pgfqpoint{1.185210in}{1.503053in}}%
\pgfpathlineto{\pgfqpoint{1.297703in}{1.490902in}}%
\pgfpathlineto{\pgfqpoint{1.413907in}{1.461417in}}%
\pgfpathlineto{\pgfqpoint{1.528690in}{1.465716in}}%
\pgfpathlineto{\pgfqpoint{1.644015in}{1.445863in}}%
\pgfpathlineto{\pgfqpoint{1.759133in}{1.440533in}}%
\pgfpathlineto{\pgfqpoint{1.874330in}{1.880537in}}%
\pgfpathlineto{\pgfqpoint{1.989498in}{1.894583in}}%
\pgfpathlineto{\pgfqpoint{2.104676in}{1.902187in}}%
\pgfpathlineto{\pgfqpoint{2.219850in}{1.921769in}}%
\pgfpathlineto{\pgfqpoint{2.335026in}{1.895153in}}%
\pgfpathlineto{\pgfqpoint{2.450201in}{1.876415in}}%
\pgfpathlineto{\pgfqpoint{2.565377in}{1.861546in}}%
\pgfpathlineto{\pgfqpoint{2.680552in}{1.836307in}}%
\pgfpathlineto{\pgfqpoint{2.795728in}{1.841078in}}%
\pgfpathlineto{\pgfqpoint{2.910903in}{1.886181in}}%
\pgfpathlineto{\pgfqpoint{3.026079in}{1.967249in}}%
\pgfpathlineto{\pgfqpoint{3.141254in}{2.057044in}}%
\pgfpathlineto{\pgfqpoint{3.256429in}{2.149767in}}%
\pgfpathlineto{\pgfqpoint{3.371605in}{2.158759in}}%
\pgfpathlineto{\pgfqpoint{3.486780in}{2.186940in}}%
\pgfpathlineto{\pgfqpoint{3.601956in}{2.219745in}}%
\pgfpathlineto{\pgfqpoint{3.717131in}{2.258837in}}%
\pgfpathlineto{\pgfqpoint{3.832306in}{2.291895in}}%
\pgfpathlineto{\pgfqpoint{3.947482in}{2.391281in}}%
\pgfpathlineto{\pgfqpoint{4.062657in}{2.905683in}}%
\pgfpathlineto{\pgfqpoint{4.072024in}{4.234000in}}%
\pgfusepath{stroke}%
\end{pgfscope}%
\begin{pgfscope}%
\pgfpathrectangle{\pgfqpoint{0.800000in}{0.528000in}}{\pgfqpoint{4.960000in}{3.696000in}}%
\pgfusepath{clip}%
\pgfsetrectcap%
\pgfsetroundjoin%
\pgfsetlinewidth{1.505625pt}%
\definecolor{currentstroke}{rgb}{0.498039,0.498039,0.498039}%
\pgfsetstrokecolor{currentstroke}%
\pgfsetdash{}{0pt}%
\pgfpathmoveto{\pgfqpoint{3.287834in}{4.234000in}}%
\pgfpathlineto{\pgfqpoint{3.371605in}{3.225751in}}%
\pgfpathlineto{\pgfqpoint{3.486780in}{2.368056in}}%
\pgfpathlineto{\pgfqpoint{3.601956in}{1.838625in}}%
\pgfpathlineto{\pgfqpoint{3.717131in}{1.480223in}}%
\pgfpathlineto{\pgfqpoint{3.832306in}{1.213987in}}%
\pgfpathlineto{\pgfqpoint{3.947482in}{1.007076in}}%
\pgfpathlineto{\pgfqpoint{4.062657in}{0.826673in}}%
\pgfpathlineto{\pgfqpoint{4.177833in}{0.662686in}}%
\pgfusepath{stroke}%
\end{pgfscope}%
\begin{pgfscope}%
\pgfpathrectangle{\pgfqpoint{0.800000in}{0.528000in}}{\pgfqpoint{4.960000in}{3.696000in}}%
\pgfusepath{clip}%
\pgfsetrectcap%
\pgfsetroundjoin%
\pgfsetlinewidth{1.505625pt}%
\definecolor{currentstroke}{rgb}{0.737255,0.741176,0.133333}%
\pgfsetstrokecolor{currentstroke}%
\pgfsetdash{}{0pt}%
\pgfpathmoveto{\pgfqpoint{2.658875in}{4.234000in}}%
\pgfpathlineto{\pgfqpoint{2.680552in}{3.860760in}}%
\pgfpathlineto{\pgfqpoint{2.795728in}{2.646013in}}%
\pgfpathlineto{\pgfqpoint{2.910903in}{1.892039in}}%
\pgfpathlineto{\pgfqpoint{3.026079in}{1.424223in}}%
\pgfpathlineto{\pgfqpoint{3.141254in}{1.140330in}}%
\pgfpathlineto{\pgfqpoint{3.256429in}{0.958839in}}%
\pgfpathlineto{\pgfqpoint{3.371605in}{0.849767in}}%
\pgfpathlineto{\pgfqpoint{3.486780in}{0.783271in}}%
\pgfpathlineto{\pgfqpoint{3.601956in}{0.753106in}}%
\pgfpathlineto{\pgfqpoint{3.717131in}{0.715117in}}%
\pgfpathlineto{\pgfqpoint{3.832306in}{0.697975in}}%
\pgfpathlineto{\pgfqpoint{3.947482in}{0.690147in}}%
\pgfpathlineto{\pgfqpoint{4.062657in}{0.687751in}}%
\pgfpathlineto{\pgfqpoint{4.177833in}{0.678022in}}%
\pgfusepath{stroke}%
\end{pgfscope}%
\begin{pgfscope}%
\pgfsetrectcap%
\pgfsetmiterjoin%
\pgfsetlinewidth{0.803000pt}%
\definecolor{currentstroke}{rgb}{0.000000,0.000000,0.000000}%
\pgfsetstrokecolor{currentstroke}%
\pgfsetdash{}{0pt}%
\pgfpathmoveto{\pgfqpoint{0.800000in}{0.528000in}}%
\pgfpathlineto{\pgfqpoint{0.800000in}{4.224000in}}%
\pgfusepath{stroke}%
\end{pgfscope}%
\begin{pgfscope}%
\pgfsetrectcap%
\pgfsetmiterjoin%
\pgfsetlinewidth{0.803000pt}%
\definecolor{currentstroke}{rgb}{0.000000,0.000000,0.000000}%
\pgfsetstrokecolor{currentstroke}%
\pgfsetdash{}{0pt}%
\pgfpathmoveto{\pgfqpoint{5.760000in}{0.528000in}}%
\pgfpathlineto{\pgfqpoint{5.760000in}{4.224000in}}%
\pgfusepath{stroke}%
\end{pgfscope}%
\begin{pgfscope}%
\pgfsetrectcap%
\pgfsetmiterjoin%
\pgfsetlinewidth{0.803000pt}%
\definecolor{currentstroke}{rgb}{0.000000,0.000000,0.000000}%
\pgfsetstrokecolor{currentstroke}%
\pgfsetdash{}{0pt}%
\pgfpathmoveto{\pgfqpoint{0.800000in}{0.528000in}}%
\pgfpathlineto{\pgfqpoint{5.760000in}{0.528000in}}%
\pgfusepath{stroke}%
\end{pgfscope}%
\begin{pgfscope}%
\pgfsetrectcap%
\pgfsetmiterjoin%
\pgfsetlinewidth{0.803000pt}%
\definecolor{currentstroke}{rgb}{0.000000,0.000000,0.000000}%
\pgfsetstrokecolor{currentstroke}%
\pgfsetdash{}{0pt}%
\pgfpathmoveto{\pgfqpoint{0.800000in}{4.224000in}}%
\pgfpathlineto{\pgfqpoint{5.760000in}{4.224000in}}%
\pgfusepath{stroke}%
\end{pgfscope}%
\begin{pgfscope}%
\definecolor{textcolor}{rgb}{0.000000,0.000000,0.000000}%
\pgfsetstrokecolor{textcolor}%
\pgfsetfillcolor{textcolor}%
\pgftext[x=3.280000in,y=4.260960in,,base]{\color{textcolor}\rmfamily\fontsize{10.000000}{12.000000}\selectfont Random order n=1.3E+06}%
\end{pgfscope}%
\begin{pgfscope}%
\pgfsetbuttcap%
\pgfsetmiterjoin%
\definecolor{currentfill}{rgb}{1.000000,1.000000,1.000000}%
\pgfsetfillcolor{currentfill}%
\pgfsetfillopacity{0.800000}%
\pgfsetlinewidth{1.003750pt}%
\definecolor{currentstroke}{rgb}{0.800000,0.800000,0.800000}%
\pgfsetstrokecolor{currentstroke}%
\pgfsetstrokeopacity{0.800000}%
\pgfsetdash{}{0pt}%
\pgfpathmoveto{\pgfqpoint{4.279722in}{2.364417in}}%
\pgfpathlineto{\pgfqpoint{5.662778in}{2.364417in}}%
\pgfpathquadraticcurveto{\pgfqpoint{5.690556in}{2.364417in}}{\pgfqpoint{5.690556in}{2.392194in}}%
\pgfpathlineto{\pgfqpoint{5.690556in}{4.126778in}}%
\pgfpathquadraticcurveto{\pgfqpoint{5.690556in}{4.154556in}}{\pgfqpoint{5.662778in}{4.154556in}}%
\pgfpathlineto{\pgfqpoint{4.279722in}{4.154556in}}%
\pgfpathquadraticcurveto{\pgfqpoint{4.251944in}{4.154556in}}{\pgfqpoint{4.251944in}{4.126778in}}%
\pgfpathlineto{\pgfqpoint{4.251944in}{2.392194in}}%
\pgfpathquadraticcurveto{\pgfqpoint{4.251944in}{2.364417in}}{\pgfqpoint{4.279722in}{2.364417in}}%
\pgfpathlineto{\pgfqpoint{4.279722in}{2.364417in}}%
\pgfpathclose%
\pgfusepath{stroke,fill}%
\end{pgfscope}%
\begin{pgfscope}%
\pgfsetrectcap%
\pgfsetroundjoin%
\pgfsetlinewidth{1.505625pt}%
\definecolor{currentstroke}{rgb}{0.121569,0.466667,0.705882}%
\pgfsetstrokecolor{currentstroke}%
\pgfsetdash{}{0pt}%
\pgfpathmoveto{\pgfqpoint{4.307500in}{4.050389in}}%
\pgfpathlineto{\pgfqpoint{4.446389in}{4.050389in}}%
\pgfpathlineto{\pgfqpoint{4.585278in}{4.050389in}}%
\pgfusepath{stroke}%
\end{pgfscope}%
\begin{pgfscope}%
\definecolor{textcolor}{rgb}{0.000000,0.000000,0.000000}%
\pgfsetstrokecolor{textcolor}%
\pgfsetfillcolor{textcolor}%
\pgftext[x=4.696389in,y=4.001778in,left,base]{\color{textcolor}\rmfamily\fontsize{10.000000}{12.000000}\selectfont cardchoose}%
\end{pgfscope}%
\begin{pgfscope}%
\pgfsetrectcap%
\pgfsetroundjoin%
\pgfsetlinewidth{1.505625pt}%
\definecolor{currentstroke}{rgb}{1.000000,0.498039,0.054902}%
\pgfsetstrokecolor{currentstroke}%
\pgfsetdash{}{0pt}%
\pgfpathmoveto{\pgfqpoint{4.307500in}{3.856083in}}%
\pgfpathlineto{\pgfqpoint{4.446389in}{3.856083in}}%
\pgfpathlineto{\pgfqpoint{4.585278in}{3.856083in}}%
\pgfusepath{stroke}%
\end{pgfscope}%
\begin{pgfscope}%
\definecolor{textcolor}{rgb}{0.000000,0.000000,0.000000}%
\pgfsetstrokecolor{textcolor}%
\pgfsetfillcolor{textcolor}%
\pgftext[x=4.696389in,y=3.807472in,left,base]{\color{textcolor}\rmfamily\fontsize{10.000000}{12.000000}\selectfont floydf2}%
\end{pgfscope}%
\begin{pgfscope}%
\pgfsetrectcap%
\pgfsetroundjoin%
\pgfsetlinewidth{1.505625pt}%
\definecolor{currentstroke}{rgb}{0.172549,0.627451,0.172549}%
\pgfsetstrokecolor{currentstroke}%
\pgfsetdash{}{0pt}%
\pgfpathmoveto{\pgfqpoint{4.307500in}{3.660944in}}%
\pgfpathlineto{\pgfqpoint{4.446389in}{3.660944in}}%
\pgfpathlineto{\pgfqpoint{4.585278in}{3.660944in}}%
\pgfusepath{stroke}%
\end{pgfscope}%
\begin{pgfscope}%
\definecolor{textcolor}{rgb}{0.000000,0.000000,0.000000}%
\pgfsetstrokecolor{textcolor}%
\pgfsetfillcolor{textcolor}%
\pgftext[x=4.696389in,y=3.612333in,left,base]{\color{textcolor}\rmfamily\fontsize{10.000000}{12.000000}\selectfont hsel}%
\end{pgfscope}%
\begin{pgfscope}%
\pgfsetrectcap%
\pgfsetroundjoin%
\pgfsetlinewidth{1.505625pt}%
\definecolor{currentstroke}{rgb}{0.839216,0.152941,0.156863}%
\pgfsetstrokecolor{currentstroke}%
\pgfsetdash{}{0pt}%
\pgfpathmoveto{\pgfqpoint{4.307500in}{3.467333in}}%
\pgfpathlineto{\pgfqpoint{4.446389in}{3.467333in}}%
\pgfpathlineto{\pgfqpoint{4.585278in}{3.467333in}}%
\pgfusepath{stroke}%
\end{pgfscope}%
\begin{pgfscope}%
\definecolor{textcolor}{rgb}{0.000000,0.000000,0.000000}%
\pgfsetstrokecolor{textcolor}%
\pgfsetfillcolor{textcolor}%
\pgftext[x=4.696389in,y=3.418722in,left,base]{\color{textcolor}\rmfamily\fontsize{10.000000}{12.000000}\selectfont iterativechoose}%
\end{pgfscope}%
\begin{pgfscope}%
\pgfsetrectcap%
\pgfsetroundjoin%
\pgfsetlinewidth{1.505625pt}%
\definecolor{currentstroke}{rgb}{0.580392,0.403922,0.741176}%
\pgfsetstrokecolor{currentstroke}%
\pgfsetdash{}{0pt}%
\pgfpathmoveto{\pgfqpoint{4.307500in}{3.273028in}}%
\pgfpathlineto{\pgfqpoint{4.446389in}{3.273028in}}%
\pgfpathlineto{\pgfqpoint{4.585278in}{3.273028in}}%
\pgfusepath{stroke}%
\end{pgfscope}%
\begin{pgfscope}%
\definecolor{textcolor}{rgb}{0.000000,0.000000,0.000000}%
\pgfsetstrokecolor{textcolor}%
\pgfsetfillcolor{textcolor}%
\pgftext[x=4.696389in,y=3.224417in,left,base]{\color{textcolor}\rmfamily\fontsize{10.000000}{12.000000}\selectfont quadraticf2}%
\end{pgfscope}%
\begin{pgfscope}%
\pgfsetrectcap%
\pgfsetroundjoin%
\pgfsetlinewidth{1.505625pt}%
\definecolor{currentstroke}{rgb}{0.549020,0.337255,0.294118}%
\pgfsetstrokecolor{currentstroke}%
\pgfsetdash{}{0pt}%
\pgfpathmoveto{\pgfqpoint{4.307500in}{3.079417in}}%
\pgfpathlineto{\pgfqpoint{4.446389in}{3.079417in}}%
\pgfpathlineto{\pgfqpoint{4.585278in}{3.079417in}}%
\pgfusepath{stroke}%
\end{pgfscope}%
\begin{pgfscope}%
\definecolor{textcolor}{rgb}{0.000000,0.000000,0.000000}%
\pgfsetstrokecolor{textcolor}%
\pgfsetfillcolor{textcolor}%
\pgftext[x=4.696389in,y=3.030806in,left,base]{\color{textcolor}\rmfamily\fontsize{10.000000}{12.000000}\selectfont quadraticreject}%
\end{pgfscope}%
\begin{pgfscope}%
\pgfsetrectcap%
\pgfsetroundjoin%
\pgfsetlinewidth{1.505625pt}%
\definecolor{currentstroke}{rgb}{0.890196,0.466667,0.760784}%
\pgfsetstrokecolor{currentstroke}%
\pgfsetdash{}{0pt}%
\pgfpathmoveto{\pgfqpoint{4.307500in}{2.884278in}}%
\pgfpathlineto{\pgfqpoint{4.446389in}{2.884278in}}%
\pgfpathlineto{\pgfqpoint{4.585278in}{2.884278in}}%
\pgfusepath{stroke}%
\end{pgfscope}%
\begin{pgfscope}%
\definecolor{textcolor}{rgb}{0.000000,0.000000,0.000000}%
\pgfsetstrokecolor{textcolor}%
\pgfsetfillcolor{textcolor}%
\pgftext[x=4.696389in,y=2.835667in,left,base]{\color{textcolor}\rmfamily\fontsize{10.000000}{12.000000}\selectfont rejectionsample}%
\end{pgfscope}%
\begin{pgfscope}%
\pgfsetrectcap%
\pgfsetroundjoin%
\pgfsetlinewidth{1.505625pt}%
\definecolor{currentstroke}{rgb}{0.498039,0.498039,0.498039}%
\pgfsetstrokecolor{currentstroke}%
\pgfsetdash{}{0pt}%
\pgfpathmoveto{\pgfqpoint{4.307500in}{2.689139in}}%
\pgfpathlineto{\pgfqpoint{4.446389in}{2.689139in}}%
\pgfpathlineto{\pgfqpoint{4.585278in}{2.689139in}}%
\pgfusepath{stroke}%
\end{pgfscope}%
\begin{pgfscope}%
\definecolor{textcolor}{rgb}{0.000000,0.000000,0.000000}%
\pgfsetstrokecolor{textcolor}%
\pgfsetfillcolor{textcolor}%
\pgftext[x=4.696389in,y=2.640528in,left,base]{\color{textcolor}\rmfamily\fontsize{10.000000}{12.000000}\selectfont reservoirsample}%
\end{pgfscope}%
\begin{pgfscope}%
\pgfsetrectcap%
\pgfsetroundjoin%
\pgfsetlinewidth{1.505625pt}%
\definecolor{currentstroke}{rgb}{0.737255,0.741176,0.133333}%
\pgfsetstrokecolor{currentstroke}%
\pgfsetdash{}{0pt}%
\pgfpathmoveto{\pgfqpoint{4.307500in}{2.495528in}}%
\pgfpathlineto{\pgfqpoint{4.446389in}{2.495528in}}%
\pgfpathlineto{\pgfqpoint{4.585278in}{2.495528in}}%
\pgfusepath{stroke}%
\end{pgfscope}%
\begin{pgfscope}%
\definecolor{textcolor}{rgb}{0.000000,0.000000,0.000000}%
\pgfsetstrokecolor{textcolor}%
\pgfsetfillcolor{textcolor}%
\pgftext[x=4.696389in,y=2.446917in,left,base]{\color{textcolor}\rmfamily\fontsize{10.000000}{12.000000}\selectfont select}%
\end{pgfscope}%
\end{pgfpicture}%
\makeatother%
\endgroup%

    \caption{Random order, \(n=1.3 \times 10^{6}\)}
    \label{randomsmalln}
\end{figure}

\begin{figure}
    %% Creator: Matplotlib, PGF backend
%%
%% To include the figure in your LaTeX document, write
%%   \input{<filename>.pgf}
%%
%% Make sure the required packages are loaded in your preamble
%%   \usepackage{pgf}
%%
%% Also ensure that all the required font packages are loaded; for instance,
%% the lmodern package is sometimes necessary when using math font.
%%   \usepackage{lmodern}
%%
%% Figures using additional raster images can only be included by \input if
%% they are in the same directory as the main LaTeX file. For loading figures
%% from other directories you can use the `import` package
%%   \usepackage{import}
%%
%% and then include the figures with
%%   \import{<path to file>}{<filename>.pgf}
%%
%% Matplotlib used the following preamble
%%   
%%   \usepackage{fontspec}
%%   \makeatletter\@ifpackageloaded{underscore}{}{\usepackage[strings]{underscore}}\makeatother
%%
\begingroup%
\makeatletter%
\begin{pgfpicture}%
\pgfpathrectangle{\pgfpointorigin}{\pgfqpoint{6.400000in}{4.800000in}}%
\pgfusepath{use as bounding box, clip}%
\begin{pgfscope}%
\pgfsetbuttcap%
\pgfsetmiterjoin%
\definecolor{currentfill}{rgb}{1.000000,1.000000,1.000000}%
\pgfsetfillcolor{currentfill}%
\pgfsetlinewidth{0.000000pt}%
\definecolor{currentstroke}{rgb}{1.000000,1.000000,1.000000}%
\pgfsetstrokecolor{currentstroke}%
\pgfsetdash{}{0pt}%
\pgfpathmoveto{\pgfqpoint{0.000000in}{0.000000in}}%
\pgfpathlineto{\pgfqpoint{6.400000in}{0.000000in}}%
\pgfpathlineto{\pgfqpoint{6.400000in}{4.800000in}}%
\pgfpathlineto{\pgfqpoint{0.000000in}{4.800000in}}%
\pgfpathlineto{\pgfqpoint{0.000000in}{0.000000in}}%
\pgfpathclose%
\pgfusepath{fill}%
\end{pgfscope}%
\begin{pgfscope}%
\pgfsetbuttcap%
\pgfsetmiterjoin%
\definecolor{currentfill}{rgb}{1.000000,1.000000,1.000000}%
\pgfsetfillcolor{currentfill}%
\pgfsetlinewidth{0.000000pt}%
\definecolor{currentstroke}{rgb}{0.000000,0.000000,0.000000}%
\pgfsetstrokecolor{currentstroke}%
\pgfsetstrokeopacity{0.000000}%
\pgfsetdash{}{0pt}%
\pgfpathmoveto{\pgfqpoint{0.800000in}{0.528000in}}%
\pgfpathlineto{\pgfqpoint{5.760000in}{0.528000in}}%
\pgfpathlineto{\pgfqpoint{5.760000in}{4.224000in}}%
\pgfpathlineto{\pgfqpoint{0.800000in}{4.224000in}}%
\pgfpathlineto{\pgfqpoint{0.800000in}{0.528000in}}%
\pgfpathclose%
\pgfusepath{fill}%
\end{pgfscope}%
\begin{pgfscope}%
\pgfsetbuttcap%
\pgfsetroundjoin%
\definecolor{currentfill}{rgb}{0.000000,0.000000,0.000000}%
\pgfsetfillcolor{currentfill}%
\pgfsetlinewidth{0.803000pt}%
\definecolor{currentstroke}{rgb}{0.000000,0.000000,0.000000}%
\pgfsetstrokecolor{currentstroke}%
\pgfsetdash{}{0pt}%
\pgfsys@defobject{currentmarker}{\pgfqpoint{0.000000in}{-0.048611in}}{\pgfqpoint{0.000000in}{0.000000in}}{%
\pgfpathmoveto{\pgfqpoint{0.000000in}{0.000000in}}%
\pgfpathlineto{\pgfqpoint{0.000000in}{-0.048611in}}%
\pgfusepath{stroke,fill}%
}%
\begin{pgfscope}%
\pgfsys@transformshift{0.800000in}{0.528000in}%
\pgfsys@useobject{currentmarker}{}%
\end{pgfscope}%
\end{pgfscope}%
\begin{pgfscope}%
\definecolor{textcolor}{rgb}{0.000000,0.000000,0.000000}%
\pgfsetstrokecolor{textcolor}%
\pgfsetfillcolor{textcolor}%
\pgftext[x=0.800000in,y=0.430778in,,top]{\color{textcolor}\rmfamily\fontsize{10.000000}{12.000000}\selectfont \(\displaystyle {10^{0}}\)}%
\end{pgfscope}%
\begin{pgfscope}%
\pgfsetbuttcap%
\pgfsetroundjoin%
\definecolor{currentfill}{rgb}{0.000000,0.000000,0.000000}%
\pgfsetfillcolor{currentfill}%
\pgfsetlinewidth{0.803000pt}%
\definecolor{currentstroke}{rgb}{0.000000,0.000000,0.000000}%
\pgfsetstrokecolor{currentstroke}%
\pgfsetdash{}{0pt}%
\pgfsys@defobject{currentmarker}{\pgfqpoint{0.000000in}{-0.048611in}}{\pgfqpoint{0.000000in}{0.000000in}}{%
\pgfpathmoveto{\pgfqpoint{0.000000in}{0.000000in}}%
\pgfpathlineto{\pgfqpoint{0.000000in}{-0.048611in}}%
\pgfusepath{stroke,fill}%
}%
\begin{pgfscope}%
\pgfsys@transformshift{1.902222in}{0.528000in}%
\pgfsys@useobject{currentmarker}{}%
\end{pgfscope}%
\end{pgfscope}%
\begin{pgfscope}%
\definecolor{textcolor}{rgb}{0.000000,0.000000,0.000000}%
\pgfsetstrokecolor{textcolor}%
\pgfsetfillcolor{textcolor}%
\pgftext[x=1.902222in,y=0.430778in,,top]{\color{textcolor}\rmfamily\fontsize{10.000000}{12.000000}\selectfont \(\displaystyle {10^{2}}\)}%
\end{pgfscope}%
\begin{pgfscope}%
\pgfsetbuttcap%
\pgfsetroundjoin%
\definecolor{currentfill}{rgb}{0.000000,0.000000,0.000000}%
\pgfsetfillcolor{currentfill}%
\pgfsetlinewidth{0.803000pt}%
\definecolor{currentstroke}{rgb}{0.000000,0.000000,0.000000}%
\pgfsetstrokecolor{currentstroke}%
\pgfsetdash{}{0pt}%
\pgfsys@defobject{currentmarker}{\pgfqpoint{0.000000in}{-0.048611in}}{\pgfqpoint{0.000000in}{0.000000in}}{%
\pgfpathmoveto{\pgfqpoint{0.000000in}{0.000000in}}%
\pgfpathlineto{\pgfqpoint{0.000000in}{-0.048611in}}%
\pgfusepath{stroke,fill}%
}%
\begin{pgfscope}%
\pgfsys@transformshift{3.004444in}{0.528000in}%
\pgfsys@useobject{currentmarker}{}%
\end{pgfscope}%
\end{pgfscope}%
\begin{pgfscope}%
\definecolor{textcolor}{rgb}{0.000000,0.000000,0.000000}%
\pgfsetstrokecolor{textcolor}%
\pgfsetfillcolor{textcolor}%
\pgftext[x=3.004444in,y=0.430778in,,top]{\color{textcolor}\rmfamily\fontsize{10.000000}{12.000000}\selectfont \(\displaystyle {10^{4}}\)}%
\end{pgfscope}%
\begin{pgfscope}%
\pgfsetbuttcap%
\pgfsetroundjoin%
\definecolor{currentfill}{rgb}{0.000000,0.000000,0.000000}%
\pgfsetfillcolor{currentfill}%
\pgfsetlinewidth{0.803000pt}%
\definecolor{currentstroke}{rgb}{0.000000,0.000000,0.000000}%
\pgfsetstrokecolor{currentstroke}%
\pgfsetdash{}{0pt}%
\pgfsys@defobject{currentmarker}{\pgfqpoint{0.000000in}{-0.048611in}}{\pgfqpoint{0.000000in}{0.000000in}}{%
\pgfpathmoveto{\pgfqpoint{0.000000in}{0.000000in}}%
\pgfpathlineto{\pgfqpoint{0.000000in}{-0.048611in}}%
\pgfusepath{stroke,fill}%
}%
\begin{pgfscope}%
\pgfsys@transformshift{4.106667in}{0.528000in}%
\pgfsys@useobject{currentmarker}{}%
\end{pgfscope}%
\end{pgfscope}%
\begin{pgfscope}%
\definecolor{textcolor}{rgb}{0.000000,0.000000,0.000000}%
\pgfsetstrokecolor{textcolor}%
\pgfsetfillcolor{textcolor}%
\pgftext[x=4.106667in,y=0.430778in,,top]{\color{textcolor}\rmfamily\fontsize{10.000000}{12.000000}\selectfont \(\displaystyle {10^{6}}\)}%
\end{pgfscope}%
\begin{pgfscope}%
\pgfsetbuttcap%
\pgfsetroundjoin%
\definecolor{currentfill}{rgb}{0.000000,0.000000,0.000000}%
\pgfsetfillcolor{currentfill}%
\pgfsetlinewidth{0.803000pt}%
\definecolor{currentstroke}{rgb}{0.000000,0.000000,0.000000}%
\pgfsetstrokecolor{currentstroke}%
\pgfsetdash{}{0pt}%
\pgfsys@defobject{currentmarker}{\pgfqpoint{0.000000in}{-0.048611in}}{\pgfqpoint{0.000000in}{0.000000in}}{%
\pgfpathmoveto{\pgfqpoint{0.000000in}{0.000000in}}%
\pgfpathlineto{\pgfqpoint{0.000000in}{-0.048611in}}%
\pgfusepath{stroke,fill}%
}%
\begin{pgfscope}%
\pgfsys@transformshift{5.208889in}{0.528000in}%
\pgfsys@useobject{currentmarker}{}%
\end{pgfscope}%
\end{pgfscope}%
\begin{pgfscope}%
\definecolor{textcolor}{rgb}{0.000000,0.000000,0.000000}%
\pgfsetstrokecolor{textcolor}%
\pgfsetfillcolor{textcolor}%
\pgftext[x=5.208889in,y=0.430778in,,top]{\color{textcolor}\rmfamily\fontsize{10.000000}{12.000000}\selectfont \(\displaystyle {10^{8}}\)}%
\end{pgfscope}%
\begin{pgfscope}%
\definecolor{textcolor}{rgb}{0.000000,0.000000,0.000000}%
\pgfsetstrokecolor{textcolor}%
\pgfsetfillcolor{textcolor}%
\pgftext[x=3.280000in,y=0.251889in,,top]{\color{textcolor}\rmfamily\fontsize{10.000000}{12.000000}\selectfont k}%
\end{pgfscope}%
\begin{pgfscope}%
\pgfsetbuttcap%
\pgfsetroundjoin%
\definecolor{currentfill}{rgb}{0.000000,0.000000,0.000000}%
\pgfsetfillcolor{currentfill}%
\pgfsetlinewidth{0.803000pt}%
\definecolor{currentstroke}{rgb}{0.000000,0.000000,0.000000}%
\pgfsetstrokecolor{currentstroke}%
\pgfsetdash{}{0pt}%
\pgfsys@defobject{currentmarker}{\pgfqpoint{-0.048611in}{0.000000in}}{\pgfqpoint{-0.000000in}{0.000000in}}{%
\pgfpathmoveto{\pgfqpoint{-0.000000in}{0.000000in}}%
\pgfpathlineto{\pgfqpoint{-0.048611in}{0.000000in}}%
\pgfusepath{stroke,fill}%
}%
\begin{pgfscope}%
\pgfsys@transformshift{0.800000in}{0.528000in}%
\pgfsys@useobject{currentmarker}{}%
\end{pgfscope}%
\end{pgfscope}%
\begin{pgfscope}%
\definecolor{textcolor}{rgb}{0.000000,0.000000,0.000000}%
\pgfsetstrokecolor{textcolor}%
\pgfsetfillcolor{textcolor}%
\pgftext[x=0.633333in, y=0.479806in, left, base]{\color{textcolor}\rmfamily\fontsize{10.000000}{12.000000}\selectfont \(\displaystyle {0}\)}%
\end{pgfscope}%
\begin{pgfscope}%
\pgfsetbuttcap%
\pgfsetroundjoin%
\definecolor{currentfill}{rgb}{0.000000,0.000000,0.000000}%
\pgfsetfillcolor{currentfill}%
\pgfsetlinewidth{0.803000pt}%
\definecolor{currentstroke}{rgb}{0.000000,0.000000,0.000000}%
\pgfsetstrokecolor{currentstroke}%
\pgfsetdash{}{0pt}%
\pgfsys@defobject{currentmarker}{\pgfqpoint{-0.048611in}{0.000000in}}{\pgfqpoint{-0.000000in}{0.000000in}}{%
\pgfpathmoveto{\pgfqpoint{-0.000000in}{0.000000in}}%
\pgfpathlineto{\pgfqpoint{-0.048611in}{0.000000in}}%
\pgfusepath{stroke,fill}%
}%
\begin{pgfscope}%
\pgfsys@transformshift{0.800000in}{0.990000in}%
\pgfsys@useobject{currentmarker}{}%
\end{pgfscope}%
\end{pgfscope}%
\begin{pgfscope}%
\definecolor{textcolor}{rgb}{0.000000,0.000000,0.000000}%
\pgfsetstrokecolor{textcolor}%
\pgfsetfillcolor{textcolor}%
\pgftext[x=0.563888in, y=0.941806in, left, base]{\color{textcolor}\rmfamily\fontsize{10.000000}{12.000000}\selectfont \(\displaystyle {25}\)}%
\end{pgfscope}%
\begin{pgfscope}%
\pgfsetbuttcap%
\pgfsetroundjoin%
\definecolor{currentfill}{rgb}{0.000000,0.000000,0.000000}%
\pgfsetfillcolor{currentfill}%
\pgfsetlinewidth{0.803000pt}%
\definecolor{currentstroke}{rgb}{0.000000,0.000000,0.000000}%
\pgfsetstrokecolor{currentstroke}%
\pgfsetdash{}{0pt}%
\pgfsys@defobject{currentmarker}{\pgfqpoint{-0.048611in}{0.000000in}}{\pgfqpoint{-0.000000in}{0.000000in}}{%
\pgfpathmoveto{\pgfqpoint{-0.000000in}{0.000000in}}%
\pgfpathlineto{\pgfqpoint{-0.048611in}{0.000000in}}%
\pgfusepath{stroke,fill}%
}%
\begin{pgfscope}%
\pgfsys@transformshift{0.800000in}{1.452000in}%
\pgfsys@useobject{currentmarker}{}%
\end{pgfscope}%
\end{pgfscope}%
\begin{pgfscope}%
\definecolor{textcolor}{rgb}{0.000000,0.000000,0.000000}%
\pgfsetstrokecolor{textcolor}%
\pgfsetfillcolor{textcolor}%
\pgftext[x=0.563888in, y=1.403806in, left, base]{\color{textcolor}\rmfamily\fontsize{10.000000}{12.000000}\selectfont \(\displaystyle {50}\)}%
\end{pgfscope}%
\begin{pgfscope}%
\pgfsetbuttcap%
\pgfsetroundjoin%
\definecolor{currentfill}{rgb}{0.000000,0.000000,0.000000}%
\pgfsetfillcolor{currentfill}%
\pgfsetlinewidth{0.803000pt}%
\definecolor{currentstroke}{rgb}{0.000000,0.000000,0.000000}%
\pgfsetstrokecolor{currentstroke}%
\pgfsetdash{}{0pt}%
\pgfsys@defobject{currentmarker}{\pgfqpoint{-0.048611in}{0.000000in}}{\pgfqpoint{-0.000000in}{0.000000in}}{%
\pgfpathmoveto{\pgfqpoint{-0.000000in}{0.000000in}}%
\pgfpathlineto{\pgfqpoint{-0.048611in}{0.000000in}}%
\pgfusepath{stroke,fill}%
}%
\begin{pgfscope}%
\pgfsys@transformshift{0.800000in}{1.914000in}%
\pgfsys@useobject{currentmarker}{}%
\end{pgfscope}%
\end{pgfscope}%
\begin{pgfscope}%
\definecolor{textcolor}{rgb}{0.000000,0.000000,0.000000}%
\pgfsetstrokecolor{textcolor}%
\pgfsetfillcolor{textcolor}%
\pgftext[x=0.563888in, y=1.865806in, left, base]{\color{textcolor}\rmfamily\fontsize{10.000000}{12.000000}\selectfont \(\displaystyle {75}\)}%
\end{pgfscope}%
\begin{pgfscope}%
\pgfsetbuttcap%
\pgfsetroundjoin%
\definecolor{currentfill}{rgb}{0.000000,0.000000,0.000000}%
\pgfsetfillcolor{currentfill}%
\pgfsetlinewidth{0.803000pt}%
\definecolor{currentstroke}{rgb}{0.000000,0.000000,0.000000}%
\pgfsetstrokecolor{currentstroke}%
\pgfsetdash{}{0pt}%
\pgfsys@defobject{currentmarker}{\pgfqpoint{-0.048611in}{0.000000in}}{\pgfqpoint{-0.000000in}{0.000000in}}{%
\pgfpathmoveto{\pgfqpoint{-0.000000in}{0.000000in}}%
\pgfpathlineto{\pgfqpoint{-0.048611in}{0.000000in}}%
\pgfusepath{stroke,fill}%
}%
\begin{pgfscope}%
\pgfsys@transformshift{0.800000in}{2.376000in}%
\pgfsys@useobject{currentmarker}{}%
\end{pgfscope}%
\end{pgfscope}%
\begin{pgfscope}%
\definecolor{textcolor}{rgb}{0.000000,0.000000,0.000000}%
\pgfsetstrokecolor{textcolor}%
\pgfsetfillcolor{textcolor}%
\pgftext[x=0.494444in, y=2.327806in, left, base]{\color{textcolor}\rmfamily\fontsize{10.000000}{12.000000}\selectfont \(\displaystyle {100}\)}%
\end{pgfscope}%
\begin{pgfscope}%
\pgfsetbuttcap%
\pgfsetroundjoin%
\definecolor{currentfill}{rgb}{0.000000,0.000000,0.000000}%
\pgfsetfillcolor{currentfill}%
\pgfsetlinewidth{0.803000pt}%
\definecolor{currentstroke}{rgb}{0.000000,0.000000,0.000000}%
\pgfsetstrokecolor{currentstroke}%
\pgfsetdash{}{0pt}%
\pgfsys@defobject{currentmarker}{\pgfqpoint{-0.048611in}{0.000000in}}{\pgfqpoint{-0.000000in}{0.000000in}}{%
\pgfpathmoveto{\pgfqpoint{-0.000000in}{0.000000in}}%
\pgfpathlineto{\pgfqpoint{-0.048611in}{0.000000in}}%
\pgfusepath{stroke,fill}%
}%
\begin{pgfscope}%
\pgfsys@transformshift{0.800000in}{2.838000in}%
\pgfsys@useobject{currentmarker}{}%
\end{pgfscope}%
\end{pgfscope}%
\begin{pgfscope}%
\definecolor{textcolor}{rgb}{0.000000,0.000000,0.000000}%
\pgfsetstrokecolor{textcolor}%
\pgfsetfillcolor{textcolor}%
\pgftext[x=0.494444in, y=2.789806in, left, base]{\color{textcolor}\rmfamily\fontsize{10.000000}{12.000000}\selectfont \(\displaystyle {125}\)}%
\end{pgfscope}%
\begin{pgfscope}%
\pgfsetbuttcap%
\pgfsetroundjoin%
\definecolor{currentfill}{rgb}{0.000000,0.000000,0.000000}%
\pgfsetfillcolor{currentfill}%
\pgfsetlinewidth{0.803000pt}%
\definecolor{currentstroke}{rgb}{0.000000,0.000000,0.000000}%
\pgfsetstrokecolor{currentstroke}%
\pgfsetdash{}{0pt}%
\pgfsys@defobject{currentmarker}{\pgfqpoint{-0.048611in}{0.000000in}}{\pgfqpoint{-0.000000in}{0.000000in}}{%
\pgfpathmoveto{\pgfqpoint{-0.000000in}{0.000000in}}%
\pgfpathlineto{\pgfqpoint{-0.048611in}{0.000000in}}%
\pgfusepath{stroke,fill}%
}%
\begin{pgfscope}%
\pgfsys@transformshift{0.800000in}{3.300000in}%
\pgfsys@useobject{currentmarker}{}%
\end{pgfscope}%
\end{pgfscope}%
\begin{pgfscope}%
\definecolor{textcolor}{rgb}{0.000000,0.000000,0.000000}%
\pgfsetstrokecolor{textcolor}%
\pgfsetfillcolor{textcolor}%
\pgftext[x=0.494444in, y=3.251806in, left, base]{\color{textcolor}\rmfamily\fontsize{10.000000}{12.000000}\selectfont \(\displaystyle {150}\)}%
\end{pgfscope}%
\begin{pgfscope}%
\pgfsetbuttcap%
\pgfsetroundjoin%
\definecolor{currentfill}{rgb}{0.000000,0.000000,0.000000}%
\pgfsetfillcolor{currentfill}%
\pgfsetlinewidth{0.803000pt}%
\definecolor{currentstroke}{rgb}{0.000000,0.000000,0.000000}%
\pgfsetstrokecolor{currentstroke}%
\pgfsetdash{}{0pt}%
\pgfsys@defobject{currentmarker}{\pgfqpoint{-0.048611in}{0.000000in}}{\pgfqpoint{-0.000000in}{0.000000in}}{%
\pgfpathmoveto{\pgfqpoint{-0.000000in}{0.000000in}}%
\pgfpathlineto{\pgfqpoint{-0.048611in}{0.000000in}}%
\pgfusepath{stroke,fill}%
}%
\begin{pgfscope}%
\pgfsys@transformshift{0.800000in}{3.762000in}%
\pgfsys@useobject{currentmarker}{}%
\end{pgfscope}%
\end{pgfscope}%
\begin{pgfscope}%
\definecolor{textcolor}{rgb}{0.000000,0.000000,0.000000}%
\pgfsetstrokecolor{textcolor}%
\pgfsetfillcolor{textcolor}%
\pgftext[x=0.494444in, y=3.713806in, left, base]{\color{textcolor}\rmfamily\fontsize{10.000000}{12.000000}\selectfont \(\displaystyle {175}\)}%
\end{pgfscope}%
\begin{pgfscope}%
\pgfsetbuttcap%
\pgfsetroundjoin%
\definecolor{currentfill}{rgb}{0.000000,0.000000,0.000000}%
\pgfsetfillcolor{currentfill}%
\pgfsetlinewidth{0.803000pt}%
\definecolor{currentstroke}{rgb}{0.000000,0.000000,0.000000}%
\pgfsetstrokecolor{currentstroke}%
\pgfsetdash{}{0pt}%
\pgfsys@defobject{currentmarker}{\pgfqpoint{-0.048611in}{0.000000in}}{\pgfqpoint{-0.000000in}{0.000000in}}{%
\pgfpathmoveto{\pgfqpoint{-0.000000in}{0.000000in}}%
\pgfpathlineto{\pgfqpoint{-0.048611in}{0.000000in}}%
\pgfusepath{stroke,fill}%
}%
\begin{pgfscope}%
\pgfsys@transformshift{0.800000in}{4.224000in}%
\pgfsys@useobject{currentmarker}{}%
\end{pgfscope}%
\end{pgfscope}%
\begin{pgfscope}%
\definecolor{textcolor}{rgb}{0.000000,0.000000,0.000000}%
\pgfsetstrokecolor{textcolor}%
\pgfsetfillcolor{textcolor}%
\pgftext[x=0.494444in, y=4.175806in, left, base]{\color{textcolor}\rmfamily\fontsize{10.000000}{12.000000}\selectfont \(\displaystyle {200}\)}%
\end{pgfscope}%
\begin{pgfscope}%
\definecolor{textcolor}{rgb}{0.000000,0.000000,0.000000}%
\pgfsetstrokecolor{textcolor}%
\pgfsetfillcolor{textcolor}%
\pgftext[x=0.438888in,y=2.376000in,,bottom,rotate=90.000000]{\color{textcolor}\rmfamily\fontsize{10.000000}{12.000000}\selectfont time/k (ns)}%
\end{pgfscope}%
\begin{pgfscope}%
\pgfpathrectangle{\pgfqpoint{0.800000in}{0.528000in}}{\pgfqpoint{4.960000in}{3.696000in}}%
\pgfusepath{clip}%
\pgfsetrectcap%
\pgfsetroundjoin%
\pgfsetlinewidth{1.505625pt}%
\definecolor{currentstroke}{rgb}{0.121569,0.466667,0.705882}%
\pgfsetstrokecolor{currentstroke}%
\pgfsetdash{}{0pt}%
\pgfpathmoveto{\pgfqpoint{0.800000in}{0.811253in}}%
\pgfpathlineto{\pgfqpoint{0.965901in}{0.845524in}}%
\pgfpathlineto{\pgfqpoint{1.062947in}{0.861238in}}%
\pgfpathlineto{\pgfqpoint{1.185210in}{0.885137in}}%
\pgfpathlineto{\pgfqpoint{1.297703in}{0.931871in}}%
\pgfpathlineto{\pgfqpoint{1.413907in}{0.929392in}}%
\pgfpathlineto{\pgfqpoint{1.528690in}{1.001910in}}%
\pgfpathlineto{\pgfqpoint{1.644015in}{1.037328in}}%
\pgfpathlineto{\pgfqpoint{1.759133in}{1.078486in}}%
\pgfpathlineto{\pgfqpoint{1.874330in}{1.112321in}}%
\pgfpathlineto{\pgfqpoint{1.989498in}{1.150012in}}%
\pgfpathlineto{\pgfqpoint{2.104676in}{1.188020in}}%
\pgfpathlineto{\pgfqpoint{2.219850in}{1.225074in}}%
\pgfpathlineto{\pgfqpoint{2.335026in}{1.266625in}}%
\pgfpathlineto{\pgfqpoint{2.450201in}{1.306765in}}%
\pgfpathlineto{\pgfqpoint{2.565377in}{1.341888in}}%
\pgfpathlineto{\pgfqpoint{2.680552in}{1.376586in}}%
\pgfpathlineto{\pgfqpoint{2.795728in}{1.415956in}}%
\pgfpathlineto{\pgfqpoint{2.910903in}{1.456768in}}%
\pgfpathlineto{\pgfqpoint{3.026079in}{1.505723in}}%
\pgfpathlineto{\pgfqpoint{3.141254in}{1.532173in}}%
\pgfpathlineto{\pgfqpoint{3.256429in}{1.562345in}}%
\pgfpathlineto{\pgfqpoint{3.371605in}{1.599610in}}%
\pgfpathlineto{\pgfqpoint{3.486780in}{1.642310in}}%
\pgfpathlineto{\pgfqpoint{3.601956in}{1.690366in}}%
\pgfpathlineto{\pgfqpoint{3.717131in}{1.739662in}}%
\pgfpathlineto{\pgfqpoint{3.832306in}{1.777067in}}%
\pgfpathlineto{\pgfqpoint{3.947482in}{1.834568in}}%
\pgfpathlineto{\pgfqpoint{4.062657in}{1.869621in}}%
\pgfpathlineto{\pgfqpoint{4.177833in}{1.905919in}}%
\pgfpathlineto{\pgfqpoint{4.293008in}{1.956934in}}%
\pgfpathlineto{\pgfqpoint{4.408183in}{1.994888in}}%
\pgfpathlineto{\pgfqpoint{4.523359in}{2.060213in}}%
\pgfpathlineto{\pgfqpoint{4.638534in}{2.142353in}}%
\pgfpathlineto{\pgfqpoint{4.753710in}{2.306768in}}%
\pgfpathlineto{\pgfqpoint{4.868885in}{2.471758in}}%
\pgfpathlineto{\pgfqpoint{4.984061in}{2.623526in}}%
\pgfpathlineto{\pgfqpoint{5.099236in}{2.785881in}}%
\pgfpathlineto{\pgfqpoint{5.214411in}{2.942490in}}%
\pgfusepath{stroke}%
\end{pgfscope}%
\begin{pgfscope}%
\pgfpathrectangle{\pgfqpoint{0.800000in}{0.528000in}}{\pgfqpoint{4.960000in}{3.696000in}}%
\pgfusepath{clip}%
\pgfsetrectcap%
\pgfsetroundjoin%
\pgfsetlinewidth{1.505625pt}%
\definecolor{currentstroke}{rgb}{1.000000,0.498039,0.054902}%
\pgfsetstrokecolor{currentstroke}%
\pgfsetdash{}{0pt}%
\pgfpathmoveto{\pgfqpoint{0.800000in}{1.691216in}}%
\pgfpathlineto{\pgfqpoint{0.965901in}{1.572515in}}%
\pgfpathlineto{\pgfqpoint{1.062947in}{1.548475in}}%
\pgfpathlineto{\pgfqpoint{1.185210in}{1.529720in}}%
\pgfpathlineto{\pgfqpoint{1.297703in}{1.620235in}}%
\pgfpathlineto{\pgfqpoint{1.413907in}{1.567360in}}%
\pgfpathlineto{\pgfqpoint{1.528690in}{1.582890in}}%
\pgfpathlineto{\pgfqpoint{1.644015in}{1.568395in}}%
\pgfpathlineto{\pgfqpoint{1.759133in}{1.560595in}}%
\pgfpathlineto{\pgfqpoint{1.874330in}{2.036507in}}%
\pgfpathlineto{\pgfqpoint{1.989498in}{2.073310in}}%
\pgfpathlineto{\pgfqpoint{2.104676in}{2.069194in}}%
\pgfpathlineto{\pgfqpoint{2.219850in}{2.069409in}}%
\pgfpathlineto{\pgfqpoint{2.335026in}{2.041928in}}%
\pgfpathlineto{\pgfqpoint{2.450201in}{2.038793in}}%
\pgfpathlineto{\pgfqpoint{2.565377in}{2.022774in}}%
\pgfpathlineto{\pgfqpoint{2.680552in}{1.989320in}}%
\pgfpathlineto{\pgfqpoint{2.795728in}{1.990868in}}%
\pgfpathlineto{\pgfqpoint{2.910903in}{2.050422in}}%
\pgfpathlineto{\pgfqpoint{3.026079in}{2.136172in}}%
\pgfpathlineto{\pgfqpoint{3.141254in}{2.224629in}}%
\pgfpathlineto{\pgfqpoint{3.256429in}{2.311938in}}%
\pgfpathlineto{\pgfqpoint{3.371605in}{2.324261in}}%
\pgfpathlineto{\pgfqpoint{3.486780in}{2.373839in}}%
\pgfpathlineto{\pgfqpoint{3.601956in}{2.424034in}}%
\pgfpathlineto{\pgfqpoint{3.717131in}{2.486521in}}%
\pgfpathlineto{\pgfqpoint{3.832306in}{2.541394in}}%
\pgfpathlineto{\pgfqpoint{3.947482in}{2.701960in}}%
\pgfpathlineto{\pgfqpoint{4.062657in}{3.460175in}}%
\pgfpathlineto{\pgfqpoint{4.169072in}{4.234000in}}%
\pgfusepath{stroke}%
\end{pgfscope}%
\begin{pgfscope}%
\pgfpathrectangle{\pgfqpoint{0.800000in}{0.528000in}}{\pgfqpoint{4.960000in}{3.696000in}}%
\pgfusepath{clip}%
\pgfsetrectcap%
\pgfsetroundjoin%
\pgfsetlinewidth{1.505625pt}%
\definecolor{currentstroke}{rgb}{0.172549,0.627451,0.172549}%
\pgfsetstrokecolor{currentstroke}%
\pgfsetdash{}{0pt}%
\pgfpathmoveto{\pgfqpoint{0.800000in}{1.699712in}}%
\pgfpathlineto{\pgfqpoint{0.965901in}{1.678157in}}%
\pgfpathlineto{\pgfqpoint{1.062947in}{1.718103in}}%
\pgfpathlineto{\pgfqpoint{1.185210in}{1.774278in}}%
\pgfpathlineto{\pgfqpoint{1.297703in}{1.837125in}}%
\pgfpathlineto{\pgfqpoint{1.413907in}{1.844373in}}%
\pgfpathlineto{\pgfqpoint{1.528690in}{1.886168in}}%
\pgfpathlineto{\pgfqpoint{1.644015in}{1.889062in}}%
\pgfpathlineto{\pgfqpoint{1.759133in}{1.889917in}}%
\pgfpathlineto{\pgfqpoint{1.874330in}{2.423834in}}%
\pgfpathlineto{\pgfqpoint{1.989498in}{2.433430in}}%
\pgfpathlineto{\pgfqpoint{2.104676in}{2.437804in}}%
\pgfpathlineto{\pgfqpoint{2.219850in}{2.436295in}}%
\pgfpathlineto{\pgfqpoint{2.335026in}{2.389009in}}%
\pgfpathlineto{\pgfqpoint{2.450201in}{2.371419in}}%
\pgfpathlineto{\pgfqpoint{2.565377in}{2.355679in}}%
\pgfpathlineto{\pgfqpoint{2.680552in}{2.318892in}}%
\pgfpathlineto{\pgfqpoint{2.795728in}{2.307920in}}%
\pgfpathlineto{\pgfqpoint{2.910903in}{2.338973in}}%
\pgfpathlineto{\pgfqpoint{3.026079in}{2.405476in}}%
\pgfpathlineto{\pgfqpoint{3.141254in}{2.486958in}}%
\pgfpathlineto{\pgfqpoint{3.256429in}{2.548599in}}%
\pgfpathlineto{\pgfqpoint{3.371605in}{2.590082in}}%
\pgfpathlineto{\pgfqpoint{3.486780in}{2.596875in}}%
\pgfpathlineto{\pgfqpoint{3.601956in}{2.639835in}}%
\pgfpathlineto{\pgfqpoint{3.717131in}{2.685166in}}%
\pgfpathlineto{\pgfqpoint{3.832306in}{2.742112in}}%
\pgfpathlineto{\pgfqpoint{3.947482in}{2.888939in}}%
\pgfpathlineto{\pgfqpoint{4.062657in}{3.617841in}}%
\pgfpathlineto{\pgfqpoint{4.142289in}{4.234000in}}%
\pgfusepath{stroke}%
\end{pgfscope}%
\begin{pgfscope}%
\pgfpathrectangle{\pgfqpoint{0.800000in}{0.528000in}}{\pgfqpoint{4.960000in}{3.696000in}}%
\pgfusepath{clip}%
\pgfsetrectcap%
\pgfsetroundjoin%
\pgfsetlinewidth{1.505625pt}%
\definecolor{currentstroke}{rgb}{0.839216,0.152941,0.156863}%
\pgfsetstrokecolor{currentstroke}%
\pgfsetdash{}{0pt}%
\pgfpathmoveto{\pgfqpoint{4.915890in}{4.234000in}}%
\pgfpathlineto{\pgfqpoint{4.984061in}{3.715170in}}%
\pgfpathlineto{\pgfqpoint{5.099236in}{3.175079in}}%
\pgfpathlineto{\pgfqpoint{5.214411in}{2.790292in}}%
\pgfpathlineto{\pgfqpoint{5.329587in}{2.459915in}}%
\pgfpathlineto{\pgfqpoint{5.444762in}{2.162067in}}%
\pgfusepath{stroke}%
\end{pgfscope}%
\begin{pgfscope}%
\pgfpathrectangle{\pgfqpoint{0.800000in}{0.528000in}}{\pgfqpoint{4.960000in}{3.696000in}}%
\pgfusepath{clip}%
\pgfsetrectcap%
\pgfsetroundjoin%
\pgfsetlinewidth{1.505625pt}%
\definecolor{currentstroke}{rgb}{0.580392,0.403922,0.741176}%
\pgfsetstrokecolor{currentstroke}%
\pgfsetdash{}{0pt}%
\pgfpathmoveto{\pgfqpoint{0.800000in}{0.678709in}}%
\pgfpathlineto{\pgfqpoint{0.965901in}{0.675205in}}%
\pgfpathlineto{\pgfqpoint{1.062947in}{0.686742in}}%
\pgfpathlineto{\pgfqpoint{1.185210in}{0.701394in}}%
\pgfpathlineto{\pgfqpoint{1.297703in}{0.712229in}}%
\pgfpathlineto{\pgfqpoint{1.413907in}{0.745319in}}%
\pgfpathlineto{\pgfqpoint{1.528690in}{0.803919in}}%
\pgfpathlineto{\pgfqpoint{1.644015in}{0.883691in}}%
\pgfpathlineto{\pgfqpoint{1.759133in}{1.016036in}}%
\pgfpathlineto{\pgfqpoint{1.874330in}{1.163596in}}%
\pgfpathlineto{\pgfqpoint{1.989498in}{1.360843in}}%
\pgfpathlineto{\pgfqpoint{2.104676in}{1.624525in}}%
\pgfpathlineto{\pgfqpoint{2.219850in}{2.053461in}}%
\pgfpathlineto{\pgfqpoint{2.335026in}{2.736098in}}%
\pgfpathlineto{\pgfqpoint{2.450201in}{3.817298in}}%
\pgfpathlineto{\pgfqpoint{2.477674in}{4.234000in}}%
\pgfusepath{stroke}%
\end{pgfscope}%
\begin{pgfscope}%
\pgfpathrectangle{\pgfqpoint{0.800000in}{0.528000in}}{\pgfqpoint{4.960000in}{3.696000in}}%
\pgfusepath{clip}%
\pgfsetrectcap%
\pgfsetroundjoin%
\pgfsetlinewidth{1.505625pt}%
\definecolor{currentstroke}{rgb}{0.549020,0.337255,0.294118}%
\pgfsetstrokecolor{currentstroke}%
\pgfsetdash{}{0pt}%
\pgfpathmoveto{\pgfqpoint{0.800000in}{0.680997in}}%
\pgfpathlineto{\pgfqpoint{0.965901in}{0.687731in}}%
\pgfpathlineto{\pgfqpoint{1.062947in}{0.681728in}}%
\pgfpathlineto{\pgfqpoint{1.185210in}{0.690684in}}%
\pgfpathlineto{\pgfqpoint{1.297703in}{0.705243in}}%
\pgfpathlineto{\pgfqpoint{1.413907in}{0.730839in}}%
\pgfpathlineto{\pgfqpoint{1.528690in}{0.779369in}}%
\pgfpathlineto{\pgfqpoint{1.644015in}{0.850465in}}%
\pgfpathlineto{\pgfqpoint{1.759133in}{0.971018in}}%
\pgfpathlineto{\pgfqpoint{1.874330in}{1.095872in}}%
\pgfpathlineto{\pgfqpoint{1.989498in}{1.276474in}}%
\pgfpathlineto{\pgfqpoint{2.104676in}{1.541252in}}%
\pgfpathlineto{\pgfqpoint{2.219850in}{1.966639in}}%
\pgfpathlineto{\pgfqpoint{2.335026in}{2.631072in}}%
\pgfpathlineto{\pgfqpoint{2.450201in}{3.704890in}}%
\pgfpathlineto{\pgfqpoint{2.485496in}{4.234000in}}%
\pgfusepath{stroke}%
\end{pgfscope}%
\begin{pgfscope}%
\pgfpathrectangle{\pgfqpoint{0.800000in}{0.528000in}}{\pgfqpoint{4.960000in}{3.696000in}}%
\pgfusepath{clip}%
\pgfsetrectcap%
\pgfsetroundjoin%
\pgfsetlinewidth{1.505625pt}%
\definecolor{currentstroke}{rgb}{0.890196,0.466667,0.760784}%
\pgfsetstrokecolor{currentstroke}%
\pgfsetdash{}{0pt}%
\pgfpathmoveto{\pgfqpoint{0.800000in}{2.091974in}}%
\pgfpathlineto{\pgfqpoint{0.965901in}{1.579484in}}%
\pgfpathlineto{\pgfqpoint{1.062947in}{1.499624in}}%
\pgfpathlineto{\pgfqpoint{1.185210in}{1.516320in}}%
\pgfpathlineto{\pgfqpoint{1.297703in}{1.582376in}}%
\pgfpathlineto{\pgfqpoint{1.413907in}{1.524640in}}%
\pgfpathlineto{\pgfqpoint{1.528690in}{1.525382in}}%
\pgfpathlineto{\pgfqpoint{1.644015in}{1.490938in}}%
\pgfpathlineto{\pgfqpoint{1.759133in}{1.466361in}}%
\pgfpathlineto{\pgfqpoint{1.874330in}{1.959543in}}%
\pgfpathlineto{\pgfqpoint{1.989498in}{1.979197in}}%
\pgfpathlineto{\pgfqpoint{2.104676in}{1.979813in}}%
\pgfpathlineto{\pgfqpoint{2.219850in}{1.989038in}}%
\pgfpathlineto{\pgfqpoint{2.335026in}{1.945285in}}%
\pgfpathlineto{\pgfqpoint{2.450201in}{1.925618in}}%
\pgfpathlineto{\pgfqpoint{2.565377in}{1.906961in}}%
\pgfpathlineto{\pgfqpoint{2.680552in}{1.880544in}}%
\pgfpathlineto{\pgfqpoint{2.795728in}{1.863335in}}%
\pgfpathlineto{\pgfqpoint{2.910903in}{1.921500in}}%
\pgfpathlineto{\pgfqpoint{3.026079in}{2.013945in}}%
\pgfpathlineto{\pgfqpoint{3.141254in}{2.094496in}}%
\pgfpathlineto{\pgfqpoint{3.256429in}{2.177339in}}%
\pgfpathlineto{\pgfqpoint{3.371605in}{2.212588in}}%
\pgfpathlineto{\pgfqpoint{3.486780in}{2.247377in}}%
\pgfpathlineto{\pgfqpoint{3.601956in}{2.290549in}}%
\pgfpathlineto{\pgfqpoint{3.717131in}{2.328883in}}%
\pgfpathlineto{\pgfqpoint{3.832306in}{2.393607in}}%
\pgfpathlineto{\pgfqpoint{3.947482in}{2.544848in}}%
\pgfpathlineto{\pgfqpoint{4.062657in}{3.221978in}}%
\pgfpathlineto{\pgfqpoint{4.177833in}{3.954344in}}%
\pgfpathlineto{\pgfqpoint{4.211233in}{4.234000in}}%
\pgfusepath{stroke}%
\end{pgfscope}%
\begin{pgfscope}%
\pgfpathrectangle{\pgfqpoint{0.800000in}{0.528000in}}{\pgfqpoint{4.960000in}{3.696000in}}%
\pgfusepath{clip}%
\pgfsetrectcap%
\pgfsetroundjoin%
\pgfsetlinewidth{1.505625pt}%
\definecolor{currentstroke}{rgb}{0.498039,0.498039,0.498039}%
\pgfsetstrokecolor{currentstroke}%
\pgfsetdash{}{0pt}%
\pgfusepath{stroke}%
\end{pgfscope}%
\begin{pgfscope}%
\pgfpathrectangle{\pgfqpoint{0.800000in}{0.528000in}}{\pgfqpoint{4.960000in}{3.696000in}}%
\pgfusepath{clip}%
\pgfsetrectcap%
\pgfsetroundjoin%
\pgfsetlinewidth{1.505625pt}%
\definecolor{currentstroke}{rgb}{0.737255,0.741176,0.133333}%
\pgfsetstrokecolor{currentstroke}%
\pgfsetdash{}{0pt}%
\pgfpathmoveto{\pgfqpoint{4.577850in}{4.234000in}}%
\pgfpathlineto{\pgfqpoint{4.638534in}{3.470275in}}%
\pgfpathlineto{\pgfqpoint{4.753710in}{2.583960in}}%
\pgfpathlineto{\pgfqpoint{4.868885in}{2.037353in}}%
\pgfpathlineto{\pgfqpoint{4.984061in}{1.700236in}}%
\pgfpathlineto{\pgfqpoint{5.099236in}{1.493712in}}%
\pgfpathlineto{\pgfqpoint{5.214411in}{1.363815in}}%
\pgfpathlineto{\pgfqpoint{5.329587in}{1.279200in}}%
\pgfpathlineto{\pgfqpoint{5.444762in}{1.223157in}}%
\pgfpathlineto{\pgfqpoint{5.559938in}{1.186826in}}%
\pgfpathlineto{\pgfqpoint{5.675113in}{1.153726in}}%
\pgfusepath{stroke}%
\end{pgfscope}%
\begin{pgfscope}%
\pgfsetrectcap%
\pgfsetmiterjoin%
\pgfsetlinewidth{0.803000pt}%
\definecolor{currentstroke}{rgb}{0.000000,0.000000,0.000000}%
\pgfsetstrokecolor{currentstroke}%
\pgfsetdash{}{0pt}%
\pgfpathmoveto{\pgfqpoint{0.800000in}{0.528000in}}%
\pgfpathlineto{\pgfqpoint{0.800000in}{4.224000in}}%
\pgfusepath{stroke}%
\end{pgfscope}%
\begin{pgfscope}%
\pgfsetrectcap%
\pgfsetmiterjoin%
\pgfsetlinewidth{0.803000pt}%
\definecolor{currentstroke}{rgb}{0.000000,0.000000,0.000000}%
\pgfsetstrokecolor{currentstroke}%
\pgfsetdash{}{0pt}%
\pgfpathmoveto{\pgfqpoint{5.760000in}{0.528000in}}%
\pgfpathlineto{\pgfqpoint{5.760000in}{4.224000in}}%
\pgfusepath{stroke}%
\end{pgfscope}%
\begin{pgfscope}%
\pgfsetrectcap%
\pgfsetmiterjoin%
\pgfsetlinewidth{0.803000pt}%
\definecolor{currentstroke}{rgb}{0.000000,0.000000,0.000000}%
\pgfsetstrokecolor{currentstroke}%
\pgfsetdash{}{0pt}%
\pgfpathmoveto{\pgfqpoint{0.800000in}{0.528000in}}%
\pgfpathlineto{\pgfqpoint{5.760000in}{0.528000in}}%
\pgfusepath{stroke}%
\end{pgfscope}%
\begin{pgfscope}%
\pgfsetrectcap%
\pgfsetmiterjoin%
\pgfsetlinewidth{0.803000pt}%
\definecolor{currentstroke}{rgb}{0.000000,0.000000,0.000000}%
\pgfsetstrokecolor{currentstroke}%
\pgfsetdash{}{0pt}%
\pgfpathmoveto{\pgfqpoint{0.800000in}{4.224000in}}%
\pgfpathlineto{\pgfqpoint{5.760000in}{4.224000in}}%
\pgfusepath{stroke}%
\end{pgfscope}%
\begin{pgfscope}%
\definecolor{textcolor}{rgb}{0.000000,0.000000,0.000000}%
\pgfsetstrokecolor{textcolor}%
\pgfsetfillcolor{textcolor}%
\pgftext[x=3.280000in,y=4.260960in,,base]{\color{textcolor}\rmfamily\fontsize{10.000000}{12.000000}\selectfont Random order n=7.0E+08}%
\end{pgfscope}%
\begin{pgfscope}%
\pgfsetbuttcap%
\pgfsetmiterjoin%
\definecolor{currentfill}{rgb}{1.000000,1.000000,1.000000}%
\pgfsetfillcolor{currentfill}%
\pgfsetfillopacity{0.800000}%
\pgfsetlinewidth{1.003750pt}%
\definecolor{currentstroke}{rgb}{0.800000,0.800000,0.800000}%
\pgfsetstrokecolor{currentstroke}%
\pgfsetstrokeopacity{0.800000}%
\pgfsetdash{}{0pt}%
\pgfpathmoveto{\pgfqpoint{0.897222in}{2.364417in}}%
\pgfpathlineto{\pgfqpoint{2.280278in}{2.364417in}}%
\pgfpathquadraticcurveto{\pgfqpoint{2.308056in}{2.364417in}}{\pgfqpoint{2.308056in}{2.392194in}}%
\pgfpathlineto{\pgfqpoint{2.308056in}{4.126778in}}%
\pgfpathquadraticcurveto{\pgfqpoint{2.308056in}{4.154556in}}{\pgfqpoint{2.280278in}{4.154556in}}%
\pgfpathlineto{\pgfqpoint{0.897222in}{4.154556in}}%
\pgfpathquadraticcurveto{\pgfqpoint{0.869444in}{4.154556in}}{\pgfqpoint{0.869444in}{4.126778in}}%
\pgfpathlineto{\pgfqpoint{0.869444in}{2.392194in}}%
\pgfpathquadraticcurveto{\pgfqpoint{0.869444in}{2.364417in}}{\pgfqpoint{0.897222in}{2.364417in}}%
\pgfpathlineto{\pgfqpoint{0.897222in}{2.364417in}}%
\pgfpathclose%
\pgfusepath{stroke,fill}%
\end{pgfscope}%
\begin{pgfscope}%
\pgfsetrectcap%
\pgfsetroundjoin%
\pgfsetlinewidth{1.505625pt}%
\definecolor{currentstroke}{rgb}{0.121569,0.466667,0.705882}%
\pgfsetstrokecolor{currentstroke}%
\pgfsetdash{}{0pt}%
\pgfpathmoveto{\pgfqpoint{0.925000in}{4.050389in}}%
\pgfpathlineto{\pgfqpoint{1.063889in}{4.050389in}}%
\pgfpathlineto{\pgfqpoint{1.202778in}{4.050389in}}%
\pgfusepath{stroke}%
\end{pgfscope}%
\begin{pgfscope}%
\definecolor{textcolor}{rgb}{0.000000,0.000000,0.000000}%
\pgfsetstrokecolor{textcolor}%
\pgfsetfillcolor{textcolor}%
\pgftext[x=1.313889in,y=4.001778in,left,base]{\color{textcolor}\rmfamily\fontsize{10.000000}{12.000000}\selectfont cardchoose}%
\end{pgfscope}%
\begin{pgfscope}%
\pgfsetrectcap%
\pgfsetroundjoin%
\pgfsetlinewidth{1.505625pt}%
\definecolor{currentstroke}{rgb}{1.000000,0.498039,0.054902}%
\pgfsetstrokecolor{currentstroke}%
\pgfsetdash{}{0pt}%
\pgfpathmoveto{\pgfqpoint{0.925000in}{3.856083in}}%
\pgfpathlineto{\pgfqpoint{1.063889in}{3.856083in}}%
\pgfpathlineto{\pgfqpoint{1.202778in}{3.856083in}}%
\pgfusepath{stroke}%
\end{pgfscope}%
\begin{pgfscope}%
\definecolor{textcolor}{rgb}{0.000000,0.000000,0.000000}%
\pgfsetstrokecolor{textcolor}%
\pgfsetfillcolor{textcolor}%
\pgftext[x=1.313889in,y=3.807472in,left,base]{\color{textcolor}\rmfamily\fontsize{10.000000}{12.000000}\selectfont floydf2}%
\end{pgfscope}%
\begin{pgfscope}%
\pgfsetrectcap%
\pgfsetroundjoin%
\pgfsetlinewidth{1.505625pt}%
\definecolor{currentstroke}{rgb}{0.172549,0.627451,0.172549}%
\pgfsetstrokecolor{currentstroke}%
\pgfsetdash{}{0pt}%
\pgfpathmoveto{\pgfqpoint{0.925000in}{3.660944in}}%
\pgfpathlineto{\pgfqpoint{1.063889in}{3.660944in}}%
\pgfpathlineto{\pgfqpoint{1.202778in}{3.660944in}}%
\pgfusepath{stroke}%
\end{pgfscope}%
\begin{pgfscope}%
\definecolor{textcolor}{rgb}{0.000000,0.000000,0.000000}%
\pgfsetstrokecolor{textcolor}%
\pgfsetfillcolor{textcolor}%
\pgftext[x=1.313889in,y=3.612333in,left,base]{\color{textcolor}\rmfamily\fontsize{10.000000}{12.000000}\selectfont hsel}%
\end{pgfscope}%
\begin{pgfscope}%
\pgfsetrectcap%
\pgfsetroundjoin%
\pgfsetlinewidth{1.505625pt}%
\definecolor{currentstroke}{rgb}{0.839216,0.152941,0.156863}%
\pgfsetstrokecolor{currentstroke}%
\pgfsetdash{}{0pt}%
\pgfpathmoveto{\pgfqpoint{0.925000in}{3.467333in}}%
\pgfpathlineto{\pgfqpoint{1.063889in}{3.467333in}}%
\pgfpathlineto{\pgfqpoint{1.202778in}{3.467333in}}%
\pgfusepath{stroke}%
\end{pgfscope}%
\begin{pgfscope}%
\definecolor{textcolor}{rgb}{0.000000,0.000000,0.000000}%
\pgfsetstrokecolor{textcolor}%
\pgfsetfillcolor{textcolor}%
\pgftext[x=1.313889in,y=3.418722in,left,base]{\color{textcolor}\rmfamily\fontsize{10.000000}{12.000000}\selectfont iterativechoose}%
\end{pgfscope}%
\begin{pgfscope}%
\pgfsetrectcap%
\pgfsetroundjoin%
\pgfsetlinewidth{1.505625pt}%
\definecolor{currentstroke}{rgb}{0.580392,0.403922,0.741176}%
\pgfsetstrokecolor{currentstroke}%
\pgfsetdash{}{0pt}%
\pgfpathmoveto{\pgfqpoint{0.925000in}{3.273028in}}%
\pgfpathlineto{\pgfqpoint{1.063889in}{3.273028in}}%
\pgfpathlineto{\pgfqpoint{1.202778in}{3.273028in}}%
\pgfusepath{stroke}%
\end{pgfscope}%
\begin{pgfscope}%
\definecolor{textcolor}{rgb}{0.000000,0.000000,0.000000}%
\pgfsetstrokecolor{textcolor}%
\pgfsetfillcolor{textcolor}%
\pgftext[x=1.313889in,y=3.224417in,left,base]{\color{textcolor}\rmfamily\fontsize{10.000000}{12.000000}\selectfont quadraticf2}%
\end{pgfscope}%
\begin{pgfscope}%
\pgfsetrectcap%
\pgfsetroundjoin%
\pgfsetlinewidth{1.505625pt}%
\definecolor{currentstroke}{rgb}{0.549020,0.337255,0.294118}%
\pgfsetstrokecolor{currentstroke}%
\pgfsetdash{}{0pt}%
\pgfpathmoveto{\pgfqpoint{0.925000in}{3.079417in}}%
\pgfpathlineto{\pgfqpoint{1.063889in}{3.079417in}}%
\pgfpathlineto{\pgfqpoint{1.202778in}{3.079417in}}%
\pgfusepath{stroke}%
\end{pgfscope}%
\begin{pgfscope}%
\definecolor{textcolor}{rgb}{0.000000,0.000000,0.000000}%
\pgfsetstrokecolor{textcolor}%
\pgfsetfillcolor{textcolor}%
\pgftext[x=1.313889in,y=3.030806in,left,base]{\color{textcolor}\rmfamily\fontsize{10.000000}{12.000000}\selectfont quadraticreject}%
\end{pgfscope}%
\begin{pgfscope}%
\pgfsetrectcap%
\pgfsetroundjoin%
\pgfsetlinewidth{1.505625pt}%
\definecolor{currentstroke}{rgb}{0.890196,0.466667,0.760784}%
\pgfsetstrokecolor{currentstroke}%
\pgfsetdash{}{0pt}%
\pgfpathmoveto{\pgfqpoint{0.925000in}{2.884278in}}%
\pgfpathlineto{\pgfqpoint{1.063889in}{2.884278in}}%
\pgfpathlineto{\pgfqpoint{1.202778in}{2.884278in}}%
\pgfusepath{stroke}%
\end{pgfscope}%
\begin{pgfscope}%
\definecolor{textcolor}{rgb}{0.000000,0.000000,0.000000}%
\pgfsetstrokecolor{textcolor}%
\pgfsetfillcolor{textcolor}%
\pgftext[x=1.313889in,y=2.835667in,left,base]{\color{textcolor}\rmfamily\fontsize{10.000000}{12.000000}\selectfont rejectionsample}%
\end{pgfscope}%
\begin{pgfscope}%
\pgfsetrectcap%
\pgfsetroundjoin%
\pgfsetlinewidth{1.505625pt}%
\definecolor{currentstroke}{rgb}{0.498039,0.498039,0.498039}%
\pgfsetstrokecolor{currentstroke}%
\pgfsetdash{}{0pt}%
\pgfpathmoveto{\pgfqpoint{0.925000in}{2.689139in}}%
\pgfpathlineto{\pgfqpoint{1.063889in}{2.689139in}}%
\pgfpathlineto{\pgfqpoint{1.202778in}{2.689139in}}%
\pgfusepath{stroke}%
\end{pgfscope}%
\begin{pgfscope}%
\definecolor{textcolor}{rgb}{0.000000,0.000000,0.000000}%
\pgfsetstrokecolor{textcolor}%
\pgfsetfillcolor{textcolor}%
\pgftext[x=1.313889in,y=2.640528in,left,base]{\color{textcolor}\rmfamily\fontsize{10.000000}{12.000000}\selectfont reservoirsample}%
\end{pgfscope}%
\begin{pgfscope}%
\pgfsetrectcap%
\pgfsetroundjoin%
\pgfsetlinewidth{1.505625pt}%
\definecolor{currentstroke}{rgb}{0.737255,0.741176,0.133333}%
\pgfsetstrokecolor{currentstroke}%
\pgfsetdash{}{0pt}%
\pgfpathmoveto{\pgfqpoint{0.925000in}{2.495528in}}%
\pgfpathlineto{\pgfqpoint{1.063889in}{2.495528in}}%
\pgfpathlineto{\pgfqpoint{1.202778in}{2.495528in}}%
\pgfusepath{stroke}%
\end{pgfscope}%
\begin{pgfscope}%
\definecolor{textcolor}{rgb}{0.000000,0.000000,0.000000}%
\pgfsetstrokecolor{textcolor}%
\pgfsetfillcolor{textcolor}%
\pgftext[x=1.313889in,y=2.446917in,left,base]{\color{textcolor}\rmfamily\fontsize{10.000000}{12.000000}\selectfont select}%
\end{pgfscope}%
\end{pgfpicture}%
\makeatother%
\endgroup%

    \caption{Random order, \(n=7.0 \times 10^{9}\)}
    \label{randomlargen}
\end{figure}

\begin{figure}
    %% Creator: Matplotlib, PGF backend
%%
%% To include the figure in your LaTeX document, write
%%   \input{<filename>.pgf}
%%
%% Make sure the required packages are loaded in your preamble
%%   \usepackage{pgf}
%%
%% Also ensure that all the required font packages are loaded; for instance,
%% the lmodern package is sometimes necessary when using math font.
%%   \usepackage{lmodern}
%%
%% Figures using additional raster images can only be included by \input if
%% they are in the same directory as the main LaTeX file. For loading figures
%% from other directories you can use the `import` package
%%   \usepackage{import}
%%
%% and then include the figures with
%%   \import{<path to file>}{<filename>.pgf}
%%
%% Matplotlib used the following preamble
%%   
%%   \usepackage{fontspec}
%%   \makeatletter\@ifpackageloaded{underscore}{}{\usepackage[strings]{underscore}}\makeatother
%%
\begingroup%
\makeatletter%
\begin{pgfpicture}%
\pgfpathrectangle{\pgfpointorigin}{\pgfqpoint{6.400000in}{4.800000in}}%
\pgfusepath{use as bounding box, clip}%
\begin{pgfscope}%
\pgfsetbuttcap%
\pgfsetmiterjoin%
\definecolor{currentfill}{rgb}{1.000000,1.000000,1.000000}%
\pgfsetfillcolor{currentfill}%
\pgfsetlinewidth{0.000000pt}%
\definecolor{currentstroke}{rgb}{1.000000,1.000000,1.000000}%
\pgfsetstrokecolor{currentstroke}%
\pgfsetdash{}{0pt}%
\pgfpathmoveto{\pgfqpoint{0.000000in}{0.000000in}}%
\pgfpathlineto{\pgfqpoint{6.400000in}{0.000000in}}%
\pgfpathlineto{\pgfqpoint{6.400000in}{4.800000in}}%
\pgfpathlineto{\pgfqpoint{0.000000in}{4.800000in}}%
\pgfpathlineto{\pgfqpoint{0.000000in}{0.000000in}}%
\pgfpathclose%
\pgfusepath{fill}%
\end{pgfscope}%
\begin{pgfscope}%
\pgfsetbuttcap%
\pgfsetmiterjoin%
\definecolor{currentfill}{rgb}{1.000000,1.000000,1.000000}%
\pgfsetfillcolor{currentfill}%
\pgfsetlinewidth{0.000000pt}%
\definecolor{currentstroke}{rgb}{0.000000,0.000000,0.000000}%
\pgfsetstrokecolor{currentstroke}%
\pgfsetstrokeopacity{0.000000}%
\pgfsetdash{}{0pt}%
\pgfpathmoveto{\pgfqpoint{0.800000in}{0.528000in}}%
\pgfpathlineto{\pgfqpoint{5.760000in}{0.528000in}}%
\pgfpathlineto{\pgfqpoint{5.760000in}{4.224000in}}%
\pgfpathlineto{\pgfqpoint{0.800000in}{4.224000in}}%
\pgfpathlineto{\pgfqpoint{0.800000in}{0.528000in}}%
\pgfpathclose%
\pgfusepath{fill}%
\end{pgfscope}%
\begin{pgfscope}%
\pgfsetbuttcap%
\pgfsetroundjoin%
\definecolor{currentfill}{rgb}{0.000000,0.000000,0.000000}%
\pgfsetfillcolor{currentfill}%
\pgfsetlinewidth{0.803000pt}%
\definecolor{currentstroke}{rgb}{0.000000,0.000000,0.000000}%
\pgfsetstrokecolor{currentstroke}%
\pgfsetdash{}{0pt}%
\pgfsys@defobject{currentmarker}{\pgfqpoint{0.000000in}{-0.048611in}}{\pgfqpoint{0.000000in}{0.000000in}}{%
\pgfpathmoveto{\pgfqpoint{0.000000in}{0.000000in}}%
\pgfpathlineto{\pgfqpoint{0.000000in}{-0.048611in}}%
\pgfusepath{stroke,fill}%
}%
\begin{pgfscope}%
\pgfsys@transformshift{0.800000in}{0.528000in}%
\pgfsys@useobject{currentmarker}{}%
\end{pgfscope}%
\end{pgfscope}%
\begin{pgfscope}%
\definecolor{textcolor}{rgb}{0.000000,0.000000,0.000000}%
\pgfsetstrokecolor{textcolor}%
\pgfsetfillcolor{textcolor}%
\pgftext[x=0.800000in,y=0.430778in,,top]{\color{textcolor}\rmfamily\fontsize{10.000000}{12.000000}\selectfont \(\displaystyle {10^{0}}\)}%
\end{pgfscope}%
\begin{pgfscope}%
\pgfsetbuttcap%
\pgfsetroundjoin%
\definecolor{currentfill}{rgb}{0.000000,0.000000,0.000000}%
\pgfsetfillcolor{currentfill}%
\pgfsetlinewidth{0.803000pt}%
\definecolor{currentstroke}{rgb}{0.000000,0.000000,0.000000}%
\pgfsetstrokecolor{currentstroke}%
\pgfsetdash{}{0pt}%
\pgfsys@defobject{currentmarker}{\pgfqpoint{0.000000in}{-0.048611in}}{\pgfqpoint{0.000000in}{0.000000in}}{%
\pgfpathmoveto{\pgfqpoint{0.000000in}{0.000000in}}%
\pgfpathlineto{\pgfqpoint{0.000000in}{-0.048611in}}%
\pgfusepath{stroke,fill}%
}%
\begin{pgfscope}%
\pgfsys@transformshift{1.902222in}{0.528000in}%
\pgfsys@useobject{currentmarker}{}%
\end{pgfscope}%
\end{pgfscope}%
\begin{pgfscope}%
\definecolor{textcolor}{rgb}{0.000000,0.000000,0.000000}%
\pgfsetstrokecolor{textcolor}%
\pgfsetfillcolor{textcolor}%
\pgftext[x=1.902222in,y=0.430778in,,top]{\color{textcolor}\rmfamily\fontsize{10.000000}{12.000000}\selectfont \(\displaystyle {10^{2}}\)}%
\end{pgfscope}%
\begin{pgfscope}%
\pgfsetbuttcap%
\pgfsetroundjoin%
\definecolor{currentfill}{rgb}{0.000000,0.000000,0.000000}%
\pgfsetfillcolor{currentfill}%
\pgfsetlinewidth{0.803000pt}%
\definecolor{currentstroke}{rgb}{0.000000,0.000000,0.000000}%
\pgfsetstrokecolor{currentstroke}%
\pgfsetdash{}{0pt}%
\pgfsys@defobject{currentmarker}{\pgfqpoint{0.000000in}{-0.048611in}}{\pgfqpoint{0.000000in}{0.000000in}}{%
\pgfpathmoveto{\pgfqpoint{0.000000in}{0.000000in}}%
\pgfpathlineto{\pgfqpoint{0.000000in}{-0.048611in}}%
\pgfusepath{stroke,fill}%
}%
\begin{pgfscope}%
\pgfsys@transformshift{3.004444in}{0.528000in}%
\pgfsys@useobject{currentmarker}{}%
\end{pgfscope}%
\end{pgfscope}%
\begin{pgfscope}%
\definecolor{textcolor}{rgb}{0.000000,0.000000,0.000000}%
\pgfsetstrokecolor{textcolor}%
\pgfsetfillcolor{textcolor}%
\pgftext[x=3.004444in,y=0.430778in,,top]{\color{textcolor}\rmfamily\fontsize{10.000000}{12.000000}\selectfont \(\displaystyle {10^{4}}\)}%
\end{pgfscope}%
\begin{pgfscope}%
\pgfsetbuttcap%
\pgfsetroundjoin%
\definecolor{currentfill}{rgb}{0.000000,0.000000,0.000000}%
\pgfsetfillcolor{currentfill}%
\pgfsetlinewidth{0.803000pt}%
\definecolor{currentstroke}{rgb}{0.000000,0.000000,0.000000}%
\pgfsetstrokecolor{currentstroke}%
\pgfsetdash{}{0pt}%
\pgfsys@defobject{currentmarker}{\pgfqpoint{0.000000in}{-0.048611in}}{\pgfqpoint{0.000000in}{0.000000in}}{%
\pgfpathmoveto{\pgfqpoint{0.000000in}{0.000000in}}%
\pgfpathlineto{\pgfqpoint{0.000000in}{-0.048611in}}%
\pgfusepath{stroke,fill}%
}%
\begin{pgfscope}%
\pgfsys@transformshift{4.106667in}{0.528000in}%
\pgfsys@useobject{currentmarker}{}%
\end{pgfscope}%
\end{pgfscope}%
\begin{pgfscope}%
\definecolor{textcolor}{rgb}{0.000000,0.000000,0.000000}%
\pgfsetstrokecolor{textcolor}%
\pgfsetfillcolor{textcolor}%
\pgftext[x=4.106667in,y=0.430778in,,top]{\color{textcolor}\rmfamily\fontsize{10.000000}{12.000000}\selectfont \(\displaystyle {10^{6}}\)}%
\end{pgfscope}%
\begin{pgfscope}%
\pgfsetbuttcap%
\pgfsetroundjoin%
\definecolor{currentfill}{rgb}{0.000000,0.000000,0.000000}%
\pgfsetfillcolor{currentfill}%
\pgfsetlinewidth{0.803000pt}%
\definecolor{currentstroke}{rgb}{0.000000,0.000000,0.000000}%
\pgfsetstrokecolor{currentstroke}%
\pgfsetdash{}{0pt}%
\pgfsys@defobject{currentmarker}{\pgfqpoint{0.000000in}{-0.048611in}}{\pgfqpoint{0.000000in}{0.000000in}}{%
\pgfpathmoveto{\pgfqpoint{0.000000in}{0.000000in}}%
\pgfpathlineto{\pgfqpoint{0.000000in}{-0.048611in}}%
\pgfusepath{stroke,fill}%
}%
\begin{pgfscope}%
\pgfsys@transformshift{5.208889in}{0.528000in}%
\pgfsys@useobject{currentmarker}{}%
\end{pgfscope}%
\end{pgfscope}%
\begin{pgfscope}%
\definecolor{textcolor}{rgb}{0.000000,0.000000,0.000000}%
\pgfsetstrokecolor{textcolor}%
\pgfsetfillcolor{textcolor}%
\pgftext[x=5.208889in,y=0.430778in,,top]{\color{textcolor}\rmfamily\fontsize{10.000000}{12.000000}\selectfont \(\displaystyle {10^{8}}\)}%
\end{pgfscope}%
\begin{pgfscope}%
\definecolor{textcolor}{rgb}{0.000000,0.000000,0.000000}%
\pgfsetstrokecolor{textcolor}%
\pgfsetfillcolor{textcolor}%
\pgftext[x=3.280000in,y=0.251889in,,top]{\color{textcolor}\rmfamily\fontsize{10.000000}{12.000000}\selectfont k}%
\end{pgfscope}%
\begin{pgfscope}%
\pgfsetbuttcap%
\pgfsetroundjoin%
\definecolor{currentfill}{rgb}{0.000000,0.000000,0.000000}%
\pgfsetfillcolor{currentfill}%
\pgfsetlinewidth{0.803000pt}%
\definecolor{currentstroke}{rgb}{0.000000,0.000000,0.000000}%
\pgfsetstrokecolor{currentstroke}%
\pgfsetdash{}{0pt}%
\pgfsys@defobject{currentmarker}{\pgfqpoint{-0.048611in}{0.000000in}}{\pgfqpoint{-0.000000in}{0.000000in}}{%
\pgfpathmoveto{\pgfqpoint{-0.000000in}{0.000000in}}%
\pgfpathlineto{\pgfqpoint{-0.048611in}{0.000000in}}%
\pgfusepath{stroke,fill}%
}%
\begin{pgfscope}%
\pgfsys@transformshift{0.800000in}{0.528000in}%
\pgfsys@useobject{currentmarker}{}%
\end{pgfscope}%
\end{pgfscope}%
\begin{pgfscope}%
\definecolor{textcolor}{rgb}{0.000000,0.000000,0.000000}%
\pgfsetstrokecolor{textcolor}%
\pgfsetfillcolor{textcolor}%
\pgftext[x=0.633333in, y=0.479806in, left, base]{\color{textcolor}\rmfamily\fontsize{10.000000}{12.000000}\selectfont \(\displaystyle {0}\)}%
\end{pgfscope}%
\begin{pgfscope}%
\pgfsetbuttcap%
\pgfsetroundjoin%
\definecolor{currentfill}{rgb}{0.000000,0.000000,0.000000}%
\pgfsetfillcolor{currentfill}%
\pgfsetlinewidth{0.803000pt}%
\definecolor{currentstroke}{rgb}{0.000000,0.000000,0.000000}%
\pgfsetstrokecolor{currentstroke}%
\pgfsetdash{}{0pt}%
\pgfsys@defobject{currentmarker}{\pgfqpoint{-0.048611in}{0.000000in}}{\pgfqpoint{-0.000000in}{0.000000in}}{%
\pgfpathmoveto{\pgfqpoint{-0.000000in}{0.000000in}}%
\pgfpathlineto{\pgfqpoint{-0.048611in}{0.000000in}}%
\pgfusepath{stroke,fill}%
}%
\begin{pgfscope}%
\pgfsys@transformshift{0.800000in}{0.990000in}%
\pgfsys@useobject{currentmarker}{}%
\end{pgfscope}%
\end{pgfscope}%
\begin{pgfscope}%
\definecolor{textcolor}{rgb}{0.000000,0.000000,0.000000}%
\pgfsetstrokecolor{textcolor}%
\pgfsetfillcolor{textcolor}%
\pgftext[x=0.563888in, y=0.941806in, left, base]{\color{textcolor}\rmfamily\fontsize{10.000000}{12.000000}\selectfont \(\displaystyle {25}\)}%
\end{pgfscope}%
\begin{pgfscope}%
\pgfsetbuttcap%
\pgfsetroundjoin%
\definecolor{currentfill}{rgb}{0.000000,0.000000,0.000000}%
\pgfsetfillcolor{currentfill}%
\pgfsetlinewidth{0.803000pt}%
\definecolor{currentstroke}{rgb}{0.000000,0.000000,0.000000}%
\pgfsetstrokecolor{currentstroke}%
\pgfsetdash{}{0pt}%
\pgfsys@defobject{currentmarker}{\pgfqpoint{-0.048611in}{0.000000in}}{\pgfqpoint{-0.000000in}{0.000000in}}{%
\pgfpathmoveto{\pgfqpoint{-0.000000in}{0.000000in}}%
\pgfpathlineto{\pgfqpoint{-0.048611in}{0.000000in}}%
\pgfusepath{stroke,fill}%
}%
\begin{pgfscope}%
\pgfsys@transformshift{0.800000in}{1.452000in}%
\pgfsys@useobject{currentmarker}{}%
\end{pgfscope}%
\end{pgfscope}%
\begin{pgfscope}%
\definecolor{textcolor}{rgb}{0.000000,0.000000,0.000000}%
\pgfsetstrokecolor{textcolor}%
\pgfsetfillcolor{textcolor}%
\pgftext[x=0.563888in, y=1.403806in, left, base]{\color{textcolor}\rmfamily\fontsize{10.000000}{12.000000}\selectfont \(\displaystyle {50}\)}%
\end{pgfscope}%
\begin{pgfscope}%
\pgfsetbuttcap%
\pgfsetroundjoin%
\definecolor{currentfill}{rgb}{0.000000,0.000000,0.000000}%
\pgfsetfillcolor{currentfill}%
\pgfsetlinewidth{0.803000pt}%
\definecolor{currentstroke}{rgb}{0.000000,0.000000,0.000000}%
\pgfsetstrokecolor{currentstroke}%
\pgfsetdash{}{0pt}%
\pgfsys@defobject{currentmarker}{\pgfqpoint{-0.048611in}{0.000000in}}{\pgfqpoint{-0.000000in}{0.000000in}}{%
\pgfpathmoveto{\pgfqpoint{-0.000000in}{0.000000in}}%
\pgfpathlineto{\pgfqpoint{-0.048611in}{0.000000in}}%
\pgfusepath{stroke,fill}%
}%
\begin{pgfscope}%
\pgfsys@transformshift{0.800000in}{1.914000in}%
\pgfsys@useobject{currentmarker}{}%
\end{pgfscope}%
\end{pgfscope}%
\begin{pgfscope}%
\definecolor{textcolor}{rgb}{0.000000,0.000000,0.000000}%
\pgfsetstrokecolor{textcolor}%
\pgfsetfillcolor{textcolor}%
\pgftext[x=0.563888in, y=1.865806in, left, base]{\color{textcolor}\rmfamily\fontsize{10.000000}{12.000000}\selectfont \(\displaystyle {75}\)}%
\end{pgfscope}%
\begin{pgfscope}%
\pgfsetbuttcap%
\pgfsetroundjoin%
\definecolor{currentfill}{rgb}{0.000000,0.000000,0.000000}%
\pgfsetfillcolor{currentfill}%
\pgfsetlinewidth{0.803000pt}%
\definecolor{currentstroke}{rgb}{0.000000,0.000000,0.000000}%
\pgfsetstrokecolor{currentstroke}%
\pgfsetdash{}{0pt}%
\pgfsys@defobject{currentmarker}{\pgfqpoint{-0.048611in}{0.000000in}}{\pgfqpoint{-0.000000in}{0.000000in}}{%
\pgfpathmoveto{\pgfqpoint{-0.000000in}{0.000000in}}%
\pgfpathlineto{\pgfqpoint{-0.048611in}{0.000000in}}%
\pgfusepath{stroke,fill}%
}%
\begin{pgfscope}%
\pgfsys@transformshift{0.800000in}{2.376000in}%
\pgfsys@useobject{currentmarker}{}%
\end{pgfscope}%
\end{pgfscope}%
\begin{pgfscope}%
\definecolor{textcolor}{rgb}{0.000000,0.000000,0.000000}%
\pgfsetstrokecolor{textcolor}%
\pgfsetfillcolor{textcolor}%
\pgftext[x=0.494444in, y=2.327806in, left, base]{\color{textcolor}\rmfamily\fontsize{10.000000}{12.000000}\selectfont \(\displaystyle {100}\)}%
\end{pgfscope}%
\begin{pgfscope}%
\pgfsetbuttcap%
\pgfsetroundjoin%
\definecolor{currentfill}{rgb}{0.000000,0.000000,0.000000}%
\pgfsetfillcolor{currentfill}%
\pgfsetlinewidth{0.803000pt}%
\definecolor{currentstroke}{rgb}{0.000000,0.000000,0.000000}%
\pgfsetstrokecolor{currentstroke}%
\pgfsetdash{}{0pt}%
\pgfsys@defobject{currentmarker}{\pgfqpoint{-0.048611in}{0.000000in}}{\pgfqpoint{-0.000000in}{0.000000in}}{%
\pgfpathmoveto{\pgfqpoint{-0.000000in}{0.000000in}}%
\pgfpathlineto{\pgfqpoint{-0.048611in}{0.000000in}}%
\pgfusepath{stroke,fill}%
}%
\begin{pgfscope}%
\pgfsys@transformshift{0.800000in}{2.838000in}%
\pgfsys@useobject{currentmarker}{}%
\end{pgfscope}%
\end{pgfscope}%
\begin{pgfscope}%
\definecolor{textcolor}{rgb}{0.000000,0.000000,0.000000}%
\pgfsetstrokecolor{textcolor}%
\pgfsetfillcolor{textcolor}%
\pgftext[x=0.494444in, y=2.789806in, left, base]{\color{textcolor}\rmfamily\fontsize{10.000000}{12.000000}\selectfont \(\displaystyle {125}\)}%
\end{pgfscope}%
\begin{pgfscope}%
\pgfsetbuttcap%
\pgfsetroundjoin%
\definecolor{currentfill}{rgb}{0.000000,0.000000,0.000000}%
\pgfsetfillcolor{currentfill}%
\pgfsetlinewidth{0.803000pt}%
\definecolor{currentstroke}{rgb}{0.000000,0.000000,0.000000}%
\pgfsetstrokecolor{currentstroke}%
\pgfsetdash{}{0pt}%
\pgfsys@defobject{currentmarker}{\pgfqpoint{-0.048611in}{0.000000in}}{\pgfqpoint{-0.000000in}{0.000000in}}{%
\pgfpathmoveto{\pgfqpoint{-0.000000in}{0.000000in}}%
\pgfpathlineto{\pgfqpoint{-0.048611in}{0.000000in}}%
\pgfusepath{stroke,fill}%
}%
\begin{pgfscope}%
\pgfsys@transformshift{0.800000in}{3.300000in}%
\pgfsys@useobject{currentmarker}{}%
\end{pgfscope}%
\end{pgfscope}%
\begin{pgfscope}%
\definecolor{textcolor}{rgb}{0.000000,0.000000,0.000000}%
\pgfsetstrokecolor{textcolor}%
\pgfsetfillcolor{textcolor}%
\pgftext[x=0.494444in, y=3.251806in, left, base]{\color{textcolor}\rmfamily\fontsize{10.000000}{12.000000}\selectfont \(\displaystyle {150}\)}%
\end{pgfscope}%
\begin{pgfscope}%
\pgfsetbuttcap%
\pgfsetroundjoin%
\definecolor{currentfill}{rgb}{0.000000,0.000000,0.000000}%
\pgfsetfillcolor{currentfill}%
\pgfsetlinewidth{0.803000pt}%
\definecolor{currentstroke}{rgb}{0.000000,0.000000,0.000000}%
\pgfsetstrokecolor{currentstroke}%
\pgfsetdash{}{0pt}%
\pgfsys@defobject{currentmarker}{\pgfqpoint{-0.048611in}{0.000000in}}{\pgfqpoint{-0.000000in}{0.000000in}}{%
\pgfpathmoveto{\pgfqpoint{-0.000000in}{0.000000in}}%
\pgfpathlineto{\pgfqpoint{-0.048611in}{0.000000in}}%
\pgfusepath{stroke,fill}%
}%
\begin{pgfscope}%
\pgfsys@transformshift{0.800000in}{3.762000in}%
\pgfsys@useobject{currentmarker}{}%
\end{pgfscope}%
\end{pgfscope}%
\begin{pgfscope}%
\definecolor{textcolor}{rgb}{0.000000,0.000000,0.000000}%
\pgfsetstrokecolor{textcolor}%
\pgfsetfillcolor{textcolor}%
\pgftext[x=0.494444in, y=3.713806in, left, base]{\color{textcolor}\rmfamily\fontsize{10.000000}{12.000000}\selectfont \(\displaystyle {175}\)}%
\end{pgfscope}%
\begin{pgfscope}%
\pgfsetbuttcap%
\pgfsetroundjoin%
\definecolor{currentfill}{rgb}{0.000000,0.000000,0.000000}%
\pgfsetfillcolor{currentfill}%
\pgfsetlinewidth{0.803000pt}%
\definecolor{currentstroke}{rgb}{0.000000,0.000000,0.000000}%
\pgfsetstrokecolor{currentstroke}%
\pgfsetdash{}{0pt}%
\pgfsys@defobject{currentmarker}{\pgfqpoint{-0.048611in}{0.000000in}}{\pgfqpoint{-0.000000in}{0.000000in}}{%
\pgfpathmoveto{\pgfqpoint{-0.000000in}{0.000000in}}%
\pgfpathlineto{\pgfqpoint{-0.048611in}{0.000000in}}%
\pgfusepath{stroke,fill}%
}%
\begin{pgfscope}%
\pgfsys@transformshift{0.800000in}{4.224000in}%
\pgfsys@useobject{currentmarker}{}%
\end{pgfscope}%
\end{pgfscope}%
\begin{pgfscope}%
\definecolor{textcolor}{rgb}{0.000000,0.000000,0.000000}%
\pgfsetstrokecolor{textcolor}%
\pgfsetfillcolor{textcolor}%
\pgftext[x=0.494444in, y=4.175806in, left, base]{\color{textcolor}\rmfamily\fontsize{10.000000}{12.000000}\selectfont \(\displaystyle {200}\)}%
\end{pgfscope}%
\begin{pgfscope}%
\definecolor{textcolor}{rgb}{0.000000,0.000000,0.000000}%
\pgfsetstrokecolor{textcolor}%
\pgfsetfillcolor{textcolor}%
\pgftext[x=0.438888in,y=2.376000in,,bottom,rotate=90.000000]{\color{textcolor}\rmfamily\fontsize{10.000000}{12.000000}\selectfont time/k (ns)}%
\end{pgfscope}%
\begin{pgfscope}%
\pgfpathrectangle{\pgfqpoint{0.800000in}{0.528000in}}{\pgfqpoint{4.960000in}{3.696000in}}%
\pgfusepath{clip}%
\pgfsetrectcap%
\pgfsetroundjoin%
\pgfsetlinewidth{1.505625pt}%
\definecolor{currentstroke}{rgb}{0.121569,0.466667,0.705882}%
\pgfsetstrokecolor{currentstroke}%
\pgfsetdash{}{0pt}%
\pgfpathmoveto{\pgfqpoint{0.800000in}{0.780267in}}%
\pgfpathlineto{\pgfqpoint{0.965901in}{0.778902in}}%
\pgfpathlineto{\pgfqpoint{1.062947in}{0.778050in}}%
\pgfpathlineto{\pgfqpoint{1.185210in}{0.789730in}}%
\pgfpathlineto{\pgfqpoint{1.297703in}{0.821665in}}%
\pgfpathlineto{\pgfqpoint{1.413907in}{0.836819in}}%
\pgfpathlineto{\pgfqpoint{1.528690in}{0.917023in}}%
\pgfpathlineto{\pgfqpoint{1.644015in}{0.956316in}}%
\pgfpathlineto{\pgfqpoint{1.759133in}{0.996136in}}%
\pgfpathlineto{\pgfqpoint{1.874330in}{1.033502in}}%
\pgfpathlineto{\pgfqpoint{1.989498in}{1.073423in}}%
\pgfpathlineto{\pgfqpoint{2.104676in}{1.117492in}}%
\pgfpathlineto{\pgfqpoint{2.219850in}{1.151393in}}%
\pgfpathlineto{\pgfqpoint{2.335026in}{1.189728in}}%
\pgfpathlineto{\pgfqpoint{2.450201in}{1.228283in}}%
\pgfpathlineto{\pgfqpoint{2.565377in}{1.264329in}}%
\pgfpathlineto{\pgfqpoint{2.680552in}{1.300925in}}%
\pgfpathlineto{\pgfqpoint{2.795728in}{1.334854in}}%
\pgfpathlineto{\pgfqpoint{2.910903in}{1.378306in}}%
\pgfpathlineto{\pgfqpoint{3.026079in}{1.419954in}}%
\pgfpathlineto{\pgfqpoint{3.141254in}{1.457107in}}%
\pgfpathlineto{\pgfqpoint{3.256429in}{1.490763in}}%
\pgfpathlineto{\pgfqpoint{3.371605in}{1.531493in}}%
\pgfpathlineto{\pgfqpoint{3.486780in}{1.562739in}}%
\pgfpathlineto{\pgfqpoint{3.601956in}{1.598953in}}%
\pgfpathlineto{\pgfqpoint{3.717131in}{1.630725in}}%
\pgfpathlineto{\pgfqpoint{3.832306in}{1.672126in}}%
\pgfpathlineto{\pgfqpoint{3.947482in}{1.709573in}}%
\pgfpathlineto{\pgfqpoint{4.062657in}{1.753648in}}%
\pgfpathlineto{\pgfqpoint{4.177833in}{1.790126in}}%
\pgfpathlineto{\pgfqpoint{4.293008in}{1.811709in}}%
\pgfpathlineto{\pgfqpoint{4.408183in}{1.846487in}}%
\pgfpathlineto{\pgfqpoint{4.523359in}{1.906861in}}%
\pgfpathlineto{\pgfqpoint{4.638534in}{1.945628in}}%
\pgfpathlineto{\pgfqpoint{4.753710in}{2.009180in}}%
\pgfpathlineto{\pgfqpoint{4.868885in}{2.049046in}}%
\pgfpathlineto{\pgfqpoint{4.984061in}{2.130175in}}%
\pgfpathlineto{\pgfqpoint{5.099236in}{2.203892in}}%
\pgfpathlineto{\pgfqpoint{5.214411in}{2.295089in}}%
\pgfpathlineto{\pgfqpoint{5.329587in}{2.391706in}}%
\pgfusepath{stroke}%
\end{pgfscope}%
\begin{pgfscope}%
\pgfpathrectangle{\pgfqpoint{0.800000in}{0.528000in}}{\pgfqpoint{4.960000in}{3.696000in}}%
\pgfusepath{clip}%
\pgfsetrectcap%
\pgfsetroundjoin%
\pgfsetlinewidth{1.505625pt}%
\definecolor{currentstroke}{rgb}{1.000000,0.498039,0.054902}%
\pgfsetstrokecolor{currentstroke}%
\pgfsetdash{}{0pt}%
\pgfpathmoveto{\pgfqpoint{0.800000in}{1.723591in}}%
\pgfpathlineto{\pgfqpoint{0.965901in}{1.606894in}}%
\pgfpathlineto{\pgfqpoint{1.062947in}{1.572078in}}%
\pgfpathlineto{\pgfqpoint{1.185210in}{1.579539in}}%
\pgfpathlineto{\pgfqpoint{1.297703in}{1.695730in}}%
\pgfpathlineto{\pgfqpoint{1.413907in}{1.638833in}}%
\pgfpathlineto{\pgfqpoint{1.528690in}{1.743597in}}%
\pgfpathlineto{\pgfqpoint{1.644015in}{1.779206in}}%
\pgfpathlineto{\pgfqpoint{1.759133in}{1.811653in}}%
\pgfpathlineto{\pgfqpoint{1.874330in}{2.350842in}}%
\pgfpathlineto{\pgfqpoint{1.989498in}{2.431605in}}%
\pgfpathlineto{\pgfqpoint{2.104676in}{2.477203in}}%
\pgfpathlineto{\pgfqpoint{2.219850in}{2.511277in}}%
\pgfpathlineto{\pgfqpoint{2.335026in}{2.529999in}}%
\pgfpathlineto{\pgfqpoint{2.450201in}{2.551432in}}%
\pgfpathlineto{\pgfqpoint{2.565377in}{2.575470in}}%
\pgfpathlineto{\pgfqpoint{2.680552in}{2.582346in}}%
\pgfpathlineto{\pgfqpoint{2.795728in}{2.644087in}}%
\pgfpathlineto{\pgfqpoint{2.910903in}{2.698125in}}%
\pgfpathlineto{\pgfqpoint{3.026079in}{2.805631in}}%
\pgfpathlineto{\pgfqpoint{3.141254in}{2.892583in}}%
\pgfpathlineto{\pgfqpoint{3.256429in}{3.025873in}}%
\pgfpathlineto{\pgfqpoint{3.371605in}{3.060055in}}%
\pgfpathlineto{\pgfqpoint{3.486780in}{3.122551in}}%
\pgfpathlineto{\pgfqpoint{3.601956in}{3.224908in}}%
\pgfpathlineto{\pgfqpoint{3.717131in}{3.271063in}}%
\pgfpathlineto{\pgfqpoint{3.832306in}{3.357202in}}%
\pgfpathlineto{\pgfqpoint{3.947482in}{3.571260in}}%
\pgfpathlineto{\pgfqpoint{4.044029in}{4.234000in}}%
\pgfusepath{stroke}%
\end{pgfscope}%
\begin{pgfscope}%
\pgfpathrectangle{\pgfqpoint{0.800000in}{0.528000in}}{\pgfqpoint{4.960000in}{3.696000in}}%
\pgfusepath{clip}%
\pgfsetrectcap%
\pgfsetroundjoin%
\pgfsetlinewidth{1.505625pt}%
\definecolor{currentstroke}{rgb}{0.172549,0.627451,0.172549}%
\pgfsetstrokecolor{currentstroke}%
\pgfsetdash{}{0pt}%
\pgfpathmoveto{\pgfqpoint{0.800000in}{1.770115in}}%
\pgfpathlineto{\pgfqpoint{0.965901in}{1.778153in}}%
\pgfpathlineto{\pgfqpoint{1.062947in}{1.828624in}}%
\pgfpathlineto{\pgfqpoint{1.185210in}{1.903425in}}%
\pgfpathlineto{\pgfqpoint{1.297703in}{2.006002in}}%
\pgfpathlineto{\pgfqpoint{1.413907in}{2.061384in}}%
\pgfpathlineto{\pgfqpoint{1.528690in}{2.158663in}}%
\pgfpathlineto{\pgfqpoint{1.644015in}{2.196072in}}%
\pgfpathlineto{\pgfqpoint{1.759133in}{2.235619in}}%
\pgfpathlineto{\pgfqpoint{1.874330in}{2.802415in}}%
\pgfpathlineto{\pgfqpoint{1.989498in}{2.891592in}}%
\pgfpathlineto{\pgfqpoint{2.104676in}{2.921318in}}%
\pgfpathlineto{\pgfqpoint{2.219850in}{2.968939in}}%
\pgfpathlineto{\pgfqpoint{2.335026in}{3.006608in}}%
\pgfpathlineto{\pgfqpoint{2.450201in}{3.008019in}}%
\pgfpathlineto{\pgfqpoint{2.565377in}{2.998069in}}%
\pgfpathlineto{\pgfqpoint{2.680552in}{2.995637in}}%
\pgfpathlineto{\pgfqpoint{2.795728in}{3.033352in}}%
\pgfpathlineto{\pgfqpoint{2.910903in}{3.086832in}}%
\pgfpathlineto{\pgfqpoint{3.026079in}{3.169301in}}%
\pgfpathlineto{\pgfqpoint{3.141254in}{3.292487in}}%
\pgfpathlineto{\pgfqpoint{3.256429in}{3.360205in}}%
\pgfpathlineto{\pgfqpoint{3.371605in}{3.406731in}}%
\pgfpathlineto{\pgfqpoint{3.486780in}{3.469356in}}%
\pgfpathlineto{\pgfqpoint{3.601956in}{3.544348in}}%
\pgfpathlineto{\pgfqpoint{3.717131in}{3.657146in}}%
\pgfpathlineto{\pgfqpoint{3.832306in}{3.720828in}}%
\pgfpathlineto{\pgfqpoint{3.947482in}{3.907439in}}%
\pgfpathlineto{\pgfqpoint{3.997746in}{4.234000in}}%
\pgfusepath{stroke}%
\end{pgfscope}%
\begin{pgfscope}%
\pgfpathrectangle{\pgfqpoint{0.800000in}{0.528000in}}{\pgfqpoint{4.960000in}{3.696000in}}%
\pgfusepath{clip}%
\pgfsetrectcap%
\pgfsetroundjoin%
\pgfsetlinewidth{1.505625pt}%
\definecolor{currentstroke}{rgb}{0.839216,0.152941,0.156863}%
\pgfsetstrokecolor{currentstroke}%
\pgfsetdash{}{0pt}%
\pgfpathmoveto{\pgfqpoint{4.832645in}{4.234000in}}%
\pgfpathlineto{\pgfqpoint{4.868885in}{3.662905in}}%
\pgfpathlineto{\pgfqpoint{4.984061in}{2.560425in}}%
\pgfpathlineto{\pgfqpoint{5.099236in}{1.869360in}}%
\pgfpathlineto{\pgfqpoint{5.214411in}{1.437362in}}%
\pgfpathlineto{\pgfqpoint{5.329587in}{1.157282in}}%
\pgfpathlineto{\pgfqpoint{5.444762in}{0.983735in}}%
\pgfpathlineto{\pgfqpoint{5.559938in}{0.811748in}}%
\pgfpathlineto{\pgfqpoint{5.675113in}{0.632409in}}%
\pgfusepath{stroke}%
\end{pgfscope}%
\begin{pgfscope}%
\pgfpathrectangle{\pgfqpoint{0.800000in}{0.528000in}}{\pgfqpoint{4.960000in}{3.696000in}}%
\pgfusepath{clip}%
\pgfsetrectcap%
\pgfsetroundjoin%
\pgfsetlinewidth{1.505625pt}%
\definecolor{currentstroke}{rgb}{0.580392,0.403922,0.741176}%
\pgfsetstrokecolor{currentstroke}%
\pgfsetdash{}{0pt}%
\pgfpathmoveto{\pgfqpoint{0.800000in}{0.752663in}}%
\pgfpathlineto{\pgfqpoint{0.965901in}{0.780070in}}%
\pgfpathlineto{\pgfqpoint{1.062947in}{0.800288in}}%
\pgfpathlineto{\pgfqpoint{1.185210in}{0.819115in}}%
\pgfpathlineto{\pgfqpoint{1.297703in}{0.855318in}}%
\pgfpathlineto{\pgfqpoint{1.413907in}{0.894336in}}%
\pgfpathlineto{\pgfqpoint{1.528690in}{1.029931in}}%
\pgfpathlineto{\pgfqpoint{1.644015in}{1.146282in}}%
\pgfpathlineto{\pgfqpoint{1.759133in}{1.308831in}}%
\pgfpathlineto{\pgfqpoint{1.874330in}{1.489758in}}%
\pgfpathlineto{\pgfqpoint{1.989498in}{1.706615in}}%
\pgfpathlineto{\pgfqpoint{2.104676in}{2.017656in}}%
\pgfpathlineto{\pgfqpoint{2.219850in}{2.489358in}}%
\pgfpathlineto{\pgfqpoint{2.335026in}{3.183131in}}%
\pgfpathlineto{\pgfqpoint{2.443138in}{4.234000in}}%
\pgfusepath{stroke}%
\end{pgfscope}%
\begin{pgfscope}%
\pgfpathrectangle{\pgfqpoint{0.800000in}{0.528000in}}{\pgfqpoint{4.960000in}{3.696000in}}%
\pgfusepath{clip}%
\pgfsetrectcap%
\pgfsetroundjoin%
\pgfsetlinewidth{1.505625pt}%
\definecolor{currentstroke}{rgb}{0.549020,0.337255,0.294118}%
\pgfsetstrokecolor{currentstroke}%
\pgfsetdash{}{0pt}%
\pgfpathmoveto{\pgfqpoint{0.800000in}{0.755942in}}%
\pgfpathlineto{\pgfqpoint{0.965901in}{0.780221in}}%
\pgfpathlineto{\pgfqpoint{1.062947in}{0.793823in}}%
\pgfpathlineto{\pgfqpoint{1.185210in}{0.832884in}}%
\pgfpathlineto{\pgfqpoint{1.297703in}{0.871829in}}%
\pgfpathlineto{\pgfqpoint{1.413907in}{0.919336in}}%
\pgfpathlineto{\pgfqpoint{1.528690in}{1.070410in}}%
\pgfpathlineto{\pgfqpoint{1.644015in}{1.193042in}}%
\pgfpathlineto{\pgfqpoint{1.759133in}{1.374175in}}%
\pgfpathlineto{\pgfqpoint{1.874330in}{1.563570in}}%
\pgfpathlineto{\pgfqpoint{1.989498in}{1.793493in}}%
\pgfpathlineto{\pgfqpoint{2.104676in}{2.105017in}}%
\pgfpathlineto{\pgfqpoint{2.219850in}{2.568249in}}%
\pgfpathlineto{\pgfqpoint{2.335026in}{3.275244in}}%
\pgfpathlineto{\pgfqpoint{2.433030in}{4.234000in}}%
\pgfusepath{stroke}%
\end{pgfscope}%
\begin{pgfscope}%
\pgfpathrectangle{\pgfqpoint{0.800000in}{0.528000in}}{\pgfqpoint{4.960000in}{3.696000in}}%
\pgfusepath{clip}%
\pgfsetrectcap%
\pgfsetroundjoin%
\pgfsetlinewidth{1.505625pt}%
\definecolor{currentstroke}{rgb}{0.890196,0.466667,0.760784}%
\pgfsetstrokecolor{currentstroke}%
\pgfsetdash{}{0pt}%
\pgfpathmoveto{\pgfqpoint{0.800000in}{1.700801in}}%
\pgfpathlineto{\pgfqpoint{0.965901in}{1.576902in}}%
\pgfpathlineto{\pgfqpoint{1.062947in}{1.540465in}}%
\pgfpathlineto{\pgfqpoint{1.185210in}{1.556581in}}%
\pgfpathlineto{\pgfqpoint{1.297703in}{1.627265in}}%
\pgfpathlineto{\pgfqpoint{1.413907in}{1.620247in}}%
\pgfpathlineto{\pgfqpoint{1.528690in}{1.734148in}}%
\pgfpathlineto{\pgfqpoint{1.644015in}{1.759308in}}%
\pgfpathlineto{\pgfqpoint{1.759133in}{1.794153in}}%
\pgfpathlineto{\pgfqpoint{1.874330in}{2.337336in}}%
\pgfpathlineto{\pgfqpoint{1.989498in}{2.419107in}}%
\pgfpathlineto{\pgfqpoint{2.104676in}{2.463336in}}%
\pgfpathlineto{\pgfqpoint{2.219850in}{2.519388in}}%
\pgfpathlineto{\pgfqpoint{2.335026in}{2.519711in}}%
\pgfpathlineto{\pgfqpoint{2.450201in}{2.536947in}}%
\pgfpathlineto{\pgfqpoint{2.565377in}{2.561777in}}%
\pgfpathlineto{\pgfqpoint{2.680552in}{2.552098in}}%
\pgfpathlineto{\pgfqpoint{2.795728in}{2.597748in}}%
\pgfpathlineto{\pgfqpoint{2.910903in}{2.676586in}}%
\pgfpathlineto{\pgfqpoint{3.026079in}{2.775077in}}%
\pgfpathlineto{\pgfqpoint{3.141254in}{2.879160in}}%
\pgfpathlineto{\pgfqpoint{3.256429in}{2.997515in}}%
\pgfpathlineto{\pgfqpoint{3.371605in}{3.043686in}}%
\pgfpathlineto{\pgfqpoint{3.486780in}{3.130919in}}%
\pgfpathlineto{\pgfqpoint{3.601956in}{3.185403in}}%
\pgfpathlineto{\pgfqpoint{3.717131in}{3.263231in}}%
\pgfpathlineto{\pgfqpoint{3.832306in}{3.414959in}}%
\pgfpathlineto{\pgfqpoint{3.947482in}{3.568208in}}%
\pgfpathlineto{\pgfqpoint{4.062657in}{4.224551in}}%
\pgfpathlineto{\pgfqpoint{4.063991in}{4.234000in}}%
\pgfusepath{stroke}%
\end{pgfscope}%
\begin{pgfscope}%
\pgfpathrectangle{\pgfqpoint{0.800000in}{0.528000in}}{\pgfqpoint{4.960000in}{3.696000in}}%
\pgfusepath{clip}%
\pgfsetrectcap%
\pgfsetroundjoin%
\pgfsetlinewidth{1.505625pt}%
\definecolor{currentstroke}{rgb}{0.498039,0.498039,0.498039}%
\pgfsetstrokecolor{currentstroke}%
\pgfsetdash{}{0pt}%
\pgfusepath{stroke}%
\end{pgfscope}%
\begin{pgfscope}%
\pgfpathrectangle{\pgfqpoint{0.800000in}{0.528000in}}{\pgfqpoint{4.960000in}{3.696000in}}%
\pgfusepath{clip}%
\pgfsetrectcap%
\pgfsetroundjoin%
\pgfsetlinewidth{1.505625pt}%
\definecolor{currentstroke}{rgb}{0.737255,0.741176,0.133333}%
\pgfsetstrokecolor{currentstroke}%
\pgfsetdash{}{0pt}%
\pgfpathmoveto{\pgfqpoint{4.703502in}{4.234000in}}%
\pgfpathlineto{\pgfqpoint{4.753710in}{3.868968in}}%
\pgfpathlineto{\pgfqpoint{4.868885in}{3.379026in}}%
\pgfpathlineto{\pgfqpoint{4.984061in}{3.099777in}}%
\pgfpathlineto{\pgfqpoint{5.099236in}{2.933595in}}%
\pgfpathlineto{\pgfqpoint{5.214411in}{2.844715in}}%
\pgfusepath{stroke}%
\end{pgfscope}%
\begin{pgfscope}%
\pgfsetrectcap%
\pgfsetmiterjoin%
\pgfsetlinewidth{0.803000pt}%
\definecolor{currentstroke}{rgb}{0.000000,0.000000,0.000000}%
\pgfsetstrokecolor{currentstroke}%
\pgfsetdash{}{0pt}%
\pgfpathmoveto{\pgfqpoint{0.800000in}{0.528000in}}%
\pgfpathlineto{\pgfqpoint{0.800000in}{4.224000in}}%
\pgfusepath{stroke}%
\end{pgfscope}%
\begin{pgfscope}%
\pgfsetrectcap%
\pgfsetmiterjoin%
\pgfsetlinewidth{0.803000pt}%
\definecolor{currentstroke}{rgb}{0.000000,0.000000,0.000000}%
\pgfsetstrokecolor{currentstroke}%
\pgfsetdash{}{0pt}%
\pgfpathmoveto{\pgfqpoint{5.760000in}{0.528000in}}%
\pgfpathlineto{\pgfqpoint{5.760000in}{4.224000in}}%
\pgfusepath{stroke}%
\end{pgfscope}%
\begin{pgfscope}%
\pgfsetrectcap%
\pgfsetmiterjoin%
\pgfsetlinewidth{0.803000pt}%
\definecolor{currentstroke}{rgb}{0.000000,0.000000,0.000000}%
\pgfsetstrokecolor{currentstroke}%
\pgfsetdash{}{0pt}%
\pgfpathmoveto{\pgfqpoint{0.800000in}{0.528000in}}%
\pgfpathlineto{\pgfqpoint{5.760000in}{0.528000in}}%
\pgfusepath{stroke}%
\end{pgfscope}%
\begin{pgfscope}%
\pgfsetrectcap%
\pgfsetmiterjoin%
\pgfsetlinewidth{0.803000pt}%
\definecolor{currentstroke}{rgb}{0.000000,0.000000,0.000000}%
\pgfsetstrokecolor{currentstroke}%
\pgfsetdash{}{0pt}%
\pgfpathmoveto{\pgfqpoint{0.800000in}{4.224000in}}%
\pgfpathlineto{\pgfqpoint{5.760000in}{4.224000in}}%
\pgfusepath{stroke}%
\end{pgfscope}%
\begin{pgfscope}%
\definecolor{textcolor}{rgb}{0.000000,0.000000,0.000000}%
\pgfsetstrokecolor{textcolor}%
\pgfsetfillcolor{textcolor}%
\pgftext[x=3.280000in,y=4.260960in,,base]{\color{textcolor}\rmfamily\fontsize{10.000000}{12.000000}\selectfont Sorted order n=7.0E+08}%
\end{pgfscope}%
\begin{pgfscope}%
\pgfsetbuttcap%
\pgfsetmiterjoin%
\definecolor{currentfill}{rgb}{1.000000,1.000000,1.000000}%
\pgfsetfillcolor{currentfill}%
\pgfsetfillopacity{0.800000}%
\pgfsetlinewidth{1.003750pt}%
\definecolor{currentstroke}{rgb}{0.800000,0.800000,0.800000}%
\pgfsetstrokecolor{currentstroke}%
\pgfsetstrokeopacity{0.800000}%
\pgfsetdash{}{0pt}%
\pgfpathmoveto{\pgfqpoint{4.279722in}{2.364417in}}%
\pgfpathlineto{\pgfqpoint{5.662778in}{2.364417in}}%
\pgfpathquadraticcurveto{\pgfqpoint{5.690556in}{2.364417in}}{\pgfqpoint{5.690556in}{2.392194in}}%
\pgfpathlineto{\pgfqpoint{5.690556in}{4.126778in}}%
\pgfpathquadraticcurveto{\pgfqpoint{5.690556in}{4.154556in}}{\pgfqpoint{5.662778in}{4.154556in}}%
\pgfpathlineto{\pgfqpoint{4.279722in}{4.154556in}}%
\pgfpathquadraticcurveto{\pgfqpoint{4.251944in}{4.154556in}}{\pgfqpoint{4.251944in}{4.126778in}}%
\pgfpathlineto{\pgfqpoint{4.251944in}{2.392194in}}%
\pgfpathquadraticcurveto{\pgfqpoint{4.251944in}{2.364417in}}{\pgfqpoint{4.279722in}{2.364417in}}%
\pgfpathlineto{\pgfqpoint{4.279722in}{2.364417in}}%
\pgfpathclose%
\pgfusepath{stroke,fill}%
\end{pgfscope}%
\begin{pgfscope}%
\pgfsetrectcap%
\pgfsetroundjoin%
\pgfsetlinewidth{1.505625pt}%
\definecolor{currentstroke}{rgb}{0.121569,0.466667,0.705882}%
\pgfsetstrokecolor{currentstroke}%
\pgfsetdash{}{0pt}%
\pgfpathmoveto{\pgfqpoint{4.307500in}{4.050389in}}%
\pgfpathlineto{\pgfqpoint{4.446389in}{4.050389in}}%
\pgfpathlineto{\pgfqpoint{4.585278in}{4.050389in}}%
\pgfusepath{stroke}%
\end{pgfscope}%
\begin{pgfscope}%
\definecolor{textcolor}{rgb}{0.000000,0.000000,0.000000}%
\pgfsetstrokecolor{textcolor}%
\pgfsetfillcolor{textcolor}%
\pgftext[x=4.696389in,y=4.001778in,left,base]{\color{textcolor}\rmfamily\fontsize{10.000000}{12.000000}\selectfont cardchoose}%
\end{pgfscope}%
\begin{pgfscope}%
\pgfsetrectcap%
\pgfsetroundjoin%
\pgfsetlinewidth{1.505625pt}%
\definecolor{currentstroke}{rgb}{1.000000,0.498039,0.054902}%
\pgfsetstrokecolor{currentstroke}%
\pgfsetdash{}{0pt}%
\pgfpathmoveto{\pgfqpoint{4.307500in}{3.856083in}}%
\pgfpathlineto{\pgfqpoint{4.446389in}{3.856083in}}%
\pgfpathlineto{\pgfqpoint{4.585278in}{3.856083in}}%
\pgfusepath{stroke}%
\end{pgfscope}%
\begin{pgfscope}%
\definecolor{textcolor}{rgb}{0.000000,0.000000,0.000000}%
\pgfsetstrokecolor{textcolor}%
\pgfsetfillcolor{textcolor}%
\pgftext[x=4.696389in,y=3.807472in,left,base]{\color{textcolor}\rmfamily\fontsize{10.000000}{12.000000}\selectfont floydf2}%
\end{pgfscope}%
\begin{pgfscope}%
\pgfsetrectcap%
\pgfsetroundjoin%
\pgfsetlinewidth{1.505625pt}%
\definecolor{currentstroke}{rgb}{0.172549,0.627451,0.172549}%
\pgfsetstrokecolor{currentstroke}%
\pgfsetdash{}{0pt}%
\pgfpathmoveto{\pgfqpoint{4.307500in}{3.660944in}}%
\pgfpathlineto{\pgfqpoint{4.446389in}{3.660944in}}%
\pgfpathlineto{\pgfqpoint{4.585278in}{3.660944in}}%
\pgfusepath{stroke}%
\end{pgfscope}%
\begin{pgfscope}%
\definecolor{textcolor}{rgb}{0.000000,0.000000,0.000000}%
\pgfsetstrokecolor{textcolor}%
\pgfsetfillcolor{textcolor}%
\pgftext[x=4.696389in,y=3.612333in,left,base]{\color{textcolor}\rmfamily\fontsize{10.000000}{12.000000}\selectfont hsel}%
\end{pgfscope}%
\begin{pgfscope}%
\pgfsetrectcap%
\pgfsetroundjoin%
\pgfsetlinewidth{1.505625pt}%
\definecolor{currentstroke}{rgb}{0.839216,0.152941,0.156863}%
\pgfsetstrokecolor{currentstroke}%
\pgfsetdash{}{0pt}%
\pgfpathmoveto{\pgfqpoint{4.307500in}{3.467333in}}%
\pgfpathlineto{\pgfqpoint{4.446389in}{3.467333in}}%
\pgfpathlineto{\pgfqpoint{4.585278in}{3.467333in}}%
\pgfusepath{stroke}%
\end{pgfscope}%
\begin{pgfscope}%
\definecolor{textcolor}{rgb}{0.000000,0.000000,0.000000}%
\pgfsetstrokecolor{textcolor}%
\pgfsetfillcolor{textcolor}%
\pgftext[x=4.696389in,y=3.418722in,left,base]{\color{textcolor}\rmfamily\fontsize{10.000000}{12.000000}\selectfont iterativechoose}%
\end{pgfscope}%
\begin{pgfscope}%
\pgfsetrectcap%
\pgfsetroundjoin%
\pgfsetlinewidth{1.505625pt}%
\definecolor{currentstroke}{rgb}{0.580392,0.403922,0.741176}%
\pgfsetstrokecolor{currentstroke}%
\pgfsetdash{}{0pt}%
\pgfpathmoveto{\pgfqpoint{4.307500in}{3.273028in}}%
\pgfpathlineto{\pgfqpoint{4.446389in}{3.273028in}}%
\pgfpathlineto{\pgfqpoint{4.585278in}{3.273028in}}%
\pgfusepath{stroke}%
\end{pgfscope}%
\begin{pgfscope}%
\definecolor{textcolor}{rgb}{0.000000,0.000000,0.000000}%
\pgfsetstrokecolor{textcolor}%
\pgfsetfillcolor{textcolor}%
\pgftext[x=4.696389in,y=3.224417in,left,base]{\color{textcolor}\rmfamily\fontsize{10.000000}{12.000000}\selectfont quadraticf2}%
\end{pgfscope}%
\begin{pgfscope}%
\pgfsetrectcap%
\pgfsetroundjoin%
\pgfsetlinewidth{1.505625pt}%
\definecolor{currentstroke}{rgb}{0.549020,0.337255,0.294118}%
\pgfsetstrokecolor{currentstroke}%
\pgfsetdash{}{0pt}%
\pgfpathmoveto{\pgfqpoint{4.307500in}{3.079417in}}%
\pgfpathlineto{\pgfqpoint{4.446389in}{3.079417in}}%
\pgfpathlineto{\pgfqpoint{4.585278in}{3.079417in}}%
\pgfusepath{stroke}%
\end{pgfscope}%
\begin{pgfscope}%
\definecolor{textcolor}{rgb}{0.000000,0.000000,0.000000}%
\pgfsetstrokecolor{textcolor}%
\pgfsetfillcolor{textcolor}%
\pgftext[x=4.696389in,y=3.030806in,left,base]{\color{textcolor}\rmfamily\fontsize{10.000000}{12.000000}\selectfont quadraticreject}%
\end{pgfscope}%
\begin{pgfscope}%
\pgfsetrectcap%
\pgfsetroundjoin%
\pgfsetlinewidth{1.505625pt}%
\definecolor{currentstroke}{rgb}{0.890196,0.466667,0.760784}%
\pgfsetstrokecolor{currentstroke}%
\pgfsetdash{}{0pt}%
\pgfpathmoveto{\pgfqpoint{4.307500in}{2.884278in}}%
\pgfpathlineto{\pgfqpoint{4.446389in}{2.884278in}}%
\pgfpathlineto{\pgfqpoint{4.585278in}{2.884278in}}%
\pgfusepath{stroke}%
\end{pgfscope}%
\begin{pgfscope}%
\definecolor{textcolor}{rgb}{0.000000,0.000000,0.000000}%
\pgfsetstrokecolor{textcolor}%
\pgfsetfillcolor{textcolor}%
\pgftext[x=4.696389in,y=2.835667in,left,base]{\color{textcolor}\rmfamily\fontsize{10.000000}{12.000000}\selectfont rejectionsample}%
\end{pgfscope}%
\begin{pgfscope}%
\pgfsetrectcap%
\pgfsetroundjoin%
\pgfsetlinewidth{1.505625pt}%
\definecolor{currentstroke}{rgb}{0.498039,0.498039,0.498039}%
\pgfsetstrokecolor{currentstroke}%
\pgfsetdash{}{0pt}%
\pgfpathmoveto{\pgfqpoint{4.307500in}{2.689139in}}%
\pgfpathlineto{\pgfqpoint{4.446389in}{2.689139in}}%
\pgfpathlineto{\pgfqpoint{4.585278in}{2.689139in}}%
\pgfusepath{stroke}%
\end{pgfscope}%
\begin{pgfscope}%
\definecolor{textcolor}{rgb}{0.000000,0.000000,0.000000}%
\pgfsetstrokecolor{textcolor}%
\pgfsetfillcolor{textcolor}%
\pgftext[x=4.696389in,y=2.640528in,left,base]{\color{textcolor}\rmfamily\fontsize{10.000000}{12.000000}\selectfont reservoirsample}%
\end{pgfscope}%
\begin{pgfscope}%
\pgfsetrectcap%
\pgfsetroundjoin%
\pgfsetlinewidth{1.505625pt}%
\definecolor{currentstroke}{rgb}{0.737255,0.741176,0.133333}%
\pgfsetstrokecolor{currentstroke}%
\pgfsetdash{}{0pt}%
\pgfpathmoveto{\pgfqpoint{4.307500in}{2.495528in}}%
\pgfpathlineto{\pgfqpoint{4.446389in}{2.495528in}}%
\pgfpathlineto{\pgfqpoint{4.585278in}{2.495528in}}%
\pgfusepath{stroke}%
\end{pgfscope}%
\begin{pgfscope}%
\definecolor{textcolor}{rgb}{0.000000,0.000000,0.000000}%
\pgfsetstrokecolor{textcolor}%
\pgfsetfillcolor{textcolor}%
\pgftext[x=4.696389in,y=2.446917in,left,base]{\color{textcolor}\rmfamily\fontsize{10.000000}{12.000000}\selectfont select}%
\end{pgfscope}%
\end{pgfpicture}%
\makeatother%
\endgroup%

    \caption{Sorted order, \(n=7.0 \times 10^{9}\)}
    \label{sortedlargen}
\end{figure}

Our algorithm outperforms all hash- and set-based algorithms for all
values of \(n\) and \(k\) tested. In theory, the \(\bigO{k}\)
performance of these algorithms compared to the \(\bigO{k \log k}\)
performance of our algorithm should mean that there is a value of
\(k\) at which these algorithms perform better. However, the poor
cache coherence of these algorithms means that their performance more
closely resembles \(\bigO{k \log k}\) in practice. For very large
\(k\), a parallel algorithm such as \cite{sandersetal} should
generally be used; this algorithm calls a sequential algorithm at the
leaves of the parallel tree it builds.

Quadratic rejection and quadratic F2 both outperform our algorithm for
small \(k\), while SELECT, reservoir sampling, and iterative choosing
all outperform our algorithm for small \(n/k\). However all of these
behave pathologically for other values of \(n, k\); ours is the
fastest of all of the algorithms that perform reasonably for all
values of \(n, k\).

A good implementation will implement multiple algorithms, and choose
which one to use based on the values of \(n\) and \(k\). For random
order output, a combination of quadratic F2, our algorithm, and SELECT
is likely to be near-optimal for all values, or if allocating extra
memory on demand is not an option then reservoir sampling can be
substituted for SELECT. For sorted order output, a combination of our
algorithm and iterative choosing should suffice. Based on a
least-squares curve fitting, the runtime estimates from
\autoref{randomtable} and \autoref{sortedtable} can be used to select
which algorithm to use.

\begin{table}
    \begin{tabular}{ lr }
        Quadratic F2 & \((1.551 \times 10^{-10})k^2 + (1.617 \times 10^{-8})k - 1.195 \times 10^{-7}\)\\
Our algorithm & \((4.675 \times 10^{-9})(k \log k) + (8.123 \times 10^{-9})k - 3.299 \times 10^{-9}\)\\
SELECT & \((3.204 \times 10^{-10})n + (4.683 \times 10^{-9})k - 2.895 \times 10^{-8}\)\\
Reservoir sampling & \((4.372 \times 10^{-9})n + (5.517 \times 10^{-9})k - 2.982 \times 10^{-8}\)\\

    \end{tabular}
    \caption{Time estimates for random order algorithms}\label{randomtable}
\end{table}

\begin{table}
    \begin{tabular}{ lr } 
        Our algorithm & \((4.360 \times 10^{-9})(k \log k) + (5.911 \times 10^{-9})k - 2.552 \times 10^{-8}\)\\
Iterative choosing & \((4.077 \times 10^{-9})n + (8.378 \times 10^{-9})k + 4.224 \times 10^{-9}\)\\

    \end{tabular}
    \caption{Time estimates for sorted order algorithms}\label{sortedtable}
\end{table}

\printbibliography

\end{document}
