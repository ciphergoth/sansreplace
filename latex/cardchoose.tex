% Copyright 2018 Google LLC
%
% Licensed under the Apache License, Version 2.0 (the "License");
% you may not use this file except in compliance with the License.
% You may obtain a copy of the License at
%
%     https://www.apache.org/licenses/LICENSE-2.0
%
% Unless required by applicable law or agreed to in writing, software
% distributed under the License is distributed on an "AS IS" BASIS,
% WITHOUT WARRANTIES OR CONDITIONS OF ANY KIND, either express or implied.
% See the License for the specific language governing permissions and
% limitations under the License.

%!BIB program = biber
%!TeX program = lualatex
%!TeX spellcheck = en-US

\documentclass[letterpaper,luatex,11pt]{article}

\usepackage{fontspec}
\usepackage[hmargin=4cm,vmargin=3cm,nohead]{geometry}
\usepackage{parskip}
\usepackage[style]{abstract}
\usepackage{titling}
\usepackage{sectsty}
\usepackage{authblk}
\usepackage{floatrow}
\usepackage[svgnames]{xcolor}
\usepackage[breaklinks,colorlinks,urlcolor=DarkBlue,linkcolor=DarkRed,citecolor=DarkGreen]{hyperref}
\usepackage[style=alphabetic,backend=biber]{biblatex}
\usepackage{amsmath}
\usepackage{amssymb}
\usepackage{algorithm}
\usepackage{algpseudocode}
\usepackage[landau,operators,probability,sets]{cryptocode}
\usepackage[outputdir=build/latex.out,cachedir=build/_minted-cardchoose]{minted}

% https://tex.stackexchange.com/a/350287
\newcommand{\multichoose}[2]{
\left.\mathchoice
  {\left(\kern-0.48em\binom{#1}{#2}\kern-0.48em\right)}
  {\big(\kern-0.30em\binom{\smash{#1}}{\smash{#2}}\kern-0.30em\big)}
  {\left(\kern-0.30em\binom{\smash{#1}}{\smash{#2}}\kern-0.30em\right)}
  {\left(\kern-0.30em\binom{\smash{#1}}{\smash{#2}}\kern-0.30em\right)}
\right.}

\newcommand*{\defeq}{\stackrel{\text{def}}{=}}

\addbibresource{bib.bib}

\setmainfont{TeX Gyre Pagella}
\setsansfont{TeX Gyre Heros}
\setmathrm{Latin Modern Roman}
\defaultfontfeatures{}
\setmonofont{TeX Gyre Cursor}[Ligatures={NoCommon,NoRequired,NoContextual},Scale=0.9]

\pretitle{\begin{center}\LARGE\bfseries\sffamily} % sf title
\renewcommand{\abstitlestyle}{\sffamily\bfseries\centering} % sf abstract
\allsectionsfont{\sffamily} % sf sections

\raggedright
\raggedbottom

\DeclareMathOperator{\Supp}{Supp}

\title{A new algorithm for sampling without replacement}
\author{Paul~Crowley}
\affil{Google LLC}

\begin{document}
\maketitle
\begin{abstract}
    The literature contains several algorithms for fairly choosing $k$ distinct natural numbers below \(n\).
    However, they either require a hash-based data structure such as a set or dictionary, or
    show asymptotically poor performance for some values of $k, n$. We present here an algorithm
    that requires no such data structure or auxiliary storage, only an integer sort taking
    $\bigO{k \log k}$ time; the algorithm is based on a method of fair multiset choosing.
    In our benchmarks the algorithm outperforms all we compare it to wherever
    \(k > 100\) and \(n > 100k\), and has acceptable performance for all values of \(k, n\).
\end{abstract}

\section{Introduction}

The problem of choosing a subset of $\{0, 1, \ldots, n-1\}$ such that every \(k\)-element subset is
equally likely arises in several contexts.
Several programming languages include facilities for this in their standard libraries,
for example Python's \mintinline{python}{random.sample()}
or Rust's \mintinline{rust}{rand::seq::index::sample}.
In \autoref{priorwork} we discuss several known algorithms to this end; these all have
their strengths, but all suffer from one of two disadvantages:
\begin{itemize}
    \item Some are fast for certain
    values of \(k, n\), e.g. where \(k\) is small or where \(\frac{n}{k}\) is small, but very slow
    for other values.
    \item Others achieve much better asymptotic performance, but at the cost of
    updating a hash-based data structure such as a set or dictionary for each of the \(k\) values
    generated, adding a significant constant multiplier to the overall runtime as well as memory cost.
\end{itemize}

We here present an algorithm which requires no storage beyond the array in which the result
is written but achieves a \(\bigO{k \log k}\) runtime; the extra \(\log k\) term is due to
an integer sort over the results array.

In Python the algorithm works as follows:

\inputminted{Python}{code/cardchoose.py}

\section{Prior work}\label{priorwork}
Let \(\NN_{<n}\) denote the \(n\)-element set \(\{0, 1, \ldots, n-1\}\). Someone who calls a routine that promises a fairly drawn \(k\)-element subset of
\(\NN_{<n}\) might be expecting it to return one of three different things:

\begin{itemize}
    \item an array of integers in sorted order;
    \item an array of integers in fair random order;
    \item a set of integers in some kind of efficient set structure (e.g. a hash table).
\end{itemize}

Both Python's \mintinline{python}{random.sample()} and 
Rust's \mintinline{rust}{rand::seq::index::sample} return in a fair random order.

A routine which returns one of these can straightforwardly be converted into 
a routine that returns another, using e.g. a sort or a Fisher-Yates shuffle as appropriate.
However each of these algorithms returns one of the above types ``naturally'', and
so returning a different one will take an extra step, which means that which algorithm
is most efficient can depend on which of the above three the caller wants to receive.

\begin{center}
    \begin{tabular}{l|l|l|l}
    Algorithm & Order & Data structure & Time \\
    \hline
    Quadratic rejection sampling & Random &  & \(k^2\) \\
    Set-based rejection sampling & Random & Set & \(k\) \\
    Algorithm S & Sorted &  & \(n\) \\
    Algorithm R & Random &  & \(n\) \\
    SELECT & Random & \(n\)-element array & \(n\) \\
    HSEL & Random & Dictionary & \(k\) \\
    Floyd's F2 &  & Set & \(k\) \\
    Quadratic F2 &  & & \(k^2\) \\
    Our work & Sorted &  & \(k \log k\)
    \end{tabular}
\end{center}

\subsection{Quadratic rejection sampling}
\inputminted{Python}{code/quadraticreject.py}
Start with an empty array of integers. Generate an integer
\(0 \leq x < n\), and check if \(x\) is already present in the array
by iterating through the array checking every element. If it is not present,
append it to the array. Repeat this process until the array is of the desired length.

\subsection{Set-based rejection sampling}
\inputminted{Python}{code/rejectionsample.py}
As above, but use a data structure representing sets with an efficient membership test.
\cite{goodman1985introduction}

\subsection{Algorithm S}
\inputminted{Python}{code/iterativechoose.py}
Iterate in order through each candidate to add to the list,
calculate for each the probability it should be part of the list given
the number of items added so far and the number remaining, and add
it with that probability.
\cite{taocp2}
\cite[Method 1, p.391]{fanetal}

\subsection{Algorithm R}
\inputminted{Python}{code/reservoirsample.py}
Initialize a \(k\)-element array with the elements of \(\NN_{<k}\)
in random order. Iterate over the elements of \(\NN_{<n} \setminus \NN_{<k}\),
and replace a random element of the array with each element with the
appropriate probability.
\cite{taocp2}

\subsection{SELECT}
\inputminted{Python}{code/select.py}
Use a Fisher-Yates shuffle to randomize only the first \(k\) items
of an array of the elements of \(\NN_{<n}\). \cite{goodman1985introduction}

\subsection{HSEL}
\inputminted{Python}{code/hsel.py}
As above, except that instead of allocating and initializing an \(n\)-element array,
we use a ``virtual'' array, with a hash table storing only the modified elements.
\cite{hsel}

\subsection{F2}
\inputminted{Python}{code/floydf2.py}
A twist on set-based rejection that uses compensatingly biased
samples to avoid multiple tests.
\cite{floydf2}

\subsection{Quadratic F2}
\inputminted{Python}{code/quadraticf2.py}
Quadratic variant on Floyd's F2 which needs no external data structure.
\cite{floydf2}

\section{Notation}

Let \(\NN_{<n} \defeq \{i \in \NN: i < n\}\), the set of natural numbers less than \(n\)
where \(\NN = \{0, 1, 2, \ldots\}\). 
Let \(\binom{S}{k} \defeq \{s \subseteq S: |s| = k\}\), the set of all \(k\)-element subsets
of \(S\); note that \(\left|\binom{S}{k}\right| = \binom{|S|}{k}\).

A \emph{multiset} is an extension of a set in which elements can appear more than once;
we use brackets to delimit the elements of a multiset, so
$[0, 1, 1]$ is the same multiset as $[1, 0, 1]$ but distinct from $[0, 0, 1]$.
We represent multisets as functions $m: U \rightarrow \NN$; the set $U$ is the \emph{universe},
and in what follows we consider only finite universes. For any $y \in U$ we call
$m(y)$ the \emph{multiplicity} of $y$ in $m$.
A multiset has a \emph{cardinality} $|m| \defeq \sum_{x \in U} m(x)$
and a \emph{support} set $\Supp(m) \defeq \{x \in U: m(x) > 0\}$.

Where $U$ is clear from context, for any set $S \subseteq U$ we consider
$\overline{S}$ to be $S$ viewed as a multiset, i.e. the multiset
such that $\Supp(\overline{S}) = S$ and $|\overline{S}| = |S|$:
\begin{displaymath}
    \overline{S}(x) =
    \begin{cases}
        1 & \text{if $x \in S$} \\
        0 & \text{otherwise} \\
    \end{cases}
\end{displaymath}

The sum of multisets $m_1 \uplus m_2$ is the multiset such that
$(m_1 \uplus m_2)(x) = m_1(x) + m_2(x)$ for all $x \in U$.
$m = m_1 \ominus m_2$ is the unique multiset such that $m_1 = m \uplus m_2$,
and is defined only if this exists.

We define random sampling from a multiset to be analogous to drawing from a set,
where each element's probability of being drawn is proportional to its multiplicity:
$\prob{x = y |x \sample m} = \frac{m(y)}{|m|}$.

Define $\multichoose{U}{k} \defeq \{m \in U \rightarrow \NN: |m| = k\}$ the set of multisets over
universe \(U\) of cardinality \(k\), and \(\multichoose{n}{k} \defeq \left|\multichoose{\NN_{<n}}{k}\right|\)
the number of distinct \(k\)-cardinality multisets over an \(n\)-element universe.

\section{Intuition behind the algorithm}

Following \cite{feller}, we can represent an element of \(\multichoose{\NN_{<7}}{8}\)
such as \([0,0,0,1,5,5,5,5]\) using six bars and eight stars:

\begin{displaymath}
    \begin{array}{ c c c c c c c c c c c c c c c }
        \star & \star & \star & \big| & \star & \big| & \big| & \big| & \big| & \star & \star & \star & \star & \big| \\
    \end{array}
\end{displaymath}

The six bars divide the line into seven ``bins'', one for each element of \(\NN_{<7}\); the
number of stars in each bin indicates the multiplicity of that element in our multiset.
With this technique, we define a bijection between \(\multichoose{\NN_{<n}}{k}\) and
\(\binom{\NN_{<n + k - 1}}{k}\) wherever \(n > 0\), from which we infer that
\(\multichoose{n}{k} = \binom{n + k - 1}{k}\) where \(n > 0\).

Suppose now that we want to choose 6 integers from the range $\NN_{<11}$ 
at random. As per the above we see that this 
is equivalent to choosing a multiset from \(\multichoose{\NN_{<6}}{6}\)
or a way of arranging 6 stars and 5 bars into a sequence.
To choose fairly, we start with a sequence of five bars \(|||||\) and insert six stars,
one after another, in randomly chosen positions.

For the first star, there are six possible places it can go, and we choose one at random:
\(|||\star||\). We record that it has three bars to its left \([3]\).

There are now six items in the sequence, and thus seven possible places to place the second star.
In two of those seven cases---before the existing star, and after it---it will have three bars
to its left. Let's suppose we choose the first position: \(\star|||\star||\). We append
the number of bars to the left of the new star to our record, which becomes \([3, 0]\).
The positions have changed from \(\{3\}\)
to \(\{4, 0\}\) but because we're not recording positions, only bars to the left, we don't need to
update the first entry.

We place three more stars in random positions, ending up with \(\star\star\star|\star||\star||\)
and a record of \([3, 0, 0, 1, 0]\). Now there's one star left to place; there is one
position it can be placed after the last bar, but four before the first bar, so it is four times
more likely to be placed before the first bar than after the last. Let's suppose it's placed
at the fifth position: \(\star\star\star|\star\star||\star||\), \([3, 0, 0, 1, 0, 1]\) We now
want to know the position of each star; we find this by sorting the list \([0, 0, 0, 1, 1, 3]\)
and adding to each entry its index so that the value reflects the stars as well as the bars to its
left, returning the answer $\{0, 1, 2, 4, 5, 8\}$

\begin{displaymath}
\begin{array}{ c c c c c c c c c c c }
    0 & 0 & 0 & & 1 & 1 & & & 3 & & \\
    \star & \star & \star & \big| & \star & \star & \big| & \big| & \star & \big| & \big| \\
    0 & 1 & 2 & & 4 & 5 & & & 8 & & \\
\end{array}
\end{displaymath}

Thus to get a sequence without duplicates, we start with a procedure that deliberately
biases towards duplicates.

\section{Proof}

\subsection{Multiset choosing}

We consider the problem of choosing an element from $\multichoose{U}{k}$ fairly.
For example, $\multichoose{\NN_{<2}}{3} = \{[0, 0, 0], [0, 0, 1], [0, 1, 1], [1, 1, 1]\}$; 
for each of these four, our algorithm should output it with probability $\frac{1}{4}$.
If we choose three independent elements from $U$ and add them together to make a multiset,
our answer will favour multisets with lower multiplicities; in accordance
with the binomial theorem, $[0, 0, 0]$ will be drawn with probability $\frac{1}{8}$, while
$[0, 0, 1]$ will be drawn with probability $\frac{3}{8}$, reflecting the three ways this multiset
can be written as a sequence.

%It is straightforward to show by
%induction that in general, this naive method chooses $m \in \multichoose{U}{k}$
%with probability

%\begin{displaymath}
%    \frac{k!}{{|U|}^k \prod_{x \in U} m(x)!}
%\end{displaymath}

\begin{algorithm}
\caption{Fair multiset choosing}
\begin{algorithmic}[0]
\Procedure{ChooseMultiset}{$U, k$}
    \If{k = 0}
        \State \textbf{return} $\overline{\varnothing}$
    \Else
        \State $m' \leftarrow \textproc{ChooseMultiset}(U, k-1)$
        \State $x \sample \overline{U} \uplus m'$
        \State \textbf{return} $m' \uplus \overline{\{x\}}$
    \EndIf
\EndProcedure
\end{algorithmic}
\end{algorithm}

To address this, in \textproc{ChooseMultiset}
we introduce a counter-bias in the selection of $x$, which favours duplicates.
\textproc{ChooseMultiset} is trivially fair for $k = 0$, so
we assume it is fair for $k - 1$ and proceed by induction.
For a multiset $m \in \multichoose{U}{k}$:

\begin{align*}
    &\prob{m' = m | m' \leftarrow \textproc{ChooseMultiset}(U, k)}
    \\
    =&
    \prob{m' \uplus \overline{\{x\}} = m
        | m' \leftarrow \textproc{ChooseMultiset}(U, k-1), x \sample \overline{U} \uplus m'}
    \\
    =&
    \prob{m' \uplus \overline{\{x\}} = m
        | m' \sample \multichoose{U}{k - 1}, x \sample \overline{U} \uplus m'}
    \\
    =&
    \sum_{y \in \Supp(m)}
    \prob{m' = m \ominus \overline{\{y\}} | m' \sample \multichoose{U}{k - 1}}
    \prob{x = y | x \sample \overline{U} \uplus (m \ominus \overline{\{y\}})}
    \\
    =&
    \sum_{y \in \Supp(m)}
    \frac{1}{\multichoose{|U|}{k - 1}}
    \frac{(U \uplus (m \ominus \overline{\{y\}}))(y)}{|U \uplus (m \ominus \overline{\{y\}})|}
    \\
    =&
    \frac{1}{\multichoose{|U|}{k - 1}}
    \sum_{y \in \Supp(m)}
    \frac{1 + (m \ominus \overline{\{y\}})(y)}{|U| + |m \ominus \overline{\{y\}}|}
    \\
    =&
    \frac{1}{\binom{|U| + k - 2}{k-1}}
    \sum_{y \in \Supp(m)}
    \frac{m(y)}{|U| + k -1}
    \\
    =& \frac{k}{(|U| + k -1)\binom{|U| + k - 2}{k-1}}
    \\
    =& \frac{1}{\binom{|U| + k - 1}{k}}
    \\
    =& \frac{1}{\multichoose{|U|}{k}}
\end{align*}

This algorithm is straightforward to implement. This Python implementation takes integers
$n, k$ and returns a sorted list of integers in \(\NN_{<n}\).

\inputminted{Python}{code/choose_multiset.py}

\subsection{Multisets and choices}

To generate a random \(k\)-element subset of \(\NN_{<n}\),
we can apply this method to generate a random multiset from the universe \(\NN_{<n-k + 1}\)
and use the ``stars and bars'' bijection to convert to the desired subset.

Our implementation of \textproc{ChooseMultiset}
represents its result in \(\multichoose{\NN_{<n-k + 1}}{k}\) as a sorted list of integers.
In ``stars and bars'' representation,
each entry in the list represents a star, and the integer is the number of bars to its left.
Converting this to a sorted \(k\)-element subset of \(\NN_{<n}\) simply means adding to each
the number of stars to its left, which is equal to its position in the sequence; the Python code
below returns a sorted list of $k$ distinct integers in \(\NN_{<n}\) fairly among all ways
of doing so.

\inputminted{Python}{code/choose_binom.py}

\section{Benchmarks}

C++ implementations of all of these algorithms can be found at 
\url{https://github.com/ciphergoth/sansreplace/cpp}. I benchmarked
all of these for a variety of values of values of \(n\) and \(k\),
for both sorted output and random output. 

\printbibliography

\end{document}
